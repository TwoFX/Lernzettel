\documentclass[a4paper]{article}

\usepackage[l2tabu, orthodox]{nag}

\usepackage[utf8]{inputenc}
\usepackage[T1]{fontenc}

\usepackage[ngerman]{babel}

\usepackage{amsmath}
\usepackage{amssymb}
\usepackage{mathtools}
\usepackage{physics}

\usepackage[framed]{ntheorem}

\usepackage{csquotes}
\usepackage{lmodern}
\usepackage{microtype}
\usepackage{enumitem}

\usepackage{parskip}

\usepackage[hidelinks]{hyperref}
\usepackage[left=1.8cm, right=1.8cm, top=1.8cm, bottom=2.5cm]{geometry}

\newcounter{Sec}

\theoremstyle{marginbreak}
\theorembodyfont{\normalfont}
\newtheorem{definition}{Definition}[Sec]
\newtheorem{satz}[definition]{Satz}
\newtheorem{defsatz}[definition]{Definition und Satz}
\newtheorem{verfahren}[definition]{Verfahren}
\newtheorem{defver}[definition]{Definition und Verfahren}
\newtheorem{defsatzver}[definition]{Definition, Satz und Verfahren}
\newtheorem{satzver}[definition]{Satz und Verfahren}
\newtheorem{folgerung}[definition]{Folgerung}

\MakeOuterQuote{"}

\newcommand{\sep}{%
	\rule{\textwidth}{0.3pt}%
	\stepcounter{Sec}%
	}
\newcommand{\defiff}{\mathrel{\vcentcolon\Longleftrightarrow}}
\newcommand\defas\coloneqq

\DeclarePairedDelimiterX\set[1]\lbrace\rbrace{\def\given{\;\delimsize\vert\;}#1}
\DeclarePairedDelimiter\scp{\langle}{\rangle}
\newcommand\restr[2]{{#1_{\mkern 1mu \vrule height 2ex\mkern2mu #2}}}

\newcommand\qf{(b_j, c_j)_{j=1}^s}
\newcommand\poly{\mathcal{P}}
\newcommand\qs{\sum_{j=1}^sb_jg(c_j)}

\begin{document}
	\textsc{Numerische Mathematik I}

	\sep
	\begin{defsatz}[Orthogonalpolynome]
		\begin{enumerate}[label=(\alph*)]
			\item
				$\omega\colon(a, b)\to\mathbb{R}_{>0}$ heißt Gewichtsfunktion, falls für alle $k\in\mathbb{N}_0$ das Integral
				\[
					\mu_k\defas\int_a^b\omega(x)\abs{x}^k\dd{x}
				\]
				existiert.
			\item Auf dem Vektorraum der stetigen $\omega$-gewichtet quadratisch integrierbaren Funktionen
				gibt es das Skalarprodukt
				\[
					\scp{f, g}\defas \int_a^b\omega(x)f(x)g(x)\dd{x}.
				\]
			\item Es gibt eine, bis auf Skalierung eindeutige, Folge von Polynomen $\phi_0, \ldots$ mit
				\[
					\phi_k\in\poly_k\setminus\poly_{k-1},\qquad \forall q\in\poly_{k-1}: \scp{\phi_k, q}=0.
				\]
				$\phi_k$ heißt das $k$-te Orthogonalpolynom bezüglich $\omega$. Seine Nullstellen sind reel, einfach, und
				liegen in $(a, b)$.
			\item Die Orthogonalpolynnome bezüglich $\omega\equiv 1$ heißen Legendre-Polynome.
		\end{enumerate}

	\end{defsatz}
	\begin{defsatz}[Quadraturformeln]
		\begin{enumerate}[label=(\alph*)]
			\item
				Eine $s$-stufige Quadraturformal $\qf$ mit Knoten $(c_j)$ und Gewichten $(b_j)$ ist gegeben durch
				\[
					I(g) \defas \int_0^1g(t)\dd{t}\approx\qs.
				\]
			\item
				Eine Quadraturformel $\qf$ hat Ordnung $p$ genau dann wenn $\forall g\in\poly_{p-1}: I(g) = \qs$.
			\item
				Eine Quadraturformel $\qf$ hat Ordnung $p$ genau dann wenn
				\[
					\forall q\in\set{1,\ldots,p}: \sum_{j=1}^s b_j c_j^{q-1} = \frac{1}{q}.
				\]
				Die Ordnung einer Quatraturformel ist höchstens $2s$.
			\item $\qf$ heißt symmetrisch genau dann wenn $\forall j\in\set{1,\ldots,s}: c_j=1-c_{s+1-j}\wedge b_j=b_{s+1-j}$.
			\item Die maximale Ordnung einer symmetrischen Quadraturformel ist gerade.
			\item Für feste $c_1<\dots<c_s$ existieren eindeutige $b_1,\ldots, b_s$ sodass $\qf$ Ordnung $p\geq s$ hat. Genauer gilt
				\[
					b_j=\int_0^1\ell_j(t)\dd{t}, \qquad
					\ell_j(t) = \frac{\prod_{k\neq j}(t-c_k)}{\prod_{k\neq j}(c_j-c_k)}.
				\]
			\item Für $c_1<\dots<c_j$ heißt $\poly_s\ni M(t)\defas \prod_{j=1}^s (t-c_j)$ das Knotenpolynom.
			\item $\qf$ mit Ordnung $p\geq s$ hat Ordnung $s+m$ genau dann, wenn
			\[
				\forall g\in\poly_{m-1}: \int_0^1 M(t)g(t)\dd{t}=0.
			\]
			\item Es existiert eine eindeutige Quadraturformel der Ordnung $2s$. Sie hat die
				Knoten $c_i = \frac{1}{2}(1+\lambda_i)$, wobei $\lambda_i$ die Nullstellen des $s$-ten
				Legendrepolynoms bezeichnet.
		\end{enumerate}
		% TODO: Simpsonregel etc.
	\end{defsatz}
	\begin{defsatz}[Quadraturfehler]
		\begin{enumerate}[label=(\alph*)]
			\item $E(g)\defas \int_0^1g(t)\dd{t}-\qs$ heißt Fehlerfunktional.
			\item Der Peano-Kern ist gegeben durch
			\[
				K_p(\tau)\defas E(t\mapsto \sigma_{p,\tau}(t)), \qquad \sigma_{p,\tau}(t)\defas\frac{(t-\tau)^{p-1}_+}{(p-1)!}=
				\begin{cases}
					\frac{(t-\tau)^{p-1}}{(p-1)!}, &t>\tau\\
					0, &t\leq\tau.
				\end{cases}
			\]
			\item Hat $\qf$ Ordnung $p$ und $g$ $p$-mal stetig differenzierbar, so haben wir
			\[
				E(g)=\int_0^1 K_p(\tau)g^{(p)}(\tau)\dd{\tau}.
			\]
			\item Falls $K_p(\tau)\geq 0$ oder $K_p(\tau)\leq 0$ für alle $\tau\in(0, 1)$, so
			\[
				\exists\tau^*\in[0, 1]:E(g) = g^{(p)}(\tau^*)\int_0^1K_p(\tau)\dd{\tau}.
			\]
			\item Genaue Fehlerschranken $\leadsto$ Skript
		\end{enumerate}
	\end{defsatz}
	% TODO: Adaptives Programm? \eps-Algorithmus?
\end{document}

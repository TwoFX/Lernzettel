\documentclass[a4paper]{article}

\usepackage[l2tabu, orthodox]{nag}

\usepackage[utf8]{inputenc}
\usepackage[T1]{fontenc}

\usepackage[ngerman]{babel}

\usepackage{amsmath}
\usepackage{amssymb}
%\usepackage{amsthm}
\usepackage{mathtools}
\usepackage{physics}

\usepackage[framed]{ntheorem}

\usepackage{csquotes}
\usepackage{lmodern}
\usepackage{microtype}
\usepackage{enumitem}
\usepackage{stmaryrd}

\usepackage{parskip}
\usepackage{multicol}

\usepackage[left=1.8cm, right=1.8cm, top=1.8cm, bottom=2.5cm]{geometry}

\newcounter{Sec}

\theoremstyle{marginbreak}
\theorembodyfont{\normalfont}
\newtheorem{definition}{Definition}[Sec]
\newtheorem{satz}[definition]{Satz}
\newtheorem{defsatz}[definition]{Definition und Satz}
\newtheorem{verfahren}[definition]{Verfahren}
\newtheorem{defver}[definition]{Definition und Verfahren}
\newtheorem{defsatzver}[definition]{Definition, Satz und Verfahren}
\newtheorem{satzver}[definition]{Satz und Verfahren}

\MakeOuterQuote{"}

\newcommand{\sep}{%
	\rule{\textwidth}{0.3pt}%
	\stepcounter{Sec}%
	}
\newcommand{\defiff}{:\Longleftrightarrow}
\DeclarePairedDelimiterX\set[1]\lbrace\rbrace{\def\given{\;\delimsize\vert\;}#1}

\begin{document}
	\textsc{Analysis I}

	\sep
	\begin{defsatz}
		Zu $\varnothing\neq M\subseteq\mathbb{R}$ heißt $\gamma\in\mathbb{R}$
		\begin{description}
			\item[Obere Schranke] $\defiff \forall x\in M: \gamma\geq x$
			\item[Supremum] $\defiff \gamma~\text{ist OS} \wedge \forall~\text{OS $x$ von $M$}: \gamma\leq x$
			\item[Maximum] $\defiff \gamma = \sup M\in M$
		\end{description}
		Besitzt $M$ eine obere Schranke, besitzt es auch ein Supremum.
	\end{defsatz}
	\begin{satz}
		Für $\varnothing\neq M\subseteq\mathbb{R}$ und obere Schranke $\gamma$ von $M$ gilt
		$\gamma = \sup M\iff\forall\varepsilon>0~\exists x\in M:x>\gamma-\varepsilon$.
	\end{satz}
	\sep
	\begin{satz}[Betragssätze]
		$\forall a,b\in\mathbb{R}:$
		\begin{enumerate}[label=(\alph*)]
			\item $\abs{ab}=\abs{a}\abs{b}$
			\item $\pm a\leq\abs{a}$
			\item $\abs{a+b}\leq\abs{a}+\abs{b}$
		\end{enumerate}
	\end{satz}
	\begin{defsatz}
		\begin{enumerate}[label=(\alph*)]
			\item Für $a\in\mathbb{R}$ ist $[a]$ mit $[a]\in\mathbb{Z}\wedge [a]\leq a<[a]+1$ existent und eindeutig.
			\item Für $x,y\in\mathbb{R}$ mit $x<y$ existiert $r\in\mathbb{Q}$ mit $x<r<y$.
		\end{enumerate}
	\end{defsatz}
	\sep
	\begin{definition}[Folge]
		Eine Abbildung $\mathbb{N}\to B$ heißt Folge in $B$.
	\end{definition}
	\begin{defsatz}
		Eine nichtleere Menge $B$ heißt
		\begin{description}
			\item[endlich] $\defiff\exists n\in\mathbb{N},f\colon \set{1,\ldots,n}\to B~\text{surjektiv}$
			\item[abzählbar] $\defiff\exists f\colon\mathbb{N}\to B~\text{surjektiv}$
		\end{description}
		Die abzählbaren Mengen sind über dem kartesischen Produkt und der
		Vereinigung abzählbar vieler Mengen abgeschlossen.
	\end{defsatz}
	\sep
	\begin{defsatz}[Binomischer Lehrsatz]
		Für $n, k\in\mathbb{N}$ ist $\binom{n}{k}\coloneqq\frac{n!}{k!(n-k)!}$.
		Für $a, b\in\mathbb{R}, n\in\mathbb{N}$ gilt
		\[(a+b)^n = \sum_{k=0}^n\binom{n}{k}a^{n-k}b^k\]
	\end{defsatz}
	\begin{satz}[Bernoullische Ungleichung]
		Für $n\in\mathbb{N}, 1\leq x\in\mathbb{R}$ gilt $(1+x)^n\geq1+nx$.
	\end{satz}
	\begin{satz}
		Für $x\in\mathbb{R},n\in\mathbb{N}$ gilt \[\sum_{k=0}^nx^k=\begin{cases}
			n+1 &\text{falls $x=1$}\\
			\frac{1-x^{n+1}}{1-x} &\text{sonst}
		\end{cases}\]
	\end{satz}
%	\end{multicols}
\end{document}

\documentclass[a4paper]{article}

\usepackage[l2tabu, orthodox]{nag}

\usepackage[utf8]{inputenc}
\usepackage[T1]{fontenc}

\usepackage[ngerman]{babel}

\usepackage{amsmath}
\usepackage{amssymb}
%\usepackage{amsthm}
\usepackage{mathtools}
\usepackage{physics}

\usepackage[framed]{ntheorem}

\usepackage{csquotes}
\usepackage{lmodern}
\usepackage{microtype}
\usepackage{enumitem}
\usepackage{stmaryrd}

\usepackage{parskip}
\usepackage{multicol}

\usepackage{array}
\usepackage{blindtext}
\usepackage{float}

\usepackage[left=1.8cm, right=1.8cm, top=1.8cm, bottom=2.5cm]{geometry}

\newcounter{Sec}

\theoremstyle{marginbreak}
\theorembodyfont{\normalfont}
\newtheorem{definition}{Definition}[Sec]
\newtheorem{satz}[definition]{Satz}
\newtheorem{defsatz}[definition]{Definition und Satz}
\newtheorem{verfahren}[definition]{Verfahren}
\newtheorem{defver}[definition]{Definition und Verfahren}
\newtheorem{defsatzver}[definition]{Definition, Satz und Verfahren}
\newtheorem{satzver}[definition]{Satz und Verfahren}

\MakeOuterQuote{"}
\DeclareMathOperator{\ffa}{ffa}

\newcommand{\sep}{%
	\rule{\textwidth}{0.3pt}%
	\stepcounter{Sec}%
	}
\newcommand{\defiff}{:\Longleftrightarrow}

\newcommand{\en}{~(n\to\infty)}
\newcommand{\series}[1][1]{\sum_{n=#1}^\infty}
\newcommand{\ps}[1][a]{\series[0]#1_n(x-x_0)^n}
\renewcommand{\d}[1]{\mathrm{d}#1}

\newcolumntype{M}[1]{>{\centering\arraybackslash}m{#1}}
\newcolumntype{N}{@{}m{0pt}@{}}

\setlength\columnsep{1.5cm}

\DeclareMathOperator{\arsinh}{arsinh}
\DeclareMathOperator{\arcosh}{arcosh}
\DeclareMathOperator{\artanh}{artanh}

\DeclarePairedDelimiterX\set[1]\lbrace\rbrace{\def\given{\;\delimsize\vert\;}#1}

\begin{document}
	\textsc{Analysis I}

	\sep
	\begin{defsatz}
		Zu $\varnothing\neq M\subseteq\mathbb{R}$ heißt $\gamma\in\mathbb{R}$
		\begin{description}
			\item[Obere Schranke] $\defiff \forall x\in M: \gamma\geq x$
			\item[Supremum] $\defiff \gamma~\text{ist OS} \wedge \forall~\text{OS $x$ von $M$}: \gamma\leq x$
			\item[Maximum] $\defiff \gamma = \sup M\in M$
		\end{description}
		Besitzt $M$ eine obere Schranke, besitzt es auch ein Supremum.
	\end{defsatz}
	\begin{satz}
		Für $\varnothing\neq M\subseteq\mathbb{R}$ und obere Schranke $\gamma$ von $M$ gilt
		$\gamma = \sup M\iff\forall\varepsilon>0~\exists x\in M:x>\gamma-\varepsilon$.
	\end{satz}
	\sep
	\begin{satz}[Betragssätze]
		$\forall a,b\in\mathbb{R}:$
		\begin{enumerate}[label=(\alph*)]
			\item $\abs{ab}=\abs{a}\abs{b}$
			\item $\pm a\leq\abs{a}$
			\item $\abs{a+b}\leq\abs{a}+\abs{b}$
		\end{enumerate}
	\end{satz}
	\begin{defsatz}
		\begin{enumerate}[label=(\alph*)]
			\item Für $a\in\mathbb{R}$ ist $[a]$ mit $[a]\in\mathbb{Z}\wedge [a]\leq a<[a]+1$ existent und eindeutig.
			\item Für $x,y\in\mathbb{R}$ mit $x<y$ existiert $r\in\mathbb{Q}$ mit $x<r<y$.
		\end{enumerate}
	\end{defsatz}
	\sep
	\begin{definition}[Folge]
		Eine Abbildung $\mathbb{N}\to B$ heißt Folge in $B$.
	\end{definition}
	\begin{defsatz}
		Eine nichtleere Menge $B$ heißt
		\begin{description}
			\item[endlich] $\defiff\exists n\in\mathbb{N},f\colon \set{1,\ldots,n}\to B~\text{surjektiv}$
			\item[abzählbar] $\defiff\exists f\colon\mathbb{N}\to B~\text{surjektiv}$
		\end{description}
		Die abzählbaren Mengen sind über dem kartesischen Produkt und der
		Vereinigung abzählbar vieler Mengen abgeschlossen.
	\end{defsatz}
	\sep
	\begin{defsatz}[Binomischer Lehrsatz]
		Für $n, k\in\mathbb{N}$ ist $\binom{n}{k}\coloneqq\frac{n!}{k!(n-k)!}$.
		Für $a, b\in\mathbb{R}, n\in\mathbb{N}$ gilt
		\[(a+b)^n = \sum_{k=0}^n\binom{n}{k}a^{n-k}b^k\]
	\end{defsatz}
	\begin{satz}[Bernoullische Ungleichung]
		Für $n\in\mathbb{N}, -1\leq x\in\mathbb{R}$ gilt $(1+x)^n\geq1+nx$.
	\end{satz}
	\sep
	\begin{definition}
		Eine Folge heißt beschränkt, wenn die Menge der Folgengleider beschränkt
		ist. $\sup_{n\in\mathbb{N}}a_n\coloneqq\sup\set{a_n\given n\in\mathbb{N}}$.
	\end{definition}
	\begin{defsatz}[Konvergenz]
		Eine Folge $(a_n)$ in $\mathbb{R}$ heißt konvergent, wenn ein $a\in\mathbb{R}$ existiert, sodass
		\[\lim_{n\to\infty}(a_n) = a \defiff a_n\to a\en \defiff
			\forall\varepsilon>0~\exists n_0\in\mathbb{N}~\forall n>n_0:\abs{a_n-a}<\varepsilon\]
		Falls der Grenzwert $a$ existiert, ist er eindeutig und $(a_n)$ ist beschränkt.
	\end{defsatz}
	\begin{satz}
		Es seien $(a_n), (b_n)$ Folgen in $\mathbb{R}$ und $a\in\mathbb{R}$.
		\begin{enumerate}[label=(\alph*)]
			\item Aus $(a_n) = (b_n) \ffa n\in\mathbb{N}$ folgt für $a_n\to a\en \iff b_n\to a\en$.
			\item $a_n\to a\en \iff \abs{a_n-a}\to0\en$.
		\end{enumerate}
	\end{satz}
	\begin{satz}[Konvergenzsätze]
		Es seien $(a_n), (b_n), (c_n)$ Folgen in $\mathbb{R}$ und mit $(a_n)\to a,b_n\to b\en$.
		\begin{enumerate}[label=(\alph*)]
			\item $a_n\leq b_n\ffa n\in\mathbb{N}\implies a\leq b$.
			\item $a=b\wedge a_n\leq c_n\leq b_n\ffa n\in\mathbb{N}\implies c_n\to a\en$.
			\item $\abs{a_n}\to\abs{a}\en$.
			\item $a_n+b_n\to a+b\en$.
			\item $\alpha a_n\to\alpha a\en$ für $\alpha\in\mathbb{R}$.
			\item $a_n b_n\to ab\en$.
			\item $b\neq0\implies \exists m\in\mathbb{N}: (\forall n\geq m: b_n\neq0) \wedge ((\frac{1}{b_n})_{n\geq m}\to\frac{1}{b}\en)$.
		\end{enumerate}
	\end{satz}
	\begin{definition}
		Eine Folge $(a_n)\in\mathbb{R}$ heißt
		\begin{description}
			\item[monoton wachsend] $\defiff\forall n\in\mathbb{N}:a_{n+1}\geq a_n$
			\item[streng monoton wachsend] $\defiff\forall n\in\mathbb{N}:a_{n+1}>a_n$
		\end{description}
	\end{definition}
	\begin{satz}
		Ist $(a_n)$ streng monoton wachsend und nach oben beschränkt, dann ist $\lim_{n\to\infty} a_n=\sup_{n\in\mathbb{N}}a_n$.
	\end{satz}
	\begin{satz}
		Für $(a_n)$ Folge in $\mathbb{R}$ mit $a_n>0~\forall n\in\mathbb{N}$ und $a_n\to a\en$,
		$p\in\mathbb{N}$ mit $p\geq2$. Dann $\sqrt[p]{a_n}\to\sqrt[p]{a}\en$.
	\end{satz}
	\begin{satz}
		Es sei $x\in\mathbb{R}$ und $\forall n\in\mathbb{N}:a_n\coloneqq x^n$.
		\begin{enumerate}[label=(\alph*)]
			\item $(a_n)~\text{konvergiert}\iff x\in(-1,1]$.
			\item $\lim_{n\to\infty}a_n=0\iff x\in(-1,1)$.
			\item $\lim_{n\to\infty}a_n=1\iff x =1$.
		\end{enumerate}
	\end{satz}
	\begin{satz}
		Für $x\in\mathbb{R},n\in\mathbb{N}$ gilt \[s_n\coloneqq\sum_{k=0}^nx^k=\begin{cases}
			n+1 &\text{falls $x=1$}\\
			\frac{1-x^{n+1}}{1-x} &\text{sonst}
		\end{cases}\]
		$(s_n)~\text{konvergiert}\iff\abs{x}<1$. In diesem Fall $\lim_{n\to\infty}s_n=\frac{1}{1-x}$.
	\end{satz}
	\begin{satz}
		$\sqrt[n]{n}\to1\en$. $\forall c>0:\sqrt[n]{c}\to1\en$.
	\end{satz}
	\begin{defsatz}[Die eulersche Zahl]
		\[ e\coloneqq \lim_{n\to\infty}(1+\frac{1}{n})^n = \lim_{n\to\infty}\sum_{k=0}^n\frac{1}{k!}\]
	\end{defsatz}
	\sep
	\begin{definition}[$\varepsilon$-Umgebung]
		$\forall\alpha\in\mathbb{R},\varepsilon>0:U_\varepsilon(\alpha)\coloneqq\set{a\in\mathbb{R}\given\abs{a-\alpha}<\varepsilon}
		=(\alpha-\varepsilon,\alpha+\varepsilon)$.
	\end{definition}
	\begin{definition}[Häufungswert]
		Es sei $(a_n)$ Folge in $\mathbb{R}$. $\alpha\in\mathbb{R}~\text{ist Häufungswert von $(a_n)$}\defiff\forall\varepsilon>0:a_n\in
		U_\varepsilon(\alpha)~\text{für unendlich viele}~n\in\mathbb{N}$. $\mathcal{H}(a_n)\coloneqq\set{\alpha\in\mathbb{R}\given
		\alpha~\text{ist Häufungswert von $(a_n)$}}$.
	\end{definition}
	\begin{definition}[Teilfolge]
		Für $(a_n)_n$ Folge in $\mathbb{R}$ und $(n_k)_k$ streng monoton wachsende
		Folge in $\mathbb{N}$ heißt $(a_{n_k})_k$ Teilfolge von $(a_n)$.
	\end{definition}
	\begin{satz}
		Für Folge $(a_n)$ in $\mathbb{R}$ und $\alpha\in\mathbb{R}$ gilt
		$\alpha\in\mathcal{H}(a_n)\iff\exists (n_k)_k~\text{streng monoton wachsend in}~\mathbb{N}:a_{n_k}\to\alpha~(k\to\infty)$.
	\end{satz}
	\begin{satz}
		Für $(a_n)$ konvergent gilt $\mathcal{H}(a_n)=\set{\lim_{n\to\infty} a_n}$. Für jede Teilfolge $(a_{n_k})$ gilt
		$\lim_{k\to\infty} a_{n_k} = \lim_{n\to\infty} a_n$.
	\end{satz}
	\begin{satz}[von Bolzano-Weierstraß]
		Für $(a_n)$ beschränkt gilt $\mathcal{H}(a_n)\neq\varnothing$.
	\end{satz}
	\sep
	\begin{defsatz}[Oberer und unterer Limes]
		Für $(a_n)$ beschränkte Folge in $\mathbb{R}$ existieren $\lim\sup a_n\coloneqq\max\mathcal{H}(a_n)$
		und $\lim\inf a_n\coloneqq\min\mathcal{H}(a_n)$.
	\end{defsatz}
	\begin{satz}
		Für beschränkte Folge $(a_n)$ in $\mathbb{R}$ sind äquivalent:
		\begin{enumerate}[label=(\alph*)]
			\item $\lim\inf a_n=\lim\sup a_n$
			\item $\abs{\mathcal{H}(a_n)} = 1$
			\item $(a_n)$ ist konvergent
		\end{enumerate}
	\end{satz}
	\sep
	\begin{defsatz}[Cauchy-Kriterium]
		Eine Folge $(a_n)$ in $\mathbb{R}$ heißt Cauchyfolge, wenn
		$\forall\varepsilon>0~\exists n_0\in\mathbb{N}~\forall n,m\geq n_0:\abs{a_n-a_m}<\varepsilon$.
		Dies ist genau dann der Fall, wenn $(a_n)$ konvergent ist.
	\end{defsatz}
	\sep
	\begin{definition}[Unendliche Reihe]
		Ist $(a_n)$ eine Folge in $\mathbb{R}$, so heißt die Folge $\series a_n\coloneqq(s_n)$
		mit $\forall n\in\mathbb{N}:s_n\coloneqq\sum_{k=1}^na_k$ eine (unendliche)
		Reihe. Falls $\series a_n$ konvergent, so ist der Reihenwert
		$\series a_n\coloneqq\lim_{n\to\infty}s_n$.
	\end{definition}
	\begin{satz}
		Ist $(a_n)$ eine Folge in $\mathbb{R}$ und $\series a_n$ konvergent,
		so ist $\lim a_n=0$.
	\end{satz}
	\begin{defsatz}[Absolute Konvergenz]
		Eine Reihe $\series a_n$ ist genau dann absolut konvergent, wenn
		$\series\abs{a_n}$ konvergent ist. Eine absolut konvergente
		Reihe ist konvergent und es gilt
		$\abs{\series a_n}\leq\series\abs{a_n}$.
	\end{defsatz}
	\begin{satz}[Leibnizkriterium]
		Für monoton fallende Nullfolge $(a_n)$ ist $\series (-1)^na_n$
		konvergent.
	\end{satz}
	\begin{satz}[Majorantenkriterium]
		$\abs{a_n}\leq b_n\ffa n\in\mathbb{N}\wedge\series b_n~\text{konvergiert}
		\implies\series a_n~\text{konvergiert absolut}$.
	\end{satz}
	\begin{satz}[Minorantenkriterium]
		$a_n\geq b_n\geq0\ffa n\in\mathbb{N}\wedge\series b_n~\text{divergiert}
		\implies\series a_n~\text{divergiert}$.
	\end{satz}
	\begin{satz}[Wurzelkriterium]
		Für Folge $(a_n)$ in $\mathbb{R}$ setze $\alpha\coloneqq\lim\sup\sqrt[n]{\abs{a_n}}$.
		\begin{enumerate}[label=(\alph*)]
			\item $\alpha<1\implies\series a_n~\text{konvergiert absolut}$
			\item $\alpha>1\implies\series a_n~\text{divergiert}$
		\end{enumerate}
	\end{satz}
	\begin{satz}[Quotientenkriterium]
		Für Folge $(a_n)$ in $\mathbb{R}$ mit $a_n\neq0\ffa n\in\mathbb{N}$ definiere
		$(\alpha_n)$ mit $\alpha_n\coloneqq\frac{a_{n+1}}{a_n}\ffa n\in\mathbb{N}$.
		\begin{enumerate}[label=(\alph*)]
			\item $\abs{\alpha_n}\geq1\ffa n\in\mathbb{N}\implies\series a_n~\text{ist divergent}$
			\item Falls $(\alpha_n)$ beschränkt, $\beta\coloneqq\lim\inf\abs{\alpha_n}$,
			$\alpha\coloneqq\lim\sup\abs{\alpha_n}$.
			\begin{enumerate}[label=(\roman*)]
				\item $\beta>1\implies\series a_n~\text{ist divergent}$
				\item $\alpha<1\implies\series a_n~\text{ist absolut konvergent}$
			\end{enumerate}
		\end{enumerate}
	\end{satz}
	\sep
	\begin{definition}
		Ist $(a_n)$ eine Folge und $\sigma\colon\mathbb{N}\to\mathbb{N}$ bijektiv,
		dann heißt $(b_n)$ mit $b_n\coloneqq a_{\sigma(n)}$ eine Umordnung von
		$(a_n)$. Die Reihe über die Umordnung wird als Umordnung der Reihe bezeichnet.
	\end{definition}
	\begin{satz}
		$(b_n)$ sei eine Umordnung von $(a_n)$.
		\begin{enumerate}[label=(\alph*)]
			\item $(a_n)~\text{konvergent}\implies\lim a_n=\lim b_n$
			\item $\series a_n~\text{absolut konvergent}\implies\series b_n~\text{absolut konvergent}
				\wedge\series a_n=\series b_n$
		\end{enumerate}
	\end{satz}
	\begin{satz}[Riemannscher Umordnungssatz]
		Ist $\series a_n$ konvergent,
		aber nicht absolut konvergent, so existiert für jedes $s\in\mathbb{R}\cup
		\set{\infty,-\infty}$ eine Umordnung $(b_n)$ von $(a_n)$ mit
		$\series b_n = s$.
	\end{satz}
	\sep
	\begin{defsatz}[Produktreihe]
		$\series[0] p_n$ heißt Produktreihe von $\series[0] a_n$ und
		$\series[0] b_n$, wenn $\set{p_n\given n\in\mathbb{N}_0} = \set{a_jb_k\given j,k\in\mathbb{N}_0}$,
		weobei jedes $a_jb_k$ genau einmal vorkommt. Alle Produktreihen sind
		Umordnungen voneinander.
	\end{defsatz}
	\begin{definition}[Cauchyprodukt]
		Die unendliche Reihe $\series[0] c_n\coloneqq\series[0] \sum_{k=0}^n a_kb_{n-k}$ heißt
		Cauchyprodukt von $\series[0] a_n$ und $\series[0] b_n$.
	\end{definition}
	\begin{satz}
		Sind $\series[0] a_n$ und $\series[0] b_n$ absolut konvergent, so sind alle
		Produktreihen $\series[0] p_n$ und das Cauchyprodukt $\series[0] c_n$
		absolut konvergent mit
		$\series[0] p_n = \series[0] c_n = (\series[0] a_n)(\series[0] b_n)$.
	\end{satz}
	\begin{defsatz}[Die Exponentialfunktion]
		Die Definition $\forall x\in\mathbb{R}:e^x\coloneqq E(x)\coloneqq\series[0]\frac{x^n}{n!}$
		deckt sich mit der Potenz für rationale Exponenten.
	\end{defsatz}
	\begin{definition}[Kosinus und Sinus]
		$\forall x\in\mathbb{R}$:
		\begin{align*}
			\cos x &\coloneqq\series[0](-1)^n\frac{x^{2n}}{(2n)!}\\
			\sin x &\coloneqq\series[0](-1)^n\frac{x^{2n+1}}{(2n+1)!}
		\end{align*}
	\end{definition}
	\sep
	\begin{definition}[Potenzreihe]
		Für Folge $(a_n)_{n=0}^\infty$ in $\mathbb{R}$ und $x_0\in\mathbb{R}$
		heißt $\ps$ Potenzreihe.
	\end{definition}
	\begin{satz}[Konvergenz von Potenzreihen]
		Für Potenzreihe $\ps$ setze
		$\rho\coloneqq\lim\sup\sqrt[n]{\abs{a_n}}$ und $r\coloneqq\frac{1}{\rho}$,
		wobei $r\coloneqq0$, falls $\rho=\infty$ und $r\coloneqq\infty$, falls
		$\rho=0$.
		\begin{enumerate}[label=(\alph*)]
			\item Falls $r=0$ konvergiert die Potenzreihe genau für $x=x_0$.
			\item Falls $r=\infty$ konvergiert die Potenzreihe absolut für alle $x\in\mathbb{R}$.
			\item Falls $r\in\mathbb{R}$ konvergiert die Potenzreihe für $x \in (x_0 - r, x_0 + r)$
				absolut und divergiert für $x \in (-\infty, x_0-r) \cup (x_0+r, \infty)$. Für
				$x\in\set{x_0+r,x_0-r}$ ist keine allgemeine Aussage möglich.
		\end{enumerate}
		$r$ wird auch als der Konvergenzradius der Potenzreihe bezeichnet.
	\end{satz}
	\begin{satz}[Konvergenzradien von Cauchyprodukten]
		Für Potenzeihen $\ps$ und $\ps[b]$ mit Konvergenzradien $r_1$ und $r_2$
		setze $R\coloneqq\min\set{r_1,r_2}$, $c_n\coloneqq\sum_{k=0}^na_kb_{n-k}$.
		$r$ sei der Konvergenzradius von $\ps[c]$. Dann ist $R\leq r$ und für
		$x\in(x_0-R,x_0+R)$ gilt $\ps[c]=(\ps)(\ps[b])$.
	\end{satz}
	\sep
	\begin{definition}[Häufungspunkt]
		$x_0~\text{ist Häufungspunkt von}~D\subseteq\mathbb{R}
		\defiff\exists(x_n)~\text{in $D\setminus\set{x_0}$}:x_n\to x_0\en$
	\end{definition}
	\begin{definition}
		Für $\varnothing\neq D\subseteq\mathbb{R}$, $\delta>0$ sind
		$D_\delta(x_0)\coloneqq D\cap U_\delta(x_0)$, $\dot{D}_\delta(x_0)\coloneqq
		D_\delta(x_0)\setminus\set{x_0}$.
	\end{definition}
	\begin{definition}
		Ist $(a_n)$ Folge in $\mathbb{R}$, so ist
		$\lim a_n = \infty\defiff \forall c>0~\exists n_0\in\mathbb{N}~\forall n\geq n_0:a_n>c$.
	\end{definition}
	\begin{defsatz}[Funktionsgrenzwert]
		Für $\varnothing\neq D\subseteq\mathbb{R}$, $x_0$ Häufungspunkt von $D$,
		$f\colon D\to\mathbb{R}$ ist
		\begin{align*}
			\lim_{x\to x_0}f(x) = a
			\defiff&~\forall(x_n)~\text{in}~D\setminus\set{x_0}~\text{mit}~x_n\to x_0\en:f(x_n)\to a\en\\
			\Longleftrightarrow&~\forall\varepsilon>0~\exists\delta>0~\forall x\in\dot{D}_\delta(x_0):\abs{f(x)-a}<\varepsilon
		\end{align*}
		Existiert $\lim_{x\to x_0}f(x)$, dann ist er eindeutig.
	\end{defsatz}
	\sep
	\begin{defsatz}[Stetigkeit]
		Für $\varnothing\neq D\subseteq\mathbb{R}$, $x_0\in D$, $f\colon D\to\mathbb{R}$
		ist
		\begin{align*}
			f~\text{stetig in}~x_0
			\defiff&~\forall(x_n)~\text{in}~D~\text{mit}~x_n\to x_0\en:f(x_n)\to f(x_0)\en\\
			\Longleftrightarrow&~\forall\varepsilon>0~\exists\delta>0~\forall x\in D_\delta(x_0):\abs{f(x)-f(x_0)}<\varepsilon
		\end{align*}
	\end{defsatz}
	\begin{defsatz}
		Für $\varnothing\neq D\subseteq\mathbb{R}$, $f\colon D\to\mathbb{R}$
		ist
		\begin{align*}
			f\in C(D)\defiff
			f~\text{stetig auf $D$}
			\defiff&~\forall x_0\in D:f~\text{stetig in $x_0$}\\
			\Longleftrightarrow&~\forall x_0\in D~\forall\varepsilon>0~\exists\delta>0~\forall x\in D_\delta(x_0):\abs{f(x)-f(x_0)}<\varepsilon\\
			\Longleftrightarrow&~\forall\varepsilon>0~\forall x\in D~\exists\delta>0~\forall z\in D~\text{mit}~\abs{x-z}<\delta:\abs{f(x)-f(z)}<\varepsilon\\
			f~\text{gleichmäßig stetig auf $D$}
			\defiff&~\forall\varepsilon>0~\exists\delta>0~\forall x\in D~\forall z\in D~\text{mit}~\abs{x-z}<\delta:\abs{f(x)-f(z)}<\varepsilon\\
			f~\text{Lipschitz stetig auf $D$}
			\defiff&~\exists L>0~\forall x,y\in D:\abs{f(x)-f(z)}\leq L\abs{x-z}
		\end{align*}
		Aus der Lipschitz Stetigkeit folgt die gleichmäßige Stetigkeit.
		Aus der gleichmäßigen Stetigkeit folgt die Stetigkeit.
	\end{defsatz}
	\begin{satz}
		Ist $x_0$ ein Häufungspunkt von $D$, dann gilt $f~\text{stetig in}~x_0\iff\lim_{x\to x_0}f(x)=f(x_0)$.
	\end{satz}
	\begin{satz}
		Sind $f\colon D\to\mathbb{R}$, $g\colon D\to\mathbb{R}$ stetig in $x_0$, so
		sind $f + g$, $fg$, $\abs{f}$ stetig in $x_0$. Ist $\tilde{D}\coloneqq\set{x\in D\given f(x)\neq0}$
		und $x_0\in\tilde{D}$, so ist $\frac{1}{f}\colon\tilde{D}\to\mathbb{R}$ stetig in
		$x_0$.
	\end{satz}
	\begin{satz}
		Es sei $\varnothing\neq D, E\subseteq\mathbb{R}$, $f\colon D\to E$ stetig in $x_0$,
		$g\colon E\to\mathbb{R}$ stetig in $f(x_0)$. Dann ist $g\circ f\colon D\to\mathbb{R}$
		stetig in $x_0$.
	\end{satz}
	\begin{satz}
		Hat $\ps$ den Konvergenzradius $r>0$, setze $D\coloneqq(x_0-r,x_0+r)$,
		$f\colon D\to\mathbb{R}$ mit $\forall x\in D:f(x)\coloneqq\ps$. Dann
		ist $f\in C(D)$ und für $\tilde{x}\in D$ gilt
		\[ \lim_{x\to\tilde{x}}\ps = \lim_{x\to\tilde{x}}f(x)=f(\tilde{x})=
		\sum_{n=0}^\infty a_n(\tilde{x}-x_0)^n=\sum_{n=0}^\infty\lim_{x\to\tilde{x}}a_n(x-x_0)^n \]
	\end{satz}
	\begin{satz}[von Heine]
		Ist $D$ beschränkt und abgeschlossen und $f\in C(D)$, so ist $f$ auf
		$D$ gleichmäßig stetig.
	\end{satz}
	\sep
	\begin{satz}[Der Zwischenwertsatz]
		Ist $a<b$, $f\in C[a,b]$, $y_0\in\mathbb{R}$, $\min\set{f(a),f(b)}\leq
		y_0\leq\max\set{f(a),f(b)}$, dann existiert $x_0\in[a,b]$ mit $f(x_0)=y_0$.
	\end{satz}
	\begin{satz}[Nullstellensatz von Bolzano]
		Ist $f\in C[a,b]$ und $f(a)f(b)<0$, dann existiert $x_0\in[a,b]$ mit $f(x_0)=0$.
	\end{satz}
	\begin{definition}
		$A\subseteq\mathbb{R}$ heißt
		\begin{description}
			\item[abgeschlossen] $\defiff\forall(x_n)~\text{in}~A~\text{konvergent}:\lim x_n\in A$
			\item[offen] $\defiff\forall x\in A~\exists\delta>0:U_\delta(x)\subseteq A$
		\end{description}
	\end{definition}
	\begin{satz}
		Ist $D$ beschränkt und abgeschlossen und $f\in C(D)$, so nimmt $f$ auf $D$
		ein Minimum und ein Maximum an.
	\end{satz}
	\begin{satz}
		Ist $I\subseteq\mathbb{R}$ ein Intervall und $f\in C(I)$, dann ist
		$f(I)$ ein Intervall. Ist $f$ zusätzlich streng monoton, dann ist $f$
		umkehrbar und $f^{-1}\in C(f(I))$.
	\end{satz}
	\begin{defsatz}[Der Logarithmus]
		Ist $f\colon\mathbb{R}\to\mathbb{R}$ mit $f(x)\coloneqq e^x$, so existiert
		$\log \coloneqq f^{-1}\colon (0,\infty)\to\mathbb{R}$.
	\end{defsatz}
	\begin{definition}[Die allgemeine Potenz]
		Für alle $a>0, x\in\mathbb{R}$ ist $a^x\coloneqq e^{x\log a}$.
	\end{definition}
	\sep
	\begin{defsatz}[Punktweise und gleichmäßige Konvergenz]
		Es sei $\varnothing\neq D\subseteq\mathbb{R}$ und $(f_n)$ eine Folge mit
		$\forall n\in\mathbb{N}:f_n\colon D\to\mathbb{R}$. Ferner sei $f\colon D\to\mathbb{R}$.
		\begin{align*}
			f_n\to f~\text{pktw}
			\defiff&~\forall x\in D:\lim f_n(x)=f(x)\\
			\Longleftrightarrow&~\forall\varepsilon>0~\forall x\in D~\exists n_0\in\mathbb{N}~\forall n\geq n_0:\abs{f_n(x)-f(x)}<\varepsilon\\
			f_n\to f~\text{glm}
			\defiff&~\forall\varepsilon>0~\exists n_0\in\mathbb{N}~\forall n\geq n_0~\forall x\in D:\abs{f_n(x)-f(x)}<\varepsilon
		\end{align*}
		Aus der gleichmäßigen Konvergenz folgt stets die punktweise Konvergenz.
	\end{defsatz}
	\begin{satz}
		Konvergiert $f_n$ punktweise gegen $f\colon D\to\mathbb{R}$, dann gilt
		\[ f_n\to f~\text{glm}\iff\exists(\alpha_n)~\text{in $\mathbb{R}$ mit $\alpha_n\to0\en$}~\exists
		m\in\mathbb{N}~\forall n\geq m~\forall x\in\mathbb{R}:\abs{f_n(x)-f(x)}\leq\alpha_n\]
	\end{satz}
	\begin{satz}
		$f_n\to f~\text{glm}\iff\forall(x_n)~\text{in}~D:f_n(x_n)-f(x_n)\to0\en$.
	\end{satz}
	\begin{satz}
		Es sei $\varnothing\neq D\subseteq\mathbb{R}$ und $(f_n)$ eine Folge mit
		$\forall n\in\mathbb{N}:f_n\colon D\to\mathbb{R}$. Ferner sei $f\colon D\to\mathbb{R}$
		mit $f_n\to f~\text{glm}$.
		\begin{enumerate}[label=(\alph*)]
			\item Ist $x_0\in D$ und alle $f_n$ stetig in $x_0$, so ist $f$ stetig in $x_0$.
			\item $(\forall n\in\mathbb{N}: f_n\in C(D)) \implies f\in C(D)$.
			\item Ist $x_0$ Häufungspunkt von $D$, dann gilt
			\[ \lim_{x\to x_0}\lim_{n\to\infty}f_n(x)=\lim_{x\to x_0}f(x)=f(x_0)=\lim_{n\to\infty}f_n(x_0)=\lim_{n\to\infty}\lim_{x\to x_0}f_n(x) \]
		\end{enumerate}
	\end{satz}
	\begin{satz}
		Es sei $a,b\in\mathbb{R}$, $a<b$, $(f_n)$ eine Folge in $R[a,b]$,
		$f\colon[a,b]\to\mathbb{R}$, $f_n\to f~\text{glm}$. Dann $f\in R[a,b]$ mit
		\[ \lim_{n\to\infty}\int_a^bf_n(x)\d{x}=\int_a^bf(x)\d{x}=\int_a^b(\lim_{n\to\infty}f_n(x))\d{x} \]
	\end{satz}
	\begin{satz}
		Es sei $a,b\in\mathbb{R}$, $a<b$, $(f_n)$ eine Folge in $C^1[a,b]$,
		$x_0\in[a,b]$, $(f_n(x_0))$ konvergiere, es sei $g\colon[a,b]\to\mathbb{R}$
		und es sei $f'_n\to g~\text{glm}$
		auf $[a,b]$. Dann $f_n\to f~\text{glm}$ mit $f\colon[a,b]\to\mathbb{R}$,
		$\forall x\in[a,b]: f(x)\coloneqq\lim_{n\to\infty}f_n(x)$ und es ist $\forall x\in[a,b]$:
		\[ (\lim_{n\to\infty}f_n(x))'=f'(x)=\lim_{n\to\infty}f_n'(x) \]
	\end{satz}
	\sep
	\begin{defsatz}[Differenzierbarkeit]
		Es sei $I\subseteq\mathbb{R}$ ein Intervall und $f\colon I\to\mathbb{R}$.
		$f$ heißt in $x_0\in I$ differenzierbar, wenn
		$f'(x_0)\coloneqq\lim_{x\to x_0}\frac{f(x)-f(x_0)}{x-x_0}$ existiert.
		$f$ heißt auf $I$ differenzierbar, wenn $\forall x_0\in I:f~\text{db in}~x_0$.
		In diesem Fall heißt die so definierte Funktion $f'\colon I\to\mathbb{R}$ die
		Ableitung von $f$. Aus der Differenzierbarkeit folgt die Stetigkeit.
	\end{defsatz}
	\begin{satz}
		Es sei $I\subseteq\mathbb{R}$ ein Intervall und $f,g\colon I\to\mathbb{R}$
		differenzierbar in $x_0\in I$.
		\begin{enumerate}[label=(\alph*)]
			\item Für $\alpha,\beta\in\mathbb{R}$ ist $(\alpha f+\beta g)'(x_0)=\alpha f'(x_0)+\beta g'(x_0)$.
			\item $fg$ ist differenzierbar in $x_0$ mit $(fg)'(x_0)=f'(x_0)g(x_0)+f(x_0)g'(x_0)$.
			\item Falls $\forall x\in I:g(x)\neq0$, ist $\frac{f}{g}$ differenzierbar in $x_0$ mit
				\[ (\frac{f}{g})'(x_0) = \frac{f'(x_0)g(x_0)-f(x_0)g'(x_0)}{g(x_0)^2} \]
		\end{enumerate}
	\end{satz}
	\begin{satz}
		Es seien $I, J\subseteq\mathbb{R}$ Intervalle, $f\colon I\to\mathbb{R}$
		$g\colon J\to I$, $g$ differenzierbar in $x_0\in J$, $f$ differenzierbar
		in $g(x_0)$. Dann ist $f\circ g\colon J\to\mathbb{R}$ differenzierbar in $x_0$
		mit $(f \circ g)'(x_0)=f'(g(x_0))g'(x_0)$.
	\end{satz}
	\begin{satz}
		Es sei $I\subseteq\mathbb{R}$ ein Intervall, $f\in C(I)$ streng monoton und
		differenzierbar in $x_0\in I$ und $f'(x_0)\neq0$. Dann ist $f^{-1}\colon f(I)\to I$
		differenzierbar in $f(x_0)$ und $(f^{-1})'(f(x_0))=\frac{1}{f'(x_0)}$.
	\end{satz}
	\begin{definition}
		Mit $\varnothing\neq D\subseteq\mathbb{R}$, $g\colon D\to\mathbb{R}$
		ist $x_0\in D~\text{relatives Maximum von $g$}\defiff\exists\delta>0~\forall
		x\in D\cup U_\delta(x_0):g(x)\leq g(x_0)$.
	\end{definition}
	\begin{definition}
		Mit $\varnothing\neq M\subseteq\mathbb{R}$ ist
		$x_0\in M~\text{innerer Punkt von $M$}\defiff\exists\delta>0:U_\delta(x_0)\subseteq M$.
	\end{definition}
	\begin{satz}
		Ist $I\subseteq\mathbb{R}$ ein Intervall, $f\colon I\to\mathbb{R}$ differenzierbar
		in $x_0\in I$, $x_0$ ein relatives Extremum von $f$ und $x_0$ ein innerer
		Punkt von $I$, so ist $f'(x_0)=0$.
	\end{satz}
	\begin{satz}
		Es sei $I=[a,b]$, $f, g\in C(I)$, $f$, $g$, differenzierbar auf $(a,b)$,
		$\forall x\in(a,b): g'(x)\neq0$.
		\begin{enumerate}[label=(\alph*)]
			\item $f(a)=f(b)\implies\exists\xi\in(a,b):f'(\xi)=0$.
			\item $\exists\xi\in(a,b):f(b)-f(a)=f'(\xi)(b-a)$.
			\item $g(a)\neq g(b)\wedge\exists\xi\in(a,b):\frac{f(b)-f(a)}{g(b)-g(a)}=\frac{f'(\xi)}{g'(\xi)}$.
		\end{enumerate}
	\end{satz}
	\begin{satz}[Die Regel von l'Hospital]
		Für $a\in\mathbb{R}\cup\set{-\infty}$, $b\in\mathbb{R}\cup\set{\infty}$
		seien $f,g\colon (a,b)\to\mathbb{R}$ auf $(a,b)$ differenzierbar mit
		$\forall x\in(a,b):g'(x)\neq0$. Es sei $c\in\set{a, b}$, es existiere
		$L\coloneqq\lim_{x\to c}\frac{f'(x)}{g'(x)}\in\mathbb{R}\cup\set{-\infty, \infty}$
		und es sei $\lim_{x\to c}f(x)=\lim_{x\to c}g(x)\in\set{0,-\infty,\infty}$.
		Dann $\lim_{x\to c}\frac{f(x)}{g(x)}=L$.
	\end{satz}
	\begin{satz}
		Sei $\ps$ eine Potenzreihe mit Konvergenzradius $r>0$, $I\coloneqq(x_0-r,x_0+r)$.
		Auf $I$ gilt dann $(\ps)'=\sum_{n=0}^\infty(a_n(x-x_0)^n)'$. Diese Potenzreihe
		hat den Konvergenzradius $r$.
	\end{satz}
	\sep
	\begin{definition}
		Für Intervall $I\subseteq\mathbb{R}$, $f\colon I\to\mathbb{R}$, ist für $x_0\in I$,
		falls existent, $f^{(n)}(x_0) \coloneqq ((\cdots(f\underbrace{')'\cdots)')'}_{n~\text{mal}}(x_0)$
		Falls $\forall x\in I~\exists f^{(n)}(x)$, ist $f$ $n$-mal differenzierbar auf $I$.
		Ist zusätzlich $f^{(n)}\in C(I)$, ist $f$ $n$-mal stetig differenzierbar, also
		$f\in C^n(I)$. $C^0(I)\coloneqq C(I)$, $C^\infty(I)\coloneqq\bigcap_{n\in\mathbb{N}}C^n(I)$.
	\end{definition}
	\begin{satz}
		Ist $\ps$ eine Potenzreihe mit Konvergenzradius $r>0$, $I\coloneqq(x_0-r,x_0+r)$ und
		$f(x)\coloneqq\ps$, dann ist $f\in C^\infty(I)$, $f^{(k)}(x)=\sum_{n=k}^{\infty}\frac{n!}{(n-k)!}a_n(x-x_0)^{n-k}$.
		Es folgt $\forall k\in\mathbb{N}_0:a_k=\frac{f^{(k)}(x_0)}{k!}$.
	\end{satz}
	\begin{defsatz}
		Zu Intervall $I\subseteq\mathbb{R}$, $f\in C^\infty(I)$, $x_0\in I$ gehörige
		Taylerreihe ist $\sum_{k=0}^\infty\frac{f^{(k)}(x_0)}{k!}(x - x_0)^k$.
		Hin und wieder lässt sich $f$ in bestimmten Umgebungen um $x_0$ durch seine
		Taylorreihe approximieren.
	\end{defsatz}
	\begin{defsatz}[von Taylor]
		Für Intervall $I\subseteq\mathbb{R}$, $n\in\mathbb{N}_0$, $f\in  C^n(I)$,
		$x_0\in I$ ist $T_n(x;x_0)\coloneqq\sum_{k=0}^n\frac{f^{(k)}(x_0)}{k!}(x-x_0)^k$.
		Ist $f$ zusätzlich $n+1$-mal differenzierbar auf $I$ und $x\in I$, dann existiert
		$\xi$ zwischen $x$ und $x_0$ mit
		\[ f(x) = T_n(x;x_0)+\frac{f^{(n+1)}(\xi)}{(n+1)!}(x-x_0)^{n+1} \]
	\end{defsatz}
	\begin{satz}
		Ist $n\in\mathbb{N}$, $n>2$, $I\subseteq\mathbb{R}$ Intervall,
		$f\in C^n(I)$, $x_0\in I$ innerer Punkt von $I$, $\forall 1\leq i\leq n-1:f^{(i)}(x_0)=0$,
		$f^{(n)}(x_0)\neq0$, dann
		\begin{enumerate}[label=(\alph*)]
			\item $n~\text{gerade}\wedge f^{(n)}(x_0)<0\implies f~\text{hat ein relatives Maximum in}~x_0$
			\item $n~\text{gerade}\wedge f^{(n)}(x_0)>0\implies f~\text{hat ein relatives Minimum in}~x_0$
			\item $n~\text{ungerade}\implies f~\text{hat kein relatives Extremum in}~x_0$
		\end{enumerate}
	\end{satz}
	\sep
	\begin{definition}[Zerlegung]
		Ist $a,b\in\mathbb{R}$, $I\coloneqq[a,b]$, so heißt $Z=\set{x_0,\ldots,x_n}$ eine
		Zerlegung von $I$, wenn $a=x_0<x_1<\ldots<x_n=b$. Die Menge aller Zerlegungen
		wird mit $\mathfrak{Z}$ bezeichnet. Ist $f\colon I\to\mathbb{R}$ beschränkt, so gilt:
		$s_f(Z)\coloneqq\sum_{j=1}^n(\inf f(I_j))\abs{I_j}$ ist die Obersumme
		von $f$ bezüglich $Z$, $S_f(Z)\coloneqq\sum_{j=1}^n(\sup f(I_j))\abs{I_j}$
		ist die Obersumme von $f$ bezüglich $Z$.
	\end{definition}
	\begin{definition}[Das Riemann-Integral]
		\begin{align*}
			\underline{\int_{a}^{b}}f\d{x}\coloneqq\underline{\int_{a}^{b}}f(x)\d{x}\coloneqq&\sup\set{s_f(Z)\given Z\in\mathfrak{Z}}\\
			\overline{\int_{a}^{b}}f\d{x}\coloneqq\overline{\int_{a}^{b}}f(x)\d{x}\coloneqq&\inf\set{S_f(Z)\given Z\in\mathfrak{Z}}
		\end{align*}
		Falls $\underline{\int_{a}^{b}}f\d{x}=\overline{\int_{a}^{b}}f\d{x}$, ist $f\in R[a,b]$ und:
		\begin{align*}
			\int_{a}^{b}f\d{x}\coloneqq\int_{a}^{b}f(x)\d{x}\coloneqq\underline{\int_{a}^{b}}f\d{x}
		\end{align*}
	\end{definition}
	\begin{satz}
		Für $f,g\in R[a,b]$:
		\begin{enumerate}[label=(\alph*)]
			\item $f\leq g~\text{auf}~[a,b]\implies\int_a^bf\d{x}\leq\int_a^bg\d{x}$
			\item $\alpha,\beta\in\mathbb{R}\implies\alpha f+\beta g\in R[a,b]\wedge
				\int_a^b(\alpha f+\beta g)\d{x}=\alpha\int_a^bf\d{x}+\beta\int_a^bg\d{x}$
			\item $\abs{f}\in R[a,b]$, $\abs{\int_a^bf\d{x}}\leq\int_a^b\abs{f}\d{x}$.
			\item $fg\in R[a,b]$.
			\item $(\exists c\in\mathbb{R}~\forall x\in[a,b]:g(x)\neq0\wedge\abs{g(x)}<c)\implies\frac{1}{g}\in R[a,b]$.
		\end{enumerate}
	\end{satz}
	\begin{satz}
		\begin{enumerate}[label=(\alph*)]
			\item Jede monotone Funktion ist riemann-integrierbar.
			\item $C[a,b]\subseteq R[a,b]$.
		\end{enumerate}
	\end{satz}
	\begin{satz}
		 Ist $f\colon [a,b]\to\mathbb{R}$ beschränkt und $c\in(a,b)$, dann
		 $f\in R[a,b]\iff f\in R[a,c]\wedge f\in R[c,b]$. In diesem Fall
		 $\int_a^bf\d{x}=\int_a^cf\d{x}+\int_c^bf\d{x}$.
	\end{satz}
	\begin{satz}
		Es sei $f\in R[a,b]$, $D\coloneqq f([a,b])$, $h\colon D\to\mathbb{R}$ sei
		Lipschitzstetig auf $D$. Dann $h\circ f\in R[a,b]$.
	\end{satz}
	\begin{satz}
		\begin{enumerate}[label=(\alph*)]
			\item Ist $f\colon[a,b]\to\mathbb{R}$ beschränkt auf $[a,b]$ und
				$\set{x\in[a,b]\given f~\text{ist nicht stetig in}~x}$ endlich,
				dann ist $f\in R[a,b]$.
			\item Ist $f\in R[a,b]$ und $\set{x\in[a,b]\given f(x)\neq g(x)}$ endlich,
			dann ist $g\in R[a,b]$ und $\int_a^bg\d{x}=\int_a^bf\d{x}$.
		\end{enumerate}
	\end{satz}
	\begin{satz}[Erster Hauptsatz der Differential- und Integralrechnung]
		Es sei $f\in R[a,b]$ und $\forall x\in [a,b]:F'(x)=f(x)$. Dann
		\[ \int_a^bf(x)\d{x}=F(b)-F(a)\eqqcolon [F(x)]_a^b \]
	\end{satz}
	\begin{satz}[Zweiter Hauptsatz der Differential- und Integralrechnung]
		Mit $f\in R[a,b]$ und $F\colon[a,b]\to\mathbb{R}$, $F(x)\coloneqq\int_a^xf(t)\d{t}$:
		\begin{enumerate}[label=(\alph*)]
			\item $F$ ist auf $[a,b]$ Lipschitzstetig.
			\item $f~\text{in}~x_0\in[a,b]~\text{stetig}\implies F~\text{differenzierbar in $x_0$ mit}~F'(x_0)=f(x_0)$
			\item $f\in C[a,b]\implies F\in C^1[a,b]\wedge\forall x\in[a,b]:F'(x)=f(x)$
		\end{enumerate}
	\end{satz}
	\begin{satz}
		Ist $J\subseteq\mathbb{R}$ ein Intervall, $f\in C(J)$ und $\xi\in J$,
		$F\colon J\to\mathbb{R}$ mit $F(x)\coloneqq\int_\xi^xf(t)\d{t}$, dann ist
		$F\in C^1(J)$ und $F'=f$ auf $J$.
	\end{satz}
	\begin{satz}[Mittelwertsatz der Integralrechnung]
		Mit $f,g\in R[a,b]$, $g\geq0$ auf $[a,b]$, $m\coloneqq\inf f([a,b])$,
		$M\coloneqq\sup f([a,b])$:
		\begin{enumerate}[label=(\alph*)]
			\item $\exists\mu\in[m,M]:\int_a^bfg\d{x}=\mu\int_a^bg\d{x}$
			\item $f\in C[a,b]\implies\exists\xi\in[a,b]:\int_a^bf\d{x}=f(\xi)(b-a)$
		\end{enumerate}
	\end{satz}
	\begin{definition}
		Ist $I\subseteq\mathbb{R}$ ein Intervall und $G,g\colon I\to\mathbb{R}$, wobei
		$G$ differenzierbar auf $J$ und $G'=g$ auf $J$, dann ist $G$ auf $J$ eine
		Stammfunktion von $g$. Man schreibt auch $\int g\d{x}=G$.
	\end{definition}
	\begin{satz}[Partielle Integration]
		\begin{enumerate}[label=(\alph*)]
			\item Sind $f,g\in R[a,b]$ und $F$, $G$ Stammfunktionen von $f$, $g$
				auf $[a,b]$, dann
				\[ \int_a^bFg\d{x}=[F(x)G(x)]_a^b-\int_a^bfG\d{x} \]
			\item Sind $f,g\in C^1[a,b]$, dann
				\[ \int_a^bf'g\d{x}=[f(x)g(x)]_a^b-\int_a^bfg'\d{x} \]
			\item Sind $f,g\in C^1(I)$, dann gilt auf $I$
				\[ \int f'g\d{x}=f(x)g(x)-\int fg'\d{x} \]
		\end{enumerate}
	\end{satz}
	\begin{satz}[Integration durch Substitution]
		Es sei $f\in C(I)$, $g\in C^1(J)$, $g(J)\subseteq I$.
		\begin{enumerate}[label=(\alph*)]
			\item $J = [\alpha,\beta]\implies$
				\[ \int_\alpha^\beta f(g(t))g'(t)\d{t}=\int_{g(\alpha)}^{g(\beta)}f(t)\d{t} \]
			\item Auf $J$ gilt
				\[ \int f(g(t))g'(t)\d{t}=\int f(x)\d{x}|_{x=g(t)} \]
			\item $g~\text{strong monoton auf $J$}\implies\text{Auf $I$}$
				\[ \int f(x)\d{x}=\int f(g(t))g'(t)\d{t}|_{t=g^{-1}(x)} \]
		\end{enumerate}
	\end{satz}
	\newpage
	\sep
	\begin{multicols}{2}
		\begin{table}[H]
			\centering
			\begin{tabular}{ | M{2.5cm} | M{2.5cm} | N}
				\hline
				Funktion & Stammfunktion & \\[0.6cm] \hline \hline
				$nx^{n-1}$ & $x^n$ & \\[0.6cm] \hline
				$\frac{1}{x}$ & $\log\abs{x}$ & \\[0.6cm] \hline
				$\log x$ & $x\log x - x$ & \\[0.6cm] \hline
				$e^x$ & $e^x$ & \\[0.6cm] \hline
				$a^x\log a$ & $a^x$ & \\[0.6cm] \hline \hline
				$\sin x$ & $-\cos x$ & \\[0.6cm] \hline
				$\cos x$ & $\sin x$ & \\[0.6cm] \hline
				$\tan x$ & $-\log\abs{\cos x}$ & \\[0.6cm] \hline \hline
				$\frac{1}{\sqrt{1-x^2}}$ & $\arcsin x$ & \\[0.6cm] \hline
				$-\frac{1}{\sqrt{1-x^2}}$ & $\arccos x$ & \\[0.6cm] \hline
				$\frac{1}{1+x^2}$ & $\arctan x$ & \\[0.6cm] \hline \hline
				$\sinh x$ & $\cosh x$ & \\[0.6cm] \hline
				$\cosh x$ & $\sinh x$ & \\[0.6cm] \hline
				$\tanh x$ & $\log\cosh x$ & \\[0.6cm] \hline \hline
				$\frac{1}{\sqrt{x^2+1}}$ & $\arsinh x$ & \\[0.6cm] \hline
				$\frac{1}{\sqrt{x^2-1}}$ & $\arcosh x$ & \\[0.6cm] \hline
				$\frac{1}{1-x^2}$, $\abs{x}<1$ & $\artanh x$ & \\[0.6cm] \hline \hline
				$\frac{f'(x)}{1+f(x)^2}$ & $\arctan f(x)$ & \\[0.6cm] \hline
				$\frac{f'(x)}{f(x)}$ & $\log\abs{f(x)}$ & \\[0.6cm] \hline
				$f'(x)f(x)$ & $\frac{1}{2}f(x)^2$ & \\[0.6cm] \hline
				$f'(x)e^{f(x)}$ & $e^{f(x)}$ & \\[0.6cm] \hline
			\end{tabular}
			\caption{Liste von Stammfunktionen}
		\end{table}
		\begin{table}[H]
			\centering
			\begin{tabular}{ | M{2.5cm} | M{2.5cm} | N}
				\hline
				Reihe & Wert & \\[0.6cm] \hline \hline
				$\sum\limits_{n=0}^\infty x^n$, $\abs{x}<1$ & $\frac{1}{1-x}$ & \\[0.6cm] \hline
				$\sum\limits_{n=0}^\infty x^n$, $\abs{x}\geq1$ & $\infty$ & \\[0.6cm] \hline \hline
				$\sum\limits_{n=1}^\infty\frac{1}{n^\alpha}$, $\alpha > 1$ & konvergent & \\[0.6cm] \hline
				$\sum\limits_{n=1}^\infty\frac{1}{n^\alpha}$, $\alpha \leq 1$ & $\infty$ & \\[0.6cm] \hline
			\end{tabular}
			\caption{Liste von Reihenwerten}
		\end{table}
	\end{multicols}
\end{document}

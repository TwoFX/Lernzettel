\documentclass[a4paper]{article}

\usepackage[l2tabu, orthodox]{nag}

\usepackage[utf8]{inputenc}
\usepackage[T1]{fontenc}

\usepackage[ngerman]{babel}

\usepackage{amsmath}
\usepackage{amssymb}
%\usepackage{amsthm}
\usepackage{mathtools}
\usepackage{physics}
\usepackage{dsfont}

\usepackage[framed]{ntheorem}

\usepackage{csquotes}
\usepackage{lmodern}
\usepackage{microtype}
\usepackage{enumitem}
\usepackage{stmaryrd}

\usepackage{faktor}

\usepackage{parskip}
\usepackage{multicol}

\usepackage{array}
\usepackage{blindtext}
\usepackage{float}

\usepackage[hidelinks]{hyperref}

\usepackage[left=1.8cm, right=1.8cm, top=1.8cm, bottom=2.5cm]{geometry}

\newcounter{Sec}

\theoremstyle{marginbreak}
\theorembodyfont{\normalfont}
\newtheorem{definition}{Definition}[Sec]
\newtheorem{satz}[definition]{Satz}
\newtheorem{defsatz}[definition]{Definition und Satz}
\newtheorem{verfahren}[definition]{Verfahren}
\newtheorem{defver}[definition]{Definition und Verfahren}
\newtheorem{defsatzver}[definition]{Definition, Satz und Verfahren}
\newtheorem{satzver}[definition]{Satz und Verfahren}
\newtheorem{folgerung}[definition]{Folgerung}

\MakeOuterQuote{"}
\DeclareMathOperator{\ffa}{ffa}

\newcommand{\sep}{%
	\rule{\textwidth}{0.3pt}%
	\stepcounter{Sec}%
	}
\newcommand{\defiff}{\mathrel{\vcentcolon\Longleftrightarrow}}

\newcommand{\en}{~(n\to\infty)}
\newcommand{\series}[1][1]{\sum_{n=#1}^\infty}
\newcommand{\ps}[1][a]{\series[0]#1_n(x-x_0)^n}
\renewcommand{\d}{\dd}
\renewcommand{\P}{\mathcal{P}}
\newcommand{\R}{\mathbb{R}}
\newcommand{\C}{\mathbb{C}}
\newcommand{\A}{\mathfrak{A}}
\newcommand{\B}{\mathfrak{B}}
\renewcommand{\i}{\mathrm{i}}
\newcommand{\D}{\mathbb{D}}
\newcommand{\compl}[1]{#1^\mathsf{c}}
\newcommand{\sa}{$\sigma$-Algebra}
\newcommand{\Ri}{\mathfrak{R}}

\renewcommand{\L}[1]{\mathfrak{L}^{#1}(X)}
\newcommand{\LL}[1]{L^{#1}(X)}

\newcolumntype{M}[1]{>{\centering\arraybackslash}m{#1}}
\newcolumntype{N}{@{}m{0pt}@{}}

\setlength\columnsep{1.5cm}

\DeclareMathOperator{\Spek}{Spek}
\DeclareMathOperator{\Kern}{Kern}
\DeclareMathOperator{\arsinh}{arsinh}
\DeclareMathOperator{\arcosh}{arcosh}
\DeclareMathOperator{\artanh}{artanh}
\DeclareMathOperator{\ddiv}{div}
\DeclareMathOperator{\rot}{rot}
\DeclareMathOperator*{\esssup}{ess\,sup}
\DeclareMathOperator{\supp}{supp}
\DeclareMathOperator{\Arg}{Arg}
\DeclareMathOperator{\Log}{Log}
\DeclareMathOperator{\dist}{dist}

\DeclarePairedDelimiterX\set[1]\lbrace\rbrace{\def\given{\;\delimsize\vert\;}#1}

\newcommand\restr[2]{{#1_{\mkern 1mu \vrule height 2ex\mkern2mu #2}}}

\begin{document}
	\textsc{Analysis I}

	\sep
	\begin{defsatz}
		Zu $\varnothing\neq M\subseteq\mathbb{R}$ heißt $\gamma\in\mathbb{R}$
		\begin{description}
			\item[Obere Schranke] $\defiff \forall x\in M: \gamma\geq x$
			\item[Supremum] $\defiff \gamma~\text{ist OS} \wedge \forall~\text{OS $x$ von $M$}: \gamma\leq x$
			\item[Maximum] $\defiff \gamma = \sup M\in M$
		\end{description}
		Besitzt $M$ eine obere Schranke, besitzt es auch ein Supremum.
	\end{defsatz}
	\begin{satz}
		Für $\varnothing\neq M\subseteq\mathbb{R}$ und obere Schranke $\gamma$ von $M$ gilt
		$\gamma = \sup M\iff\forall\varepsilon>0~\exists x\in M:x>\gamma-\varepsilon$.
	\end{satz}
	\sep
	\begin{satz}[Betragssätze]
		$\forall a,b\in\mathbb{R}:$
		\begin{enumerate}[label=(\alph*)]
			\item $\abs{ab}=\abs{a}\abs{b}$
			\item $\pm a\leq\abs{a}$
			\item $\abs{a+b}\leq\abs{a}+\abs{b}$
		\end{enumerate}
	\end{satz}
	\begin{defsatz}
		\begin{enumerate}[label=(\alph*)]
			\item Für $a\in\mathbb{R}$ ist $[a]$ mit $[a]\in\mathbb{Z}\wedge [a]\leq a<[a]+1$ existent und eindeutig.
			\item Für $x,y\in\mathbb{R}$ mit $x<y$ existiert $r\in\mathbb{Q}$ mit $x<r<y$.
		\end{enumerate}
	\end{defsatz}
	\sep
	\begin{definition}[Folge]
		Eine Abbildung $\mathbb{N}\to B$ heißt Folge in $B$.
	\end{definition}
	\begin{defsatz}
		Eine nichtleere Menge $B$ heißt
		\begin{description}
			\item[endlich] $\defiff\exists n\in\mathbb{N},f\colon \set{1,\ldots,n}\to B~\text{surjektiv}$
			\item[abzählbar] $\defiff\exists f\colon\mathbb{N}\to B~\text{surjektiv}$
		\end{description}
		Die abzählbaren Mengen sind über dem kartesischen Produkt und der
		Vereinigung abzählbar vieler Mengen abgeschlossen.
	\end{defsatz}
	\sep
	\begin{defsatz}[Binomischer Lehrsatz]
		Für $n, k\in\mathbb{N}$ ist $\binom{n}{k}\coloneqq\frac{n!}{k!(n-k)!}$.
		Für $a, b\in\mathbb{R}, n\in\mathbb{N}$ gilt
		\[(a+b)^n = \sum_{k=0}^n\binom{n}{k}a^{n-k}b^k\]
	\end{defsatz}
	\begin{satz}[Bernoullische Ungleichung]
		Für $n\in\mathbb{N}, -1\leq x\in\mathbb{R}$ gilt $(1+x)^n\geq1+nx$.
	\end{satz}
	\sep
	\begin{definition}
		Eine Folge heißt beschränkt, wenn die Menge der Folgengleider beschränkt
		ist. $\sup_{n\in\mathbb{N}}a_n\coloneqq\sup\set{a_n\given n\in\mathbb{N}}$.
	\end{definition}
	\begin{defsatz}[Konvergenz]
		Eine Folge $(a_n)$ in $\mathbb{R}$ heißt konvergent, wenn ein $a\in\mathbb{R}$ existiert, sodass
		\[\lim_{n\to\infty}(a_n) = a \defiff a_n\to a\en \defiff
			\forall\varepsilon>0~\exists n_0\in\mathbb{N}~\forall n>n_0:\abs{a_n-a}<\varepsilon\]
		Falls der Grenzwert $a$ existiert, ist er eindeutig und $(a_n)$ ist beschränkt.
	\end{defsatz}
	\begin{satz}
		Es seien $(a_n), (b_n)$ Folgen in $\mathbb{R}$ und $a\in\mathbb{R}$.
		\begin{enumerate}[label=(\alph*)]
			\item Aus $(a_n) = (b_n) \ffa n\in\mathbb{N}$ folgt für $a_n\to a\en \iff b_n\to a\en$.
			\item $a_n\to a\en \iff \abs{a_n-a}\to0\en$.
		\end{enumerate}
	\end{satz}
	\begin{satz}[Konvergenzsätze]
		Es seien $(a_n), (b_n), (c_n)$ Folgen in $\mathbb{R}$ mit $a_n\to a,b_n\to b\en$.
		\begin{enumerate}[label=(\alph*)]
			\item $a_n\leq b_n\ffa n\in\mathbb{N}\implies a\leq b$.
			\item $a=b\wedge a_n\leq c_n\leq b_n\ffa n\in\mathbb{N}\implies c_n\to a\en$.
			\item $\abs{a_n}\to\abs{a}\en$.
			\item $a_n+b_n\to a+b\en$.
			\item $\alpha a_n\to\alpha a\en$ für $\alpha\in\mathbb{R}$.
			\item $a_n b_n\to ab\en$.
			\item $b\neq0\implies \exists m\in\mathbb{N}: (\forall n\geq m: b_n\neq0) \wedge ((\frac{1}{b_n})_{n\geq m}\to\frac{1}{b}\en)$.
		\end{enumerate}
	\end{satz}
	\begin{definition}
		Eine Folge $(a_n)\in\mathbb{R}$ heißt
		\begin{description}
			\item[monoton wachsend] $\defiff\forall n\in\mathbb{N}:a_{n+1}\geq a_n$
			\item[streng monoton wachsend] $\defiff\forall n\in\mathbb{N}:a_{n+1}>a_n$
		\end{description}
	\end{definition}
	\begin{satz}
		Ist $(a_n)$ streng monoton wachsend und nach oben beschränkt, dann ist $\lim_{n\to\infty} a_n=\sup_{n\in\mathbb{N}}a_n$.
	\end{satz}
	\begin{satz}
		Für $(a_n)$ Folge in $\mathbb{R}$ mit $a_n>0~\forall n\in\mathbb{N}$ und $a_n\to a\en$,
		$p\in\mathbb{N}$ mit $p\geq2$. Dann $\sqrt[p]{a_n}\to\sqrt[p]{a}\en$.
	\end{satz}
	\begin{satz}
		Es sei $x\in\mathbb{R}$ und $\forall n\in\mathbb{N}:a_n\coloneqq x^n$.
		\begin{enumerate}[label=(\alph*)]
			\item $(a_n)~\text{konvergiert}\iff x\in(-1,1]$.
			\item $\lim_{n\to\infty}a_n=0\iff x\in(-1,1)$.
			\item $\lim_{n\to\infty}a_n=1\iff x =1$.
		\end{enumerate}
	\end{satz}
	\begin{satz}
		Für $x\in\mathbb{R},n\in\mathbb{N}$ gilt \[s_n\coloneqq\sum_{k=0}^nx^k=\begin{cases}
			n+1 &\text{falls $x=1$}\\
			\frac{1-x^{n+1}}{1-x} &\text{sonst}
		\end{cases}\]
		$(s_n)~\text{konvergiert}\iff\abs{x}<1$. In diesem Fall $\lim_{n\to\infty}s_n=\frac{1}{1-x}$.
	\end{satz}
	\begin{satz}
		$\sqrt[n]{n}\to1\en$. $\forall c>0:\sqrt[n]{c}\to1\en$.
	\end{satz}
	\begin{defsatz}[Die eulersche Zahl]
		\[ e\coloneqq \lim_{n\to\infty}(1+\frac{1}{n})^n = \lim_{n\to\infty}\sum_{k=0}^n\frac{1}{k!}\]
	\end{defsatz}
	\sep
	\begin{definition}[$\varepsilon$-Umgebung]
		$\forall\alpha\in\mathbb{R},\varepsilon>0:U_\varepsilon(\alpha)\coloneqq\set{a\in\mathbb{R}\given\abs{a-\alpha}<\varepsilon}
		=(\alpha-\varepsilon,\alpha+\varepsilon)$.
	\end{definition}
	\begin{definition}[Häufungswert]
		Es sei $(a_n)$ Folge in $\mathbb{R}$. $\alpha\in\mathbb{R}~\text{ist Häufungswert von $(a_n)$}\defiff\forall\varepsilon>0:a_n\in
		U_\varepsilon(\alpha)~\text{für unendlich viele}~n\in\mathbb{N}$. $\mathcal{H}(a_n)\coloneqq\set{\alpha\in\mathbb{R}\given
		\alpha~\text{ist Häufungswert von $(a_n)$}}$.
	\end{definition}
	\begin{definition}[Teilfolge]
		Für $(a_n)_n$ Folge in $\mathbb{R}$ und $(n_k)_k$ streng monoton wachsende
		Folge in $\mathbb{N}$ heißt $(a_{n_k})_k$ Teilfolge von $(a_n)$.
	\end{definition}
	\begin{satz}
		Für Folge $(a_n)$ in $\mathbb{R}$ und $\alpha\in\mathbb{R}$ gilt
		$\alpha\in\mathcal{H}(a_n)\iff\exists (n_k)_k~\text{streng monoton wachsend in}~\mathbb{N}:a_{n_k}\to\alpha~(k\to\infty)$.
	\end{satz}
	\begin{satz}
		Für $(a_n)$ konvergent gilt $\mathcal{H}(a_n)=\set{\lim_{n\to\infty} a_n}$. Für jede Teilfolge $(a_{n_k})$ gilt
		$\lim_{k\to\infty} a_{n_k} = \lim_{n\to\infty} a_n$.
	\end{satz}
	\begin{satz}[von Bolzano-Weierstraß]
		Für $(a_n)$ beschränkt gilt $\mathcal{H}(a_n)\neq\varnothing$.
	\end{satz}
	\sep
	\begin{defsatz}[Oberer und unterer Limes]
		Für $(a_n)$ beschränkte Folge in $\mathbb{R}$ existieren $\lim\sup a_n\coloneqq\max\mathcal{H}(a_n)$
		und $\lim\inf a_n\coloneqq\min\mathcal{H}(a_n)$.
	\end{defsatz}
	\begin{satz}
		Für beschränkte Folge $(a_n)$ in $\mathbb{R}$ sind äquivalent:
		\begin{enumerate}[label=(\alph*)]
			\item $\lim\inf a_n=\lim\sup a_n$
			\item $\abs{\mathcal{H}(a_n)} = 1$
			\item $(a_n)$ ist konvergent
		\end{enumerate}
	\end{satz}
	\sep
	\begin{defsatz}[Cauchy-Kriterium]
		Eine Folge $(a_n)$ in $\mathbb{R}$ heißt Cauchyfolge, wenn
		$\forall\varepsilon>0~\exists n_0\in\mathbb{N}~\forall n,m\geq n_0:\abs{a_n-a_m}<\varepsilon$.
		Dies ist genau dann der Fall, wenn $(a_n)$ konvergent ist.
	\end{defsatz}
	\sep
	\begin{definition}[Unendliche Reihe]
		Ist $(a_n)$ eine Folge in $\mathbb{R}$, so heißt die Folge $\series a_n\coloneqq(s_n)$
		mit $\forall n\in\mathbb{N}:s_n\coloneqq\sum_{k=1}^na_k$ eine (unendliche)
		Reihe. Falls $\series a_n$ konvergent, so ist der Reihenwert
		$\series a_n\coloneqq\lim_{n\to\infty}s_n$.
	\end{definition}
	\begin{satz}
		Ist $(a_n)$ eine Folge in $\mathbb{R}$ und $\series a_n$ konvergent,
		so ist $\lim a_n=0$.
	\end{satz}
	\begin{defsatz}[Absolute Konvergenz]
		Eine Reihe $\series a_n$ ist genau dann absolut konvergent, wenn
		$\series\abs{a_n}$ konvergent ist. Eine absolut konvergente
		Reihe ist konvergent und es gilt
		$\abs{\series a_n}\leq\series\abs{a_n}$.
	\end{defsatz}
	\begin{satz}[Leibnizkriterium]
		Für monoton fallende Nullfolge $(a_n)$ ist $\series (-1)^na_n$
		konvergent.
	\end{satz}
	\begin{satz}[Majorantenkriterium]
		$\abs{a_n}\leq b_n\ffa n\in\mathbb{N}\wedge\series b_n~\text{konvergiert}
		\implies\series a_n~\text{konvergiert absolut}$.
	\end{satz}
	\begin{satz}[Minorantenkriterium]
		$a_n\geq b_n\geq0\ffa n\in\mathbb{N}\wedge\series b_n~\text{divergiert}
		\implies\series a_n~\text{divergiert}$.
	\end{satz}
	\begin{satz}[Wurzelkriterium]
		Für Folge $(a_n)$ in $\mathbb{R}$ setze $\alpha\coloneqq\lim\sup\sqrt[n]{\abs{a_n}}$.
		\begin{enumerate}[label=(\alph*)]
			\item $\alpha<1\implies\series a_n~\text{konvergiert absolut}$
			\item $\alpha>1\implies\series a_n~\text{divergiert}$
		\end{enumerate}
	\end{satz}
	\begin{satz}[Quotientenkriterium]
		Für Folge $(a_n)$ in $\mathbb{R}$ mit $a_n\neq0\ffa n\in\mathbb{N}$ definiere
		$(\alpha_n)$ mit $\alpha_n\coloneqq\frac{a_{n+1}}{a_n}\ffa n\in\mathbb{N}$.
		\begin{enumerate}[label=(\alph*)]
			\item $\abs{\alpha_n}\geq1\ffa n\in\mathbb{N}\implies\series a_n~\text{ist divergent}$
			\item Falls $(\alpha_n)$ beschränkt, $\beta\coloneqq\lim\inf\abs{\alpha_n}$,
			$\alpha\coloneqq\lim\sup\abs{\alpha_n}$.
			\begin{enumerate}[label=(\roman*)]
				\item $\beta>1\implies\series a_n~\text{ist divergent}$
				\item $\alpha<1\implies\series a_n~\text{ist absolut konvergent}$
			\end{enumerate}
		\end{enumerate}
	\end{satz}
	\sep
	\begin{definition}
		Ist $(a_n)$ eine Folge und $\sigma\colon\mathbb{N}\to\mathbb{N}$ bijektiv,
		dann heißt $(b_n)$ mit $b_n\coloneqq a_{\sigma(n)}$ eine Umordnung von
		$(a_n)$. Die Reihe über die Umordnung wird als Umordnung der Reihe bezeichnet.
	\end{definition}
	\begin{satz}
		$(b_n)$ sei eine Umordnung von $(a_n)$.
		\begin{enumerate}[label=(\alph*)]
			\item $(a_n)~\text{konvergent}\implies\lim a_n=\lim b_n$
			\item $\series a_n~\text{absolut konvergent}\implies\series b_n~\text{absolut konvergent}
				\wedge\series a_n=\series b_n$
		\end{enumerate}
	\end{satz}
	\begin{satz}[Riemannscher Umordnungssatz]
		Ist $\series a_n$ konvergent,
		aber nicht absolut konvergent, so existiert für jedes $s\in\mathbb{R}\cup
		\set{\infty,-\infty}$ eine Umordnung $(b_n)$ von $(a_n)$ mit
		$\series b_n = s$.
	\end{satz}
	\sep
	\begin{defsatz}[Produktreihe]
		$\series[0] p_n$ heißt Produktreihe von $\series[0] a_n$ und
		$\series[0] b_n$, wenn $\set{p_n\given n\in\mathbb{N}_0} = \set{a_jb_k\given j,k\in\mathbb{N}_0}$,
		weobei jedes $a_jb_k$ genau einmal vorkommt. Alle Produktreihen sind
		Umordnungen voneinander.
	\end{defsatz}
	\begin{definition}[Cauchyprodukt]
		Die unendliche Reihe $\series[0] c_n\coloneqq\series[0] \sum_{k=0}^n a_kb_{n-k}$ heißt
		Cauchyprodukt von $\series[0] a_n$ und $\series[0] b_n$.
	\end{definition}
	\begin{satz}
		Sind $\series[0] a_n$ und $\series[0] b_n$ absolut konvergent, so sind alle
		Produktreihen $\series[0] p_n$ und das Cauchyprodukt $\series[0] c_n$
		absolut konvergent mit
		$\series[0] p_n = \series[0] c_n = (\series[0] a_n)(\series[0] b_n)$.
	\end{satz}
	\begin{defsatz}[Die Exponentialfunktion]
		Die Definition $\forall x\in\mathbb{R}:e^x\coloneqq E(x)\coloneqq\series[0]\frac{x^n}{n!}$
		deckt sich mit der Potenz für rationale Exponenten.
	\end{defsatz}
	\begin{definition}[Kosinus und Sinus]
		$\forall x\in\mathbb{R}$:
		\begin{align*}
			\cos x &\coloneqq\series[0](-1)^n\frac{x^{2n}}{(2n)!}\\
			\sin x &\coloneqq\series[0](-1)^n\frac{x^{2n+1}}{(2n+1)!}
		\end{align*}
	\end{definition}
	\sep
	\begin{definition}[Potenzreihe]
		Für Folge $(a_n)_{n=0}^\infty$ in $\mathbb{R}$ und $x_0\in\mathbb{R}$
		heißt $\ps$ Potenzreihe.
	\end{definition}
	\begin{satz}[Konvergenz von Potenzreihen]
		Für Potenzreihe $\ps$ setze
		$\rho\coloneqq\lim\sup\sqrt[n]{\abs{a_n}}$ und $r\coloneqq\frac{1}{\rho}$,
		wobei $r\coloneqq0$, falls $\rho=\infty$ und $r\coloneqq\infty$, falls
		$\rho=0$.
		\begin{enumerate}[label=(\alph*)]
			\item Falls $r=0$ konvergiert die Potenzreihe genau für $x=x_0$.
			\item Falls $r=\infty$ konvergiert die Potenzreihe absolut für alle $x\in\mathbb{R}$.
			\item Falls $r\in\mathbb{R}$ konvergiert die Potenzreihe für $x \in (x_0 - r, x_0 + r)$
				absolut und divergiert für $x \in (-\infty, x_0-r) \cup (x_0+r, \infty)$. Für
				$x\in\set{x_0+r,x_0-r}$ ist keine allgemeine Aussage möglich.
		\end{enumerate}
		$r$ wird auch als der Konvergenzradius der Potenzreihe bezeichnet.
	\end{satz}
	\begin{satz}[Konvergenzradien von Cauchyprodukten]
		Für Potenzeihen $\ps$ und $\ps[b]$ mit Konvergenzradien $r_1$ und $r_2$
		setze $R\coloneqq\min\set{r_1,r_2}$, $c_n\coloneqq\sum_{k=0}^na_kb_{n-k}$.
		$r$ sei der Konvergenzradius von $\ps[c]$. Dann ist $R\leq r$ und für
		$x\in(x_0-R,x_0+R)$ gilt $\ps[c]=(\ps)(\ps[b])$.
	\end{satz}
	\sep
	\begin{definition}[Häufungspunkt]
		$x_0~\text{ist Häufungspunkt von}~D\subseteq\mathbb{R}
		\defiff\exists(x_n)~\text{in $D\setminus\set{x_0}$}:x_n\to x_0\en$
	\end{definition}
	\begin{definition}
		Für $\varnothing\neq D\subseteq\mathbb{R}$, $\delta>0$ sind
		$D_\delta(x_0)\coloneqq D\cap U_\delta(x_0)$, $\dot{D}_\delta(x_0)\coloneqq
		D_\delta(x_0)\setminus\set{x_0}$.
	\end{definition}
	\begin{definition}
		Ist $(a_n)$ Folge in $\mathbb{R}$, so ist
		$\lim a_n = \infty\defiff \forall c>0~\exists n_0\in\mathbb{N}~\forall n\geq n_0:a_n>c$.
	\end{definition}
	\begin{defsatz}[Funktionsgrenzwert]
		Für $\varnothing\neq D\subseteq\mathbb{R}$, $x_0$ Häufungspunkt von $D$,
		$f\colon D\to\mathbb{R}$ ist
		\begin{align*}
			\lim_{x\to x_0}f(x) = a
			\defiff&~\forall(x_n)~\text{in}~D\setminus\set{x_0}~\text{mit}~x_n\to x_0\en:f(x_n)\to a\en\\
			\Longleftrightarrow&~\forall\varepsilon>0~\exists\delta>0~\forall x\in\dot{D}_\delta(x_0):\abs{f(x)-a}<\varepsilon
		\end{align*}
		Existiert $\lim_{x\to x_0}f(x)$, dann ist er eindeutig.
	\end{defsatz}
	\sep
	\begin{defsatz}[Stetigkeit]
		Für $\varnothing\neq D\subseteq\mathbb{R}$, $x_0\in D$, $f\colon D\to\mathbb{R}$
		ist
		\begin{align*}
			f~\text{stetig in}~x_0
			\defiff&~\forall(x_n)~\text{in}~D~\text{mit}~x_n\to x_0\en:f(x_n)\to f(x_0)\en\\
			\Longleftrightarrow&~\forall\varepsilon>0~\exists\delta>0~\forall x\in D_\delta(x_0):\abs{f(x)-f(x_0)}<\varepsilon
		\end{align*}
	\end{defsatz}
	\begin{defsatz}
		Für $\varnothing\neq D\subseteq\mathbb{R}$, $f\colon D\to\mathbb{R}$
		ist
		\begin{align*}
			f\in C(D)\defiff
			f~\text{stetig auf $D$}
			\defiff&~\forall x_0\in D:f~\text{stetig in $x_0$}\\
			\Longleftrightarrow&~\forall x_0\in D~\forall\varepsilon>0~\exists\delta>0~\forall x\in D_\delta(x_0):\abs{f(x)-f(x_0)}<\varepsilon\\
			\Longleftrightarrow&~\forall\varepsilon>0~\forall x\in D~\exists\delta>0~\forall z\in D~\text{mit}~\abs{x-z}<\delta:\abs{f(x)-f(z)}<\varepsilon\\
			f~\text{gleichmäßig stetig auf $D$}
			\defiff&~\forall\varepsilon>0~\exists\delta>0~\forall x\in D~\forall z\in D~\text{mit}~\abs{x-z}<\delta:\abs{f(x)-f(z)}<\varepsilon\\
			f~\text{Lipschitz stetig auf $D$}
			\defiff&~\exists L>0~\forall x,z\in D:\abs{f(x)-f(z)}\leq L\abs{x-z}
		\end{align*}
		Aus der Lipschitz Stetigkeit folgt die gleichmäßige Stetigkeit.
		Aus der gleichmäßigen Stetigkeit folgt die Stetigkeit.
	\end{defsatz}
	\begin{satz}
		Ist $x_0$ ein Häufungspunkt von $D$, dann gilt $f~\text{stetig in}~x_0\iff\lim_{x\to x_0}f(x)=f(x_0)$.
	\end{satz}
	\begin{satz}
		Sind $f\colon D\to\mathbb{R}$, $g\colon D\to\mathbb{R}$ stetig in $x_0$, so
		sind $f + g$, $fg$, $\abs{f}$ stetig in $x_0$. Ist $\tilde{D}\coloneqq\set{x\in D\given f(x)\neq0}$
		und $x_0\in\tilde{D}$, so ist $\frac{1}{f}\colon\tilde{D}\to\mathbb{R}$ stetig in
		$x_0$.
	\end{satz}
	\begin{satz}
		Es sei $\varnothing\neq D, E\subseteq\mathbb{R}$, $f\colon D\to E$ stetig in $x_0$,
		$g\colon E\to\mathbb{R}$ stetig in $f(x_0)$. Dann ist $g\circ f\colon D\to\mathbb{R}$
		stetig in $x_0$.
	\end{satz}
	\begin{satz}
		Hat $\ps$ den Konvergenzradius $r>0$, setze $D\coloneqq(x_0-r,x_0+r)$,
		$f\colon D\to\mathbb{R}$ mit $\forall x\in D:f(x)\coloneqq\ps$. Dann
		ist $f\in C(D)$ und für $\tilde{x}\in D$ gilt
		\[ \lim_{x\to\tilde{x}}\ps = \lim_{x\to\tilde{x}}f(x)=f(\tilde{x})=
		\sum_{n=0}^\infty a_n(\tilde{x}-x_0)^n=\sum_{n=0}^\infty\lim_{x\to\tilde{x}}a_n(x-x_0)^n \]
	\end{satz}
	\begin{satz}[von Heine]
		Ist $D$ beschränkt und abgeschlossen und $f\in C(D)$, so ist $f$ auf
		$D$ gleichmäßig stetig.
	\end{satz}
	\sep
	\begin{satz}[Der Zwischenwertsatz]
		Ist $a<b$, $f\in C[a,b]$, $y_0\in\mathbb{R}$, $\min\set{f(a),f(b)}\leq
		y_0\leq\max\set{f(a),f(b)}$, dann existiert $x_0\in[a,b]$ mit $f(x_0)=y_0$.
	\end{satz}
	\begin{satz}[Nullstellensatz von Bolzano]
		Ist $f\in C[a,b]$ und $f(a)f(b)<0$, dann existiert $x_0\in[a,b]$ mit $f(x_0)=0$.
	\end{satz}
	\begin{definition}
		$A\subseteq\mathbb{R}$ heißt
		\begin{description}
			\item[abgeschlossen] $\defiff\forall(x_n)~\text{in}~A~\text{konvergent}:\lim x_n\in A$
			\item[offen] $\defiff\forall x\in A~\exists\delta>0:U_\delta(x)\subseteq A$
		\end{description}
	\end{definition}
	\begin{satz}
		Ist $D$ beschränkt und abgeschlossen und $f\in C(D)$, so nimmt $f$ auf $D$
		ein Minimum und ein Maximum an.
	\end{satz}
	\begin{satz}
		Ist $I\subseteq\mathbb{R}$ ein Intervall und $f\in C(I)$, dann ist
		$f(I)$ ein Intervall. Ist $f$ zusätzlich streng monoton, dann ist $f$
		umkehrbar und $f^{-1}\in C(f(I))$.
	\end{satz}
	\begin{defsatz}[Der Logarithmus]
		Ist $f\colon\mathbb{R}\to\mathbb{R}$ mit $f(x)\coloneqq e^x$, so existiert
		$\log \coloneqq f^{-1}\colon (0,\infty)\to\mathbb{R}$.
	\end{defsatz}
	\begin{definition}[Die allgemeine Potenz]
		Für alle $a>0, x\in\mathbb{R}$ ist $a^x\coloneqq e^{x\log a}$.
	\end{definition}
	\sep
	\begin{defsatz}[Punktweise und gleichmäßige Konvergenz]
		Es sei $\varnothing\neq D\subseteq\mathbb{R}$ und $(f_n)$ eine Folge mit
		$\forall n\in\mathbb{N}:f_n\colon D\to\mathbb{R}$. Ferner sei $f\colon D\to\mathbb{R}$.
		\begin{align*}
			f_n\to f~\text{pktw}
			\defiff&~\forall x\in D:\lim f_n(x)=f(x)\\
			\Longleftrightarrow&~\forall\varepsilon>0~\forall x\in D~\exists n_0\in\mathbb{N}~\forall n\geq n_0:\abs{f_n(x)-f(x)}<\varepsilon\\
			f_n\to f~\text{glm}
			\defiff&~\forall\varepsilon>0~\exists n_0\in\mathbb{N}~\forall n\geq n_0~\forall x\in D:\abs{f_n(x)-f(x)}<\varepsilon
		\end{align*}
		Aus der gleichmäßigen Konvergenz folgt stets die punktweise Konvergenz.
	\end{defsatz}
	\begin{satz}
		Konvergiert $f_n$ punktweise gegen $f\colon D\to\mathbb{R}$, dann gilt
		\[ f_n\to f~\text{glm}\iff\exists(\alpha_n)~\text{in $\mathbb{R}$ mit $\alpha_n\to0\en$}~\exists
		m\in\mathbb{N}~\forall n\geq m~\forall x\in\mathbb{R}:\abs{f_n(x)-f(x)}\leq\alpha_n\]
	\end{satz}
	\begin{satz}
		$f_n\to f~\text{glm}\iff\forall(x_n)~\text{in}~D:f_n(x_n)-f(x_n)\to0\en$.
	\end{satz}
	\begin{satz}
		Es sei $\varnothing\neq D\subseteq\mathbb{R}$ und $(f_n)$ eine Folge mit
		$\forall n\in\mathbb{N}:f_n\colon D\to\mathbb{R}$. Ferner sei $f\colon D\to\mathbb{R}$
		mit $f_n\to f~\text{glm}$.
		\begin{enumerate}[label=(\alph*)]
			\item Ist $x_0\in D$ und alle $f_n$ stetig in $x_0$, so ist $f$ stetig in $x_0$.
			\item $(\forall n\in\mathbb{N}: f_n\in C(D)) \implies f\in C(D)$.
			\item Ist $x_0$ Häufungspunkt von $D$, dann gilt
			\[ \lim_{x\to x_0}\lim_{n\to\infty}f_n(x)=\lim_{x\to x_0}f(x)=f(x_0)=\lim_{n\to\infty}f_n(x_0)=\lim_{n\to\infty}\lim_{x\to x_0}f_n(x) \]
		\end{enumerate}
	\end{satz}
	\begin{satz}
		Es sei $a,b\in\mathbb{R}$, $a<b$, $(f_n)$ eine Folge in $R[a,b]$,
		$f\colon[a,b]\to\mathbb{R}$, $f_n\to f~\text{glm}$. Dann $f\in R[a,b]$ mit
		\[ \lim_{n\to\infty}\int_a^bf_n(x)\d{x}=\int_a^bf(x)\d{x}=\int_a^b(\lim_{n\to\infty}f_n(x))\d{x} \]
	\end{satz}
	\begin{satz}
		Es sei $a,b\in\mathbb{R}$, $a<b$, $(f_n)$ eine Folge in $C^1[a,b]$,
		$x_0\in[a,b]$, $(f_n(x_0))$ konvergiere, es sei $g\colon[a,b]\to\mathbb{R}$
		und es sei $f'_n\to g~\text{glm}$
		auf $[a,b]$. Dann $f_n\to f~\text{glm}$ mit $f\colon[a,b]\to\mathbb{R}$,
		$\forall x\in[a,b]: f(x)\coloneqq\lim_{n\to\infty}f_n(x)$ und es ist $\forall x\in[a,b]$:
		\[ (\lim_{n\to\infty}f_n(x))'=f'(x)=\lim_{n\to\infty}f_n'(x) \]
	\end{satz}
	\sep
	\begin{defsatz}[Differenzierbarkeit]
		Es sei $I\subseteq\mathbb{R}$ ein Intervall und $f\colon I\to\mathbb{R}$.
		$f$ heißt in $x_0\in I$ differenzierbar, wenn
		$f'(x_0)\coloneqq\lim_{x\to x_0}\frac{f(x)-f(x_0)}{x-x_0}$ existiert.
		$f$ heißt auf $I$ differenzierbar, wenn $\forall x_0\in I:f~\text{db in}~x_0$.
		In diesem Fall heißt die so definierte Funktion $f'\colon I\to\mathbb{R}$ die
		Ableitung von $f$. Aus der Differenzierbarkeit folgt die Stetigkeit.
	\end{defsatz}
	\begin{satz}
		Es sei $I\subseteq\mathbb{R}$ ein Intervall und $f,g\colon I\to\mathbb{R}$
		differenzierbar in $x_0\in I$.
		\begin{enumerate}[label=(\alph*)]
			\item Für $\alpha,\beta\in\mathbb{R}$ ist $(\alpha f+\beta g)'(x_0)=\alpha f'(x_0)+\beta g'(x_0)$.
			\item $fg$ ist differenzierbar in $x_0$ mit $(fg)'(x_0)=f'(x_0)g(x_0)+f(x_0)g'(x_0)$.
			\item Falls $\forall x\in I:g(x)\neq0$, ist $\frac{f}{g}$ differenzierbar in $x_0$ mit
				\[ (\frac{f}{g})'(x_0) = \frac{f'(x_0)g(x_0)-f(x_0)g'(x_0)}{g(x_0)^2} \]
		\end{enumerate}
	\end{satz}
	\begin{satz}
		Es seien $I, J\subseteq\mathbb{R}$ Intervalle, $f\colon I\to\mathbb{R}$
		$g\colon J\to I$, $g$ differenzierbar in $x_0\in J$, $f$ differenzierbar
		in $g(x_0)$. Dann ist $f\circ g\colon J\to\mathbb{R}$ differenzierbar in $x_0$
		mit $(f \circ g)'(x_0)=f'(g(x_0))g'(x_0)$.
	\end{satz}
	\begin{satz}
		Es sei $I\subseteq\mathbb{R}$ ein Intervall, $f\in C(I)$ streng monoton und
		differenzierbar in $x_0\in I$ und $f'(x_0)\neq0$. Dann ist $f^{-1}\colon f(I)\to I$
		differenzierbar in $f(x_0)$ und $(f^{-1})'(f(x_0))=\frac{1}{f'(x_0)}$.
	\end{satz}
	\begin{definition}
		Mit $\varnothing\neq D\subseteq\mathbb{R}$, $g\colon D\to\mathbb{R}$
		ist $x_0\in D~\text{relatives Maximum von $g$}\defiff\exists\delta>0~\forall
		x\in D\cap U_\delta(x_0):g(x)\leq g(x_0)$.
	\end{definition}
	\begin{definition}
		Mit $\varnothing\neq M\subseteq\mathbb{R}$ ist
		$x_0\in M~\text{innerer Punkt von $M$}\defiff\exists\delta>0:U_\delta(x_0)\subseteq M$.
	\end{definition}
	\begin{satz}
		Ist $I\subseteq\mathbb{R}$ ein Intervall, $f\colon I\to\mathbb{R}$ differenzierbar
		in $x_0\in I$, $x_0$ ein relatives Extremum von $f$ und $x_0$ ein innerer
		Punkt von $I$, so ist $f'(x_0)=0$.
	\end{satz}
	\begin{satz}
		Es sei $I=[a,b]$, $f, g\in C(I)$, $f$, $g$, differenzierbar auf $(a,b)$,
		$\forall x\in(a,b): g'(x)\neq0$.
		\begin{enumerate}[label=(\alph*)]
			\item $f(a)=f(b)\implies\exists\xi\in(a,b):f'(\xi)=0$.
			\item $\exists\xi\in(a,b):f(b)-f(a)=f'(\xi)(b-a)$.
			\item $g(a)\neq g(b)\wedge\exists\xi\in(a,b):\frac{f(b)-f(a)}{g(b)-g(a)}=\frac{f'(\xi)}{g'(\xi)}$.
		\end{enumerate}
	\end{satz}
	\begin{satz}[Die Regel von l'Hospital]
		Für $a\in\mathbb{R}\cup\set{-\infty}$, $b\in\mathbb{R}\cup\set{\infty}$
		seien $f,g\colon (a,b)\to\mathbb{R}$ auf $(a,b)$ differenzierbar mit
		$\forall x\in(a,b):g'(x)\neq0$. Es sei $c\in\set{a, b}$, es existiere
		$L\coloneqq\lim_{x\to c}\frac{f'(x)}{g'(x)}\in\mathbb{R}\cup\set{-\infty, \infty}$
		und es sei $\lim_{x\to c}f(x)=\lim_{x\to c}g(x)\in\set{0,-\infty,\infty}$.
		Dann $\lim_{x\to c}\frac{f(x)}{g(x)}=L$.
	\end{satz}
	\begin{satz}
		Sei $\ps$ eine Potenzreihe mit Konvergenzradius $r>0$, $I\coloneqq(x_0-r,x_0+r)$.
		Auf $I$ gilt dann $(\ps)'=\sum_{n=0}^\infty(a_n(x-x_0)^n)'$. Diese Potenzreihe
		hat den Konvergenzradius $r$.
	\end{satz}
	\sep
	\begin{definition}
		Für Intervall $I\subseteq\mathbb{R}$, $f\colon I\to\mathbb{R}$, ist für $x_0\in I$,
		falls existent, $f^{(n)}(x_0) \coloneqq ((\cdots(f\underbrace{')'\cdots)')'}_{n~\text{mal}}(x_0)$.
		Falls $\forall x\in I~\exists f^{(n)}(x)$, ist $f$ $n$-mal differenzierbar auf $I$.
		Ist zusätzlich $f^{(n)}\in C(I)$, ist $f$ $n$-mal stetig differenzierbar, also
		$f\in C^n(I)$. $C^0(I)\coloneqq C(I)$, $C^\infty(I)\coloneqq\bigcap_{n\in\mathbb{N}}C^n(I)$.
	\end{definition}
	\begin{satz}
		Ist $\ps$ eine Potenzreihe mit Konvergenzradius $r>0$, $I\coloneqq(x_0-r,x_0+r)$ und
		$f(x)\coloneqq\ps$, dann ist $f\in C^\infty(I)$, $f^{(k)}(x)=\sum_{n=k}^{\infty}\frac{n!}{(n-k)!}a_n(x-x_0)^{n-k}$.
		Es folgt $\forall k\in\mathbb{N}_0:a_k=\frac{f^{(k)}(x_0)}{k!}$.
	\end{satz}
	\begin{defsatz}
		Zu Intervall $I\subseteq\mathbb{R}$, $f\in C^\infty(I)$, $x_0\in I$ gehörige
		Taylerreihe ist $\sum_{k=0}^\infty\frac{f^{(k)}(x_0)}{k!}(x - x_0)^k$.
		Hin und wieder lässt sich $f$ in bestimmten Umgebungen um $x_0$ durch seine
		Taylorreihe approximieren.
	\end{defsatz}
	\begin{defsatz}[von Taylor]
		Für Intervall $I\subseteq\mathbb{R}$, $n\in\mathbb{N}_0$, $f\in  C^n(I)$,
		$x_0\in I$ ist $T_n(x;x_0)\coloneqq\sum_{k=0}^n\frac{f^{(k)}(x_0)}{k!}(x-x_0)^k$.
		Ist $f$ zusätzlich $n+1$-mal differenzierbar auf $I$ und $x\in I$, dann existiert
		$\xi$ zwischen $x$ und $x_0$ mit
		\[ f(x) = T_n(x;x_0)+\frac{f^{(n+1)}(\xi)}{(n+1)!}(x-x_0)^{n+1} \]
	\end{defsatz}
	\begin{satz}
		Ist $n\in\mathbb{N}$, $n>2$, $I\subseteq\mathbb{R}$ Intervall,
		$f\in C^n(I)$, $x_0\in I$ innerer Punkt von $I$, $\forall 1\leq i\leq n-1:f^{(i)}(x_0)=0$,
		$f^{(n)}(x_0)\neq0$, dann
		\begin{enumerate}[label=(\alph*)]
			\item $n~\text{gerade}\wedge f^{(n)}(x_0)<0\implies f~\text{hat ein relatives Maximum in}~x_0$
			\item $n~\text{gerade}\wedge f^{(n)}(x_0)>0\implies f~\text{hat ein relatives Minimum in}~x_0$
			\item $n~\text{ungerade}\implies f~\text{hat kein relatives Extremum in}~x_0$
		\end{enumerate}
	\end{satz}
	\sep
	\begin{definition}[Zerlegung]
		Ist $a,b\in\mathbb{R}$, $I\coloneqq[a,b]$, so heißt $Z=\set{x_0,\ldots,x_n}$ eine
		Zerlegung von $I$, wenn $a=x_0<x_1<\ldots<x_n=b$. Die Menge aller Zerlegungen
		wird mit $\mathfrak{Z}$ bezeichnet. Ist $f\colon I\to\mathbb{R}$ beschränkt, so gilt:
		$s_f(Z)\coloneqq\sum_{j=1}^n(\inf f(I_j))\abs{I_j}$ ist die Untersumme
		von $f$ bezüglich $Z$, $S_f(Z)\coloneqq\sum_{j=1}^n(\sup f(I_j))\abs{I_j}$
		ist die Obersumme von $f$ bezüglich $Z$.
	\end{definition}
	\begin{definition}[Das Riemann-Integral]
		\begin{align*}
			\underline{\int_{a}^{b}}f\d{x}\coloneqq\underline{\int_{a}^{b}}f(x)\d{x}\coloneqq&\sup\set{s_f(Z)\given Z\in\mathfrak{Z}}\\
			\overline{\int_{a}^{b}}f\d{x}\coloneqq\overline{\int_{a}^{b}}f(x)\d{x}\coloneqq&\inf\set{S_f(Z)\given Z\in\mathfrak{Z}}
		\end{align*}
		Falls $\underline{\int_{a}^{b}}f\d{x}=\overline{\int_{a}^{b}}f\d{x}$, ist $f\in R[a,b]$ und:
		\begin{align*}
			\int_{a}^{b}f\d{x}\coloneqq\int_{a}^{b}f(x)\d{x}\coloneqq\underline{\int_{a}^{b}}f\d{x}
		\end{align*}
	\end{definition}
	\begin{satz}
		Für $f,g\in R[a,b]$:
		\begin{enumerate}[label=(\alph*)]
			\item $f\leq g~\text{auf}~[a,b]\implies\int_a^bf\d{x}\leq\int_a^bg\d{x}$
			\item $\alpha,\beta\in\mathbb{R}\implies\alpha f+\beta g\in R[a,b]\wedge
				\int_a^b(\alpha f+\beta g)\d{x}=\alpha\int_a^bf\d{x}+\beta\int_a^bg\d{x}$
			\item $\abs{f}\in R[a,b]$, $\abs{\int_a^bf\d{x}}\leq\int_a^b\abs{f}\d{x}$.
			\item $fg\in R[a,b]$.
			\item $(\exists c\in\mathbb{R}~\forall x\in[a,b]:g(x)\neq0\wedge\abs{g(x)}<c)\implies\frac{1}{g}\in R[a,b]$.
		\end{enumerate}
	\end{satz}
	\begin{satz}
		\begin{enumerate}[label=(\alph*)]
			\item Jede monotone Funktion ist riemann-integrierbar.
			\item $C[a,b]\subseteq R[a,b]$.
		\end{enumerate}
	\end{satz}
	\begin{satz}
		 Ist $f\colon [a,b]\to\mathbb{R}$ beschränkt und $c\in(a,b)$, dann
		 $f\in R[a,b]\iff f\in R[a,c]\wedge f\in R[c,b]$. In diesem Fall
		 $\int_a^bf\d{x}=\int_a^cf\d{x}+\int_c^bf\d{x}$.
	\end{satz}
	\begin{satz}
		Es sei $f\in R[a,b]$, $D\coloneqq f([a,b])$, $h\colon D\to\mathbb{R}$ sei
		Lipschitzstetig auf $D$. Dann $h\circ f\in R[a,b]$.
	\end{satz}
	\begin{satz}
		\begin{enumerate}[label=(\alph*)]
			\item Ist $f\colon[a,b]\to\mathbb{R}$ beschränkt auf $[a,b]$ und
				$\set{x\in[a,b]\given f~\text{ist nicht stetig in}~x}$ endlich,
				dann ist $f\in R[a,b]$.
			\item Ist $f\in R[a,b]$ und $\set{x\in[a,b]\given f(x)\neq g(x)}$ endlich,
			dann ist $g\in R[a,b]$ und $\int_a^bg\d{x}=\int_a^bf\d{x}$.
		\end{enumerate}
	\end{satz}
	\begin{satz}[Erster Hauptsatz der Differential- und Integralrechnung]
		Es sei $f\in R[a,b]$ und $\forall x\in [a,b]:F'(x)=f(x)$. Dann
		\[ \int_a^bf(x)\d{x}=F(b)-F(a)\eqqcolon [F(x)]_a^b \]
	\end{satz}
	\begin{satz}[Zweiter Hauptsatz der Differential- und Integralrechnung]
		Mit $f\in R[a,b]$ und $F\colon[a,b]\to\mathbb{R}$, $F(x)\coloneqq\int_a^xf(t)\d{t}$:
		\begin{enumerate}[label=(\alph*)]
			\item $F$ ist auf $[a,b]$ Lipschitzstetig.
			\item $f~\text{in}~x_0\in[a,b]~\text{stetig}\implies F~\text{differenzierbar in $x_0$ mit}~F'(x_0)=f(x_0)$
			\item $f\in C[a,b]\implies F\in C^1[a,b]\wedge\forall x\in[a,b]:F'(x)=f(x)$
		\end{enumerate}
	\end{satz}
	\begin{satz}
		Ist $J\subseteq\mathbb{R}$ ein Intervall, $f\in C(J)$ und $\xi\in J$,
		$F\colon J\to\mathbb{R}$ mit $F(x)\coloneqq\int_\xi^xf(t)\d{t}$, dann ist
		$F\in C^1(J)$ und $F'=f$ auf $J$.
	\end{satz}
	\begin{satz}[Mittelwertsatz der Integralrechnung]
		Mit $f,g\in R[a,b]$, $g\geq0$ auf $[a,b]$, $m\coloneqq\inf f([a,b])$,
		$M\coloneqq\sup f([a,b])$:
		\begin{enumerate}[label=(\alph*)]
			\item $\exists\mu\in[m,M]:\int_a^bfg\d{x}=\mu\int_a^bg\d{x}$
			\item $f\in C[a,b]\implies\exists\xi\in[a,b]:\int_a^bf\d{x}=f(\xi)(b-a)$
		\end{enumerate}
	\end{satz}
	\begin{definition}
		Ist $I\subseteq\mathbb{R}$ ein Intervall und $G,g\colon I\to\mathbb{R}$, wobei
		$G$ differenzierbar auf $J$ und $G'=g$ auf $J$, dann ist $G$ auf $J$ eine
		Stammfunktion von $g$. Man schreibt auch $\int g\d{x}=G$.
	\end{definition}
	\begin{satz}[Partielle Integration]
		\begin{enumerate}[label=(\alph*)]
			\item Sind $f,g\in R[a,b]$ und $F$, $G$ Stammfunktionen von $f$, $g$
				auf $[a,b]$, dann
				\[ \int_a^bFg\d{x}=[F(x)G(x)]_a^b-\int_a^bfG\d{x} \]
			\item Sind $f,g\in C^1[a,b]$, dann
				\[ \int_a^bf'g\d{x}=[f(x)g(x)]_a^b-\int_a^bfg'\d{x} \]
			\item Sind $f,g\in C^1(I)$, dann gilt auf $I$
				\[ \int f'g\d{x}=f(x)g(x)-\int fg'\d{x} \]
		\end{enumerate}
	\end{satz}
	\begin{satz}[Integration durch Substitution]
		Es sei $f\in C(I)$, $g\in C^1(J)$, $g(J)\subseteq I$.
		\begin{enumerate}[label=(\alph*)]
			\item $J = [\alpha,\beta]\implies$
				\[ \int_\alpha^\beta f(g(t))g'(t)\d{t}=\int_{g(\alpha)}^{g(\beta)}f(t)\d{t} \]
			\item Auf $J$ gilt
				\[ \int f(g(t))g'(t)\d{t}=\int f(x)\d{x}|_{x=g(t)} \]
			\item $g~\text{streng monoton auf $J$}\implies\text{Auf $I$}$
				\[ \int f(x)\d{x}=\int f(g(t))g'(t)\d{t}|_{t=g^{-1}(x)} \]
		\end{enumerate}
	\end{satz}
	\newpage
	\sep
	\begin{multicols}{2}
		\begin{table}[H]
			\centering
			\begin{tabular}{ | M{2.5cm} | M{2.5cm} | N}
				\hline
				Funktion & Stammfunktion & \\[0.6cm] \hline \hline
				$nx^{n-1}$ & $x^n$ & \\[0.6cm] \hline
				$\frac{1}{x}$ & $\log\abs{x}$ & \\[0.6cm] \hline
				$\log x$ & $x\log x - x$ & \\[0.6cm] \hline
				$e^x$ & $e^x$ & \\[0.6cm] \hline
				$a^x\log a$ & $a^x$ & \\[0.6cm] \hline \hline
				$\sin x$ & $-\cos x$ & \\[0.6cm] \hline
				$\cos x$ & $\sin x$ & \\[0.6cm] \hline
				$\tan x$ & $-\log\abs{\cos x}$ & \\[0.6cm] \hline \hline
				$\frac{1}{\sqrt{1-x^2}}$ & $\arcsin x$ & \\[0.6cm] \hline
				$-\frac{1}{\sqrt{1-x^2}}$ & $\arccos x$ & \\[0.6cm] \hline
				$\frac{1}{1+x^2}$ & $\arctan x$ & \\[0.6cm] \hline \hline
				$\sinh x$ & $\cosh x$ & \\[0.6cm] \hline
				$\cosh x$ & $\sinh x$ & \\[0.6cm] \hline
				$\tanh x$ & $\log\cosh x$ & \\[0.6cm] \hline \hline
				$\frac{1}{\sqrt{x^2+1}}$ & $\arsinh x$ & \\[0.6cm] \hline
				$\frac{1}{\sqrt{x^2-1}}$ & $\arcosh x$ & \\[0.6cm] \hline
				$\frac{1}{1-x^2}$, $\abs{x}<1$ & $\artanh x$ & \\[0.6cm] \hline \hline
				$\frac{f'(x)}{1+f(x)^2}$ & $\arctan f(x)$ & \\[0.6cm] \hline
				$\frac{f'(x)}{f(x)}$ & $\log\abs{f(x)}$ & \\[0.6cm] \hline
				$f'(x)f(x)$ & $\frac{1}{2}f(x)^2$ & \\[0.6cm] \hline
				$f'(x)e^{f(x)}$ & $e^{f(x)}$ & \\[0.6cm] \hline
			\end{tabular}
			\caption{Liste von Stammfunktionen}
		\end{table}
		\begin{table}[H]
			\centering
			\begin{tabular}{ | M{3.5cm} | M{3.5cm} | N}
				\hline
				Reihe & Wert & \\[0.6cm] \hline \hline
				$\sum\limits_{n=0}^\infty x^n$, $\abs{x}<1$ & $\frac{1}{1-x}$ & \\[0.6cm] \hline
				$\sum\limits_{n=0}^\infty x^n$, $\abs{x}\geq1$ & $\infty$ & \\[0.6cm] \hline \hline
				$\sum\limits_{n=0}^\infty (n+1)x^n$, $\abs{x}<1$ & $\frac{1}{(1-x)^2}$ & \\[0.6cm] \hline
				$\sum\limits_{n=0}^\infty (n+1)x^n$, $\abs{x}\geq1$ & $\infty$ & \\[0.6cm] \hline \hline
				$\sum\limits_{n=1}^\infty\frac{1}{n^\alpha}$, $\alpha > 1$ & konvergent & \\[0.6cm] \hline
				$\sum\limits_{n=1}^\infty\frac{1}{n^\alpha}$, $\alpha \leq 1$ & $\infty$ & \\[0.6cm] \hline
			\end{tabular}
			\caption{Liste von Reihenwerten}
		\end{table}
	\end{multicols}
	\newpage
	\textsc{Analysis II}

	\sep
	\begin{satz}[Rechenregeln zur Norm]
		Es seien $x, y, z\in\mathbb{R}^n$, $\alpha, \beta\in\mathbb{R}$, $x=(x_1,\ldots,x_n), y=(y_1,\ldots,y_n)$, $A\in\mathbb{R}^{n\times n}$.
		\begin{enumerate}[label=(\alph*)]
			\item $\abs{x\cdot y} \leq \norm{x}\norm{y}$
			\item $\norm{x + y} \leq \norm{x} + \norm{y}$
			\item $\abs{\norm{x} - \norm{y}} \leq \norm{x - y}$
			\item $\forall 1\leq j\leq n\colon \abs{x_j}\leq\norm{x}\leq\sum_{i=0}^n\abs{x_i}$
			\item $\norm{Ax}\leq\norm{A}\norm{x}$
		\end{enumerate}
	\end{satz}
	\begin{definition}
		Es seien $x_0\in\mathbb{R}^n, \delta>0, A, U\subseteq\mathbb{R}^n$.
		\begin{enumerate}[label=(\alph*)]
			\item $U_\delta(x_0)\coloneqq\set{x\in\mathbb{R}^n\given\norm{x-x_0}<\delta}$ ist die offene Kugel
				um $x_0$ mit Radius $\delta$.
			\item $U$ ist Umgebung von $x_0$ $\defiff\exists\delta>0\colon U_\delta(x_0)\subseteq U$.
			\item $A$ ist beschränkt $\defiff\exists c>0\ \forall a\in A\colon \norm{a}\leq c$.
			\item $x_0\in A$ ist innerer Punkt von $A$ $\defiff\exists\delta>0\colon U_\delta(x_0)\subseteq A$.
			\item $A^\circ\coloneqq\set{x\in A\given x\ \text{ist innerer Punkt von}\ A}$.
			\item $A$ heißt offen $\defiff A=A^\circ$.
			\item $x_0$ heißt Häufungspunkt von A $\defiff\forall\delta>0\colon(U_\delta(x_0)\setminus\set{x_0})\cap A\neq\varnothing$.
			\item $\mathcal{H}(A)\coloneqq\set{x\in A\given x\ \text{ist Häufungspunkt von }\ A}$.
			\item $\bar{A}\coloneqq A\cup\mathcal{H}(A)$ ist der Abschluss von $A$.
			\item $A$ heißt abgeschlossen $\defiff A = \bar{A}$.
			\item $\partial A\coloneqq \bar{A}\setminus A^\circ$ ist der Rand von $A$.
		\end{enumerate}
	\end{definition}
	\begin{satz}
		Es sei $A\subseteq \mathbb{R}^n$, $\mathfrak{O}$ ein System offener Mengen,
		$\mathfrak{A}$ ein System abgeschlossener Mengen.
		\begin{enumerate}[label=(\alph*)]
			\item $A\ \text{ist abgeschlossen}\iff \mathbb{R}^n\setminus A\ \text{ist offen}$.
			\item $\bigcup_{O\in\mathfrak{O}}O\ \text{ist offen}$.
			\item $\mathfrak{O}\ \text{ist endlich}\implies\bigcap_{O\in\mathfrak{O}}O\ \text{ist offen}$.
			\item $\bigcap_{A\in\mathfrak{A}}A\ \text{ist abgeschlossen}$.
			\item $\mathfrak{A}\ \text{ist endlich}\implies\bigcup_{A\in\mathfrak{A}}A\ \text{ist abgeschlossen}$.
		\end{enumerate}
	\end{satz}
	\sep
	\begin{definition}
		Es sei $(a^{(k)})$ eine Folge in $\mathbb{R}^n$.
		\begin{enumerate}[label=(\alph*)]
			\item $(a^{(k)})\ \text{heißt beschränkt}\defiff\exists c\geq 0\ \forall k\in\mathbb{N}\colon \norm{a^{(k)}}\leq c$.
			\item $(a^{(k)})\ \text{heißt konvergent}\defiff\exists a\in\mathbb{R}^n\colon \norm{a^{(k)}-a}\to 0\ (k\to\infty)$.
				In diesem Fall $\lim_{k\to\infty} a^{(k)}\coloneqq a$.
		\end{enumerate}
	\end{definition}
	\begin{satz}
		Es sei $(a^{(k)})$ eine Folge in $\mathbb{R}^n$, $A\subseteq\mathbb{R}^n$
		\begin{enumerate}[label=(\alph*)]
			\item $a^{(k)}\to a\ (k\to\infty)\iff a_1^{(k)}\to a_1,\ldots,a_n^{(k)}\to a_n\ (k\to\infty)$.
			\item $a^{(k)}\to a\ (k\to\infty)\implies (a^{(k)})$ ist beschränkt und jede Teilfolge und jede Umordnung
				von $(a^{(k)})$ konvergiert gegen $a$.
			\item Ist $(a^{(k)})$ beschränkt, so enthält $(a^{(k)})$ eine konvergente Teilfolge.
			\item $(a^{(k)})\ \text{konvergent}\iff\forall\varepsilon>0\ \exists k_0\in\mathbb{N}\ \forall k, l\geq k_0\colon
				\norm{a^{(k)} - a^{(l)}}<\varepsilon$.
			\item $x_0\in\mathcal{H}(A) \iff\exists\ \text{Folge}\ (y^{(k)})\ \text{in}\ A\setminus\set{x_0}\colon y^{(k)}\to x_0$.
			\item $x_0\in\bar{A}\iff\exists\ \text{Folge}\ (y^{(k)})\ \text{in}\ A\colon y^{(k)}\to x_0$.
			\item $A\ \text{ist abgeschlossen} \iff \text{der Grenzwert jeder konvergenten Folge in $A$ gehört zu $A$}$.
			\item Es sind äquivalent:
				\begin{enumerate}[label=(\roman*)]
					\item $A$ ist beschränkt und abgeschlossen.
					\item Jede Folge in $A$ enthält eine konvergente Teilfolge, deren Grenzwert zu $A$ gehört.
					\item $A$ ist kompakt.
				\end{enumerate}
		\end{enumerate}
	\end{satz}
	\sep
	\begin{definition}[Funktionsgrenzwert]
		Es sei $\varnothing\neq D\subseteq\mathbb{R}^n$, $f\colon D\to\mathbb{R}^m$, $x_0\in D$, $y_0\in\mathbb{R}^m$.
		\[\lim_{x\to x_0}f(x) = y_0\defiff\ \text{für jede Folge $(x^{(k)})$ in $D\setminus\set{x_0}$ mit $x^{(k)}\to x_0$}\colon
		f(x^{(k)})\to y_0\]
	\end{definition}
	\begin{definition}[Stetigkeit]
		Es sei $\varnothing\neq D\subseteq\mathbb{R}^n$, $f\colon D\to\mathbb{R}^m$, $x_0\in D$.
		\begin{enumerate}[label=(\alph*)]
			\item
				$f\ \text{stetig in $x_0$}\defiff\ \text{für jede Folge $(x^{(k)})$ in $D$ mit $x^{(k)}\to x_0$}\colon f(x^{(k)})\to f(x_0)$
				Im Fall $x_0\in D\cap\mathcal{H}(D)$ gilt $f\ \text{stetig in $x_0$}\iff \lim_{x\to x_0} f(x) = f(x_0)$.
			\item $f\in C(D, \mathbb{R}^n)\defiff\forall x\in D\colon f\ \text{stetig in $x$}$.
		\end{enumerate}
	\end{definition}
	\sep

		Differenzierbarkeit, Extremwerte, Umkehrsatz/implizite Funktionen, Extremwerte
		unter Nebenbedingungen, Wege und Wegintegrale, Stammfunktionen und der
		Banachsche Fixpunktsatz seien dem geneigten Leser als leichte Übung überlassen.

	\sep
	\begin{defsatz}[Lineare Differentialgleichung erster Ordnung]
		Es sei $I\subseteq\mathbb{R}$ ein Intervall, $a, s\colon I\to\mathbb{R}$ stetig.
		Es sei $A$ eine Stammfunktion von $a$. Betrachte die Gleichung $y' = a(x)y + s(x)$
		\begin{enumerate}[label=(\alph*)]
			\item Ist $s\equiv 0$, so haben alle Lösungen die Form $y(x) = ce^{A(x)}$ mit $c\in\mathbb{R}$.
				Das zugehörige Anfangswertproblem hat genau eine Lösung auf I.
			\item Für beliebiges $s$ sei $c$ eine Stammfunktion von $se^{-A}$.
				$y_s(x) = c(x)e^{A(x)}$ löst dann die inhomogene Gleichung. Alle
				Lösungen der inhomogenen Gleichung sind die Summe einer Lösung der homogenen
				Gleichung und der speziellen Lösung $y_s$. Das zugehörige Anfangswertproblem
				hat genau eine Lösung auf $I$.
		\end{enumerate}
	\end{defsatz}
	\begin{defsatz}[Differentialgleichungen mit getrennten Veränderlichen]
		Es seien $I, J\subseteq\mathbb{R}$ Intervalle, $f\in C(I)$, $g\in C(J)$ mit $g(y)\neq 0\forall y\in J$.
		Betrachte die Gleichung $y' = f(x)g(y)$.
		\begin{enumerate}[label=(\alph*)]
			\item Trennung der Veränderlichen: Löst man die Gleichung
				$\int \frac{1}{g(y)}\dd{y} = \int f(x)\dd{x}+c$ symbolisch nach $y$ auf erhält man einen Term für
				die allgemeine Lösung der Gleichung.
			\item Alternativ lässt sich das Anfangswertproblem auch durch Auflösen der Gleichung
				$\int_{y_0}^{y(x)}\frac{1}{g(t)}\dd{t} = \int_{x_0}^x f(t)\dd{t}$ lösen.
		\end{enumerate}
	\end{defsatz}
	\begin{definition}[System von Differentialgleichungen 1. Ordnung]
		Ist $D\subseteq\mathbb{R}^{n+1}$, $f\colon D\to\mathbb{R}^n$ und
		$(x, y)\coloneqq (x, y_1, \ldots, y_n)$, so heißt $y' = f(x, y)$ ein System
		von Differentialgleichungen 1. Ordnung.
	\end{definition}
	\begin{definition}[Lipschitzbedingungen]
		Es sei $D\subseteq\mathbb{R}^{n+1}$, $g\colon D\to\mathbb{R}^n$.
		\begin{enumerate}[label=(\alph*)]
			\item $g$ genügt auf $D$ einer Lipschitz-Bedingung bezüglich $y$,
				genau dann wenn:
				\[
					\exists L\geq 0\ \forall (x, y), (x, \bar{y})\in D
					\colon \norm{g(x, y) - g(x, \bar{y})} \leq L\norm{y - \bar{y}}
				\]
			\item $g$ genügt auf $D$ einer lokalen Lipschitz-Bedingung bezüglich $y$, genau
				dann wenn für $a\in D$ eine Umgebung $U$ von $a$ existiert, sodass $g_{\vert_{D\cap U}}$
				auf $D\cap U$ einer Lipschitzbedingung bezüglich $y$ genügt.
		\end{enumerate}
	\end{definition}
	\begin{satz}[Kriterium für lokale LB]
		Ist $D\subseteq\mathbb{R}^{n+1}$ offen, $f\colon D\to\mathbb{R}^n$ und für
		$j, k\in\set{1,\ldots,n}$ sei $\pdv{f_j}{y_k}$ vorhanden und stetig. Dann genügt
		$f$ auf $D$ einer lokalen LB bezüglich $y$.
	\end{satz}
	\begin{satz}[Existenz- und Eindeutigkeitssatz von Picard-Lindelöf, Version I]
		Es sei $I = \left[a, b\right]$, $x_0\in I$, $y_0\in\mathbb{R}^n$, $D\coloneqq I\cross\mathbb{R}^n$,
		$f\in C(D,\mathbb{R}^n)$ und $f$ genüge auf $D$ eine LB bezüglich $y$. Dann ist das Anfangswertproblem
		$y'=f(x, y)$, $y(x_0)=y_0$ auf $I$ eindeutig lösbar.
	\end{satz}
	\begin{satz}[Existenz- und Eindeutigkeitssatz von Picard-Lindelöf, Version III]
		Es sei $D\subseteq\mathbb{R}^{n+1}$ offen, $(x_0, y_0)\in D$, $f\in C(D, \mathbb{R}^n)$
		und $f$ genüge auf $D$ einer lokalen LB bezüglich $y$. Dann hat das Anfangswertproblem
		$y'=f(x, y)$, $y(x_0) = y_0$ eine eindeutige, nicht fortsetzbare Lösung.
	\end{satz}
	\begin{defsatz}[Lineare Systeme]
		Es sei $I\subseteq\mathbb{R}$ ein Intervall, $x_0\in I$, $y_0\in\mathbb{R}^n$,
		$D\coloneqq I\cross\mathbb{R}^n$, $b\colon I\to\mathbb{R}^n$ stetig, $A\colon I\to\mathbb{R}^{n\times n}$
		eintragsweise stetig. Betrachte das lineare System $y'= A(x)y + b(x)$.
		\begin{enumerate}[label=(\alph*)]
			\item Das Anfangswertproblem hat auf $I$ genau eine Lösung.
			\item Das lineare System hat Lösungen auf $I$
			\item Ist $y_s$ eine spezielle Lösung, so setzt sich jede Lösung des linearen
				Systems aus $y_s$ und eine Lösung des homogenen Systems zusammen.
			\item Die Lösungen $\mathbb{L}$ des homogenen Systems bilden einen rellen Vektorraum der Dimension $n$.
			\item Es seien $y^{(1)},\ldots,y^{(k)}\in\mathbb{L}$. Es sind äquivalent:
				\begin{enumerate}[label=(\roman*)]
					\item $\set{y^{(1)},\ldots,y^{(k)}}$ ist linear unabhängig in $\mathbb{L}$.
					\item Für alle $x\in I$ ist $\set{y^{(1)}(x),\ldots,y^{(k)}(x)}$ linear unabhängig in $\mathbb{R}^n$.
					\item Für ein $\xi\in I$ ist $\set{y^{(1)}(\xi),\ldots,y^{(k)}(\xi)}$ linear unabhängig in $\mathbb{R}^n$.
				\end{enumerate}
		\end{enumerate}
		Sind $y^{(1)},\ldots,y^{(n)}\in\mathbb{L}$, so ist $y^{(1)},\ldots,y^{(n)}$ ein Lösungssystem,
		$Y(x)\coloneqq(y^{(1)}(x),\dots,y^{(n)}(x))$ eine Lösungsmatrix, $W(x)\coloneqq\det Y(x)$ deren
		Wronskideterminante. Ist $Y(\xi)$ invertierbar für ein $\xi\in I$, so heißt
		das Lösungssystem ein Fundamentalsystem.

		Ist $Y'(x)$ die komponentenweise Ableitung eines Lösungssystems, so ist
		$\forall x\in I\colon Y'(x)=A(x)Y(x)$.

		Ist $Y(x)$ ein Fundamentalsystem, so ist $y_s(x)\coloneqq Y(x)\int Y(x)^{-1}b(x)\dd{x}$
		eine spezielle Lösung auf $I$.
	\end{defsatz}
	\begin{verfahren}[Homogene lineare Systeme mit konstanten Koeffizienten]
		Es sei $A$ wie oben, aber komponentenweise konstant.
		Bestimme $M\coloneqq\set{\lambda\in\Spek A\given \Im(\lambda)\geq 0}$.
		Für $\lambda\in M$ mit algebraischer Vielfachheit $r$:
		\begin{enumerate}
			\item Berechne $\Kern(A-\lambda E_n)$, $\Kern (A-\lambda E_n)^2$, \ldots, $H_\lambda$.
			\item Es sei $v$ Basisvektor von $H_\lambda$.
				\begin{equation*}
					y(x)\coloneqq e^{\lambda x}(v + \frac{x}{1!}(A-\lambda E_n)v + \frac{x^2}{2!}(A-\lambda E_n)^2v
						+\cdots+\frac{x^{r - 1}}{(r - 1)!}(A-\lambda E_n)^{r-1}v)
				\end{equation*}
				\begin{enumerate}[label=(\alph*)]
					\item Falls $\lambda\in\mathbb{R}$, so ist $y$ eine Lösung.
					\item Falls $\lambda\in\mathbb{C}\setminus\mathbb{R}$,
						zerlege $y(x)=y^{(1)}(x) + iy^{(2)}(x)$, $y^{(1)}(x), y^{(2)}(x)\in\mathbb{R}^n$.
						$y^{(1)}, y^{(2)}$ sind dann linear unabhängige Lösungen.
				\end{enumerate}
		\end{enumerate}
		Alle so bestimmten Lösungen ergeben zusammen ein Fundamentalsystem.
	\end{verfahren}
	\newpage
	\textsc{Analysis III}

	\sep
	\begin{definition}[\sa{}]
		Es sei $\varnothing\neq X$, $\A\subseteq\P(X)$. $\A$ heißt \sa{} auf $A$, wenn
		\begin{enumerate}[label=($\sigma_{\arabic*}$)]
			\item $X\in\A$
			\item $A\in\A\implies\compl{A}\in\A$
			\item Ist $(A_j)$ eine Folge in $\A$, so ist $\bigcup A_j\in\A$.
		\end{enumerate}
	\end{definition}
	\begin{defsatz}[Erzeugte \sa{}]
		\begin{enumerate}[label=(\alph*)]
			\item Für $\varnothing\neq\mathcal{E}\subseteq\P(X)$ ist
				\[
					\sigma(\mathcal{E})\coloneqq \bigcap_{\text{$\mathcal{F}$ \sa{} über $X$}}\mathcal{F}
				\]
				eine \sa{} und heißt die von $\mathcal{E}$ erzeugte \sa{}.
			\item Ist $\mathcal{E}$ eine \sa{}, so ist $\mathcal{E}=\sigma(\mathcal{E})$.
			\item $\mathcal{E}\subseteq\mathcal{E}'\implies \sigma(\mathcal{E})\subseteq\sigma(\mathcal{E}')$.
		\end{enumerate}
	\end{defsatz}
	\begin{defsatz}[Borelsche \sa{}]
		Für $X\subseteq\R^d$ heißt $\B(X)\coloneqq\sigma(\mathcal{O}(X))$ die Borelsche \sa{} auf $X$. Sie wird
		erzeugt von den Intervallen mit Grenzen in $\mathbb{Q}^d$, den Halbräumen, \dots.
	\end{defsatz}
	\begin{defsatz}[Spur]
		Es sei $\varnothing\neq Y\subseteq X$, $\A$ eine \sa{} auf $X$.
		\begin{enumerate}[label=(\alph*)]
			\item $\A_Y\coloneqq\set{A\cap Y\given A\in \A}$ heißt die Spur von $\A$ in $Y$ und ist eine \sa{} auf $Y$.
			\item $\A_Y\subseteq\A\iff Y\in\A$.
			\item Für $\varnothing\neq\mathcal{E}\subseteq\P(X)$ ist $\sigma(\mathcal{E}_Y)=\sigma(\mathcal{E})_Y$.
			\item $\B(X) = (\B_d)_X = \set{A\cap X\given A\in \B_d}$.
			\item $X\in\B_d\implies \B(X)\subseteq \B_d$.
		\end{enumerate}
	\end{defsatz}
	\begin{definition}[Maß]
		Es sei $\A$ eine \sa{} auf $X$, $\mu\colon\A\to[0,\infty]$. $\mu$ heißt Maß auf $\A$, wenn
		\begin{enumerate}[label=($M_{\arabic*}$)]
			\item $\mu(\varnothing)=0$.
			\item $\mu(\sum A_j) = \sum\mu(A_j)$.
		\end{enumerate}
	\end{definition}
	\begin{satz}[Eigenschaften von Maßen]
		$(X,\A,\mu)$ Maßraum, $A, B\in\A$, $(A_j)$ Folge in $\A$. Dann:
		\begin{enumerate}[label=(\alph*)]
			\item $A\subseteq B\implies \mu(A)\leq\mu(B)$.
			\item $\mu(A)<\infty, A\subseteq B\implies \mu(B\setminus A)=\mu(B)-\mu(A)$.
			\item $\mu(X)<\infty\implies \mu(A)<\infty, \mu(\compl{A})=\mu(X)-\mu(A)$.
			\item $\mu(\bigcup A_j)\leq\sum\mu(A_j)$.
			\item $A_1\subseteq A_2\subseteq\cdots\implies \mu(\bigcup A_j)=\lim_{n\to\infty}\mu(A_n)$.
			\item $A_1\supseteq A_2\supseteq\cdots\wedge\mu(A)<\infty\implies \mu(\bigcap A_j)=\lim_{n\to\infty}\mu(A_n)$.
		\end{enumerate}
	\end{satz}
	\begin{definition}[Ring, Prämaß]
		\begin{enumerate}[label=(\alph*)]
			\item $\varnothing\neq\Ri\subseteq\P(X)$ heißt Ring, wenn:
			\begin{enumerate}[label=($R_{\arabic*}$)]
				\item $\varnothing\in\Ri$.
				\item $A, B\in \Ri\implies A\cup B, B\setminus A\in\Ri$.
			\end{enumerate}
			\item $\mu\colon\Ri\to[0, \infty]$ heißt Prämaß auf $\Ri$, wenn:
			\begin{enumerate}[label=($M_{\arabic*}$)]
				\item $\mu(\varnothing) = 0$.
				\item $\sum A_j\in\Ri\implies \mu(\sum A_j)=\sum\mu(A_j)$.
			\end{enumerate}
		\end{enumerate}
	\end{definition}
	\begin{satz}[Elementarvolumen + Magie = Lebesgue-Maß]
		\begin{enumerate}[label=(\alph*)]
			\item Es sei $\Ri$ ein Ring auf $X$, $\mu\colon\Ri\to[0, \infty]$ ein Prämaß. Dann
				existiert ein Maßraum $(X, \A(\mu), \overline{\mu})$, sodass
				\begin{enumerate}[label=(\arabic*)]
					\item $\sigma(\Ri)\subseteq\A(\mu)$.
					\item $\forall A\in\Ri:\overline{\mu}(A) = \mu(A)$.
				\end{enumerate}
			\item Es sei $\mathcal{E}\subseteq\P(X)$ durchschnittsstabil,
				$\nu, \mu$ Maße auf $\sigma(\mathcal{E})$,
				$\forall E\in\mathcal{E}: \mu(E)=\nu(E)$ und es existiere $(E_n)$ in $\mathcal{E}$
				mit $\bigcup E_n=X$ und $\forall n\in\mathbb{N}:\mu(E_n)<\infty$. Dann $\mu=\nu$ auf $\sigma(\mathcal{E})$.
			\item Es existiert genau ein Maß $\lambda_d$ auf $\B_d$ mit $\lambda_d((a, b]) = \prod (b_i - a_i)$.
		\end{enumerate}
	\end{satz}
	\begin{satz}
		Ist $\mu\colon\B_d\to[0,\infty]$ ein Maß, $\forall x\in\R^d, A\in\B_d: \mu(A)=\mu(x+A)$, $c\coloneqq\mu((0, 1]^d)<\infty$,
		so gilt $\mu=c\lambda_d$.
	\end{satz}
	\sep
	\begin{definition}[Messbarkeit]
		$f\colon X\to Y$ heißt $\A$-$\B$-messbar, wenn $\forall B\in\B:f^{-1}(B)\in\A$.
	\end{definition}
	\begin{satz}
		\begin{enumerate}[label=(\alph*)]
			\item Es sei $\sigma(\mathcal{E})=\B$. Dann ist $f$ $\A$-$\B$-messbar $\iff\forall E\in\mathcal{E}:f^{-1}(E)\in\B$.
			\item \ldots
		\end{enumerate}
	\end{satz}
	\begin{definition}[Limites]
		\begin{enumerate}[label=(\alph*)]
			\item $\sup_{n\in\mathbb{N}} f_n$, $\inf_{n\in\mathbb{N}} f_n$ sind punktweise definiert.
			\item $\limsup_{n\to\infty} f_n\coloneqq \inf_{j\in\mathbb{N}}\left(\sup_{n\geq j}f_n\right)$.
			\item $\liminf_{n\to\infty} f_n\coloneqq \sup_{j\in\mathbb{N}}\left(\inf_{n\geq j}f_n\right)$.
		\end{enumerate}
	\end{definition}
	\begin{satz}
		Es sei jedes $f_n$ messbar.
		\begin{enumerate}[label=(\alph*)]
			\item $\sup_{n\in\mathbb{N}} f_n$, $\inf_{n\in\mathbb{N}} f_n$, $\limsup_{n\to\infty} f_n$, $\liminf_{n\to\infty} f_n$ sind messbar.
			\item Gilt $f_n\to f$ pktw in $\overline{\R}$, so ist $f$ messbar.
		\end{enumerate}
	\end{satz}
	\sep
	\begin{definition}[Lebesgue-Integral]
		\begin{enumerate}[label=(\alph*)]
			\item $f\colon X\to\R$ messbar heißt einfach, wenn $f=\sum_{j=1}^my_j\mathds{1}_{A_j}$ mit $y_i\in\R$, $A_i\in\B(X)$.
			\item Für $f\colon X\to[0,\infty)$ einfach: $\int_Xf\dd{x}\coloneqq\sum_{j=1}^my_j\lambda(A_j)$.
			\item $f_n$ heißt zulässig für $f\defiff f_n~\text{einfach}, 0\leq f_n\leq f_{n+1}, f_n\to f~\text{pktw}$.
			\item Sei $f_n$ zulässig für $f\colon X\to [0,\infty]$. $\int_X f\dd{x}\coloneqq \lim_{n\to\infty}\int_Xf_n\dd{x}$.
			\item $f\colon X\to\overline{\R}$ heißt integrierbar, falls $\int_X f_+\dd{x}<\infty$, $\int_X f_-\dd{x}<\infty$.
				Dann $\int_X f\dd{x}\coloneqq \int_X f_+\dd{x} - \int_X f_-\dd{x}$.
		\end{enumerate}
	\end{definition}
	\begin{satz}[Integrierbarkeit]
		\begin{enumerate}[label=(\alph*)]
			\item $f$ integrierbar $\iff$ $\abs{f}$ integrierbar.
			\item $f$ integrierbar $\iff\exists g\colon X\to[0,\infty], \abs{f}\leq g$, $g$ integriarbar.
		\end{enumerate}
	\end{satz}
	\sep
	\begin{satz}[Beppo Levi I]
		Es sei $\forall n\in\mathbb{N}: f_n\colon X\to[0,\infty]~\text{mb}, f_n\leq f_{n+1}$.
		Dann $\int_X\lim_{n\to\infty}f_n\dd{x} = \lim_{n\to\infty}\int_Xf_n\dd{x}$.
	\end{satz}
	\begin{satz}[Beppo Levi II]
		Es sei $\forall n\in\mathbb{N}: f_n\colon X\to[0,\infty]~\text{mb}$.
		Dann $\int_X\sum_{j\in\mathbb{N}}f_j\dd{x} = \sum_{j\in\mathbb{N}}\int_Xf_j\dd{x}$.
	\end{satz}
	\begin{satz}[Beppo Levi III]
		Es sei $\forall n\in\mathbb{N}: f_n\colon X\to[0,\infty]~\text{mb}, f_n\leq f_{n+1}~\text{f.ü}$.
		Dann existiert $f\colon X\to[0,\infty]$ mit $f_n\to f~\text{f.ü.}$ und $\int_Xf\dd{x}=\lim_{n\to\infty}\int_Xf_n\dd{x}$.
	\end{satz}
	\begin{satz}[Lemma von Fatou]
		$f_n\colon X\to[0,\infty]$ messbar.
		\begin{enumerate}[label=(\alph*)]
			\item $\int_X\liminf_{n\to\infty}f_n\dd{x}\leq\liminf_{n\to\infty}\int_Xf_n\dd{x}$.
			\item $f\colon X\to[0,\infty]$ messbar, $f_n\to f$ fast überall: $\int_Xf\dd{x}\leq\liminf_{n\to\infty}\int_Xf_n\dd{x}$.
			\item $(\int_Xf_n\dd{x})$ beschränkt: $f$ integrierbar.
		\end{enumerate}
	\end{satz}
	\begin{satz}[Konvergenzsatz von Lebesgue]
		$f_n\colon X\to\overline{\R}$ messbar, $(f_n)$ fast überall konvergent,
		$\abs{f_n}\leq g$ für ein $g\colon X\to[0,\infty]$ integrerbar. Dann existiert $f$ integrierbar mit:
		\begin{enumerate}[label=(\alph*)]
			\item $f_n\to f$ fast überall.
			\item $\int_Xf_n\dd{x}\to\int_Xf\dd{x}$.
			\item $\int_X\abs{f_n-f}\dd{x}\to0\en$.
		\end{enumerate}
	\end{satz}
	\sep
	\begin{satz}[Parameterintegrale]
		$U\in\B_k$, $t_0\in U$, $f\colon U\cross X\to\R$. Weiter:
		\begin{enumerate}[label=(\alph*)]
			\item $\forall t\in U: x\mapsto f(t, x)~\text{messbar}$.
			\item Für fast alle $x\in X$ ist $t\mapsto f(t, x)$ stetig in $t_0$.
			\item $g\colon X\to [0,\infty]$ integrierbar mit $\forall t\in U$:
				Für fast alle $x\in X: \abs{f(t, x)}\leq g(x)$.
		\end{enumerate}
		Dann ist $F\colon U\to\R$, $t\mapsto\int_Xf(t, x)\dd{x}$ stetig in $t_0$,
		also $\lim_{t\to t_0}\int_Xf(t, x)\dd{x} = \int_X\lim_{t\to t_0}f(t, x)\dd{x}$.
	\end{satz}
	\begin{satz}[Lebesgue-Integral und partielle Ableitungen]
		$U\subseteq\R^k$ offen, $f\colon U\cross X\to\R$, $g\colon X\to[0,\infty]$ integrierbar,
		$N\subseteq X$ Nullmenge. Weiter:
		\begin{enumerate}[label=(\alph*)]
			\item $\forall t\in U: x\mapsto f(t, x)$ integrierbar.
			\item $\forall x\in X\setminus N: t\mapsto f(t, x)$ partiell differenzierbar auf $U$.
			\item $\forall x\in X\setminus N, t\in U, j\in\set{1,\ldots,k}: \abs{\pdv{f}{t_j}}\leq g(x)$.
		\end{enumerate}
		Definiere $F\colon U\to\R$, $t\mapsto \int_Xf(t, x)\dd{x}$. Dann ist $F$ auf $U$ partiell
		differenzierbar und
		\[
			\forall t\in U, j\in\set{1,\ldots,k}:{\pdv{F}{t_j}}(t)=\int_X{\pdv{f}{t_j}}(t, x)\dd{x}.
		\]
		Also:
		\[
			\pdv{t_j}\int_Xf(t, x)\dd{x} = \int_X{\pdv{f}{t_j}}(t, x)\dd{x}.
		\]
	\end{satz}
	\sep
	\begin{satz}[Prinzip von Cavalieri]
		Sei $C\in\B_d$. Dann:
		\[
			\int_{\R^d}\mathds{1}_C(x, y)\dd{(x, y)}=\int_{\R^k}\left(\int_{\R^l}\mathds{1}_C(x, y)\dd{y}\right)\dd{x}.
		\]
	\end{satz}
	\begin{satz}[Sätze von Tonelli und Fubini]
		Sei $\varnothing\neq X\in\B_k$, $\varnothing\neq Y\in\B_l$, $D\coloneqq X\cross Y$, $f\colon D\to\overline{\R}$ messbar.
		Weiter sei entweder $f\geq 0$ oder $f$ integrierbar. Dann:
		\[
			\int_Df(x, y)\dd{(x, y)} = \int_X\left(\int_Yf(x, y)\dd{y}\right)\dd{x}.
		\]
	\end{satz}
	\begin{definition}[Diffeomorphismus]
		$X, Y\subseteq\R^d$. $\Phi\in C^1(X,\R^d)$ heißt Diffeomorphismus, wenn $\Phi$ bijektiv
		und $\Phi^{-1}\in C^1(Y,\R^d)$.
	\end{definition}
	\begin{satz}[Transformationssatz I]
		Es sei $\Phi\colon X\to Y$ ein Diffeomorphismus, $f\colon Y\to\overline{\R}$.
		Weiter sei $f\geq 0$ oder $f$ integrierbar. Dann:
		\[
			\int_Yf\dd{y} = \int_Xf(\Phi(x))\abs{\det\Phi'(x)}\dd{x}.
		\]
	\end{satz}
	\begin{satz}[Transformationssatz II]
		Sei $\varnothing\neq U\subseteq\R^d$ offen, $\Phi\in C^1(U, \R^d)$, $A\subseteq U$,
		$A\in\B_d$, $X\coloneqq A^\circ$, $A\setminus A^\circ$ eine Nullmenge, $\Phi$ injektiv auf $X$,
		$\forall x\in X:\det\Phi'\neq 0$, $B\coloneqq\Phi(A)\in\B_d$. Dann:
		\begin{enumerate}[label=(\alph*)]
			\item $Y\coloneqq \Phi(X)$ ist offen und $\Phi\colon X\to Y$ ist ein Diffeomorphismus.
			\item Ist $f\colon B\to\overline{\R}$ messbar und $f\geq 0$ oder $f$ integrierbar, so gilt
			\[
				\int_Bf\dd{y}=\int_Af(\Phi(x))\abs{\det\Phi'(x)}\dd{x}.
			\]
		\end{enumerate}
	\end{satz}
	\begin{verfahren}[Typische Transformationen]
		\begin{enumerate}[label=(\alph*)]
			\item Polar: $(r,\varphi)\in [0,\infty)\cross[0,2\pi]$, $\Phi(r, \varphi)\coloneqq (r\cos\varphi, r\sin\varphi)$, $\det\Phi'(r,\phi)=r$.
			\item Zylinder: $(r, \varphi, z)\in [0,\infty)\cross [0,2\pi]\cross\R$, $\Phi(r, \varphi, z)\coloneqq(r\cos\varphi, r\sin\varphi, z)$,
				$\det\Phi'(r, \varphi, z)=r$.
			\item Kugel: $(r, \varphi,\theta)\in [0,\infty)\cross[0,2\pi]\cross[0,\pi]$,
				$\Phi(r, \varphi,\theta)\coloneqq(r\cos\varphi\sin\theta, r\sin\varphi\sin\theta, r\cos\theta)$,
				$\det\Phi'(r, \varphi, \theta)=-r^2\sin\theta$.
		\end{enumerate}
	\end{verfahren}
	\sep
	\begin{definition}
		\begin{enumerate}[label=(\alph*)]
			\item $a=(a_1, a_2, a_3)\in\R^3$, $b=(b_1, b_2, b_3)\in\R^3$. $a\cross b\coloneqq (a_2b_3-a_3b_2, a_3b_1 - a_1b_3, a_1b_2 - a_2b_1)$.
			\item $D\subseteq \R^n$ offen, $f\in C^1(D,\R^n)$. $\ddiv f\coloneqq \pdv{f_1}{x_1} + \ldots + \pdv{f_n}{x_n}$.
			\item $D\subseteq \R^3$ offen, $F=(P, Q, R)\in C^1(D, \R^3)$. $\rot F\coloneqq (R_y-Q_z, P_z-R_x,Q_x-P_y)$.
		\end{enumerate}
	\end{definition}
	\begin{satz}[von Gauß]\label{gauss}
		Es sei $(x_0, y_0)\in\R^2$, $R\colon[0, 2\pi]\to[0,\infty)$ stetig, stückweise stetig differenzierbar, $R(0) = R(2\pi)$. Definiere
		\begin{align*}
			\forall t\in[0,2\pi]: \gamma(t)&\coloneqq (x_0 + R(t)\cos t, y_0+R(t)\sin t)\\
			B&\coloneqq\set{(x_0+r\cos t, y_0+r\sin t)\given t\in[0, 2\pi], r\in[0, R(t)]}.
		\end{align*}
		Weiter sei $D\subseteq\R^2$ offen, $B\subseteq D$, $f=(u, v)\in C^1(D,\R^2)$. Dann
		\[
			\int_B\ddiv f\dd{(x, y)} = \int_\gamma(u\dd{y} - v\dd{x})
				= \int_0^{2\pi} u(\gamma(t))\gamma_2'(t)\dd{t} - \int_0^{2\pi} v(\gamma(t))\gamma_1'(t)\dd{t}.
		\]
	\end{satz}
	\begin{verfahren}[Wegintegrale zweiter Art]
			Seien $\gamma, B$ wie in \ref{gauss}, $f=(u, v)\colon\Gamma_{\gamma}\to\R^2$. Definiere
			$\tilde{f}\coloneqq (v, -u)$. Dann gilt:
			\[
				\int_\gamma f\cdot\dd{(x, y)} = \int_\gamma v\dd{y} + u\dd{x} = \int_B \ddiv\tilde{f}\dd{(x, y)}.
			\]
	\end{verfahren}
	\begin{definition}[Flächen]
		$\varnothing\neq B\subseteq \R^2$ kompakt, $D\subseteq\R^2$ offen, $B\subseteq D$,
		$\varphi\in C^1(D,\R^3)$.
		\begin{enumerate}[label=(\alph*)]
			\item $\restr{\varphi}{B}$ heißt Fläche, $S\coloneqq\varphi(B)$ heißt Flächenstück.
			\item $N(u_0, v_0)\coloneqq \varphi_u(u_0, v_0)\cross\varphi_v(u_0, v_0)$ heißt Normalenvektor.
			\item Ist $f\in C^1(D,\R)$, $\varphi(u, v) = (u, v, f(u, v))$, so ist $\varphi$ in expliziter Darstellung. $N(u, v) = (-f_u, -f_v, 1)$.
			\item $I(\varphi)\coloneqq\int_B\norm{N(u, v)}\dd{(u, v)}$ heißt Flächeninhalt von $\varphi$.
			\item Für $f\in C(S,\R)$: $\int_\varphi f\dd{\sigma}\coloneqq\int_B f(\varphi(u, v))\norm{N(u, v)}\dd{(u, v)}$.
			\item Für $f\in C(S,\R^3)$: $\int_\varphi F\cdot n\dd{\sigma}\coloneqq\int_BF(\varphi(u, v))\cdot N(u, v)\dd{(u, v)}$.
		\end{enumerate}
	\end{definition}
	\begin{satz}[von Stokes]
		Seien $\gamma, B$ wie in \ref{gauss}, $\varphi\in C^2(D,\R^3)$, $G\subseteq\R^3$ offen, $S\subseteq G$, $F\in C^1(G,\R^3)$. Dann:
		\[
			\int_\varphi \rot F\cdot n\dd{\sigma} = \int_{\varphi\circ\gamma}F\cdot\dd{(x, y, z)}.
		\]
	\end{satz}
	\sep
	\begin{definition}
		\begin{enumerate}[label=(\alph*)]
			\item Sei $p\in[1,\infty)$. $\L{p}\coloneqq\set{f\colon X\to\R\given f~\text{messbar}, \int_X\abs{f}^p\dd{x}<\infty}$.
			\item Sei $f\in\L{p}$. $\norm{f}_p\coloneqq (\int_X\abs{f}^p\dd{x})^\frac{1}{p}$.
			\item $\L{\infty}\coloneqq\set{f\colon X\to\R\given f~\text{messbar und fast überall beschränkt}}$.
			\item Sei $f\in\L{\infty}$. $\norm{f}_\infty\coloneqq \esssup_{x\in X}\norm{f(x)}
				= \inf\set{c>0\given\exists\text{NM}~N_c\subseteq X~\forall x\in X\setminus N_c:\abs{f(x)}\leq c}$.
		\end{enumerate}
	\end{definition}
	\begin{satz}[Ungleichungen]
		Es seien $p, p', q, r\in[1,\infty]$, sodass $\frac{1}{p}+\frac{1}{p'}=1$ und $\frac{1}{r}=\frac{1}{p}+\frac{1}{q}$.
		\begin{enumerate}[label=(\alph*)]
			\item Für $f\in\L{p}$, $g\in\L{p'}$: $fg\in\L{1}$ und $\norm{fg}_1\leq\norm{f}_p\norm{g}_{p'}$.
			\item Für $f\in\L{p}$, $g\in\L{q}$: $fg\in\L{r}$ und $\norm{fg}_r\leq\norm{f}_p\norm{g}_q$.
			\item Für $f,g\in\L{p}$: $\norm{f+g}_p\leq\norm{f}_p+\norm{g}_p$.
		\end{enumerate}
	\end{satz}
	\begin{satz}
		Sei $\lambda_d(X)<\infty$, $p, q\geq 1$, $p\leq q\leq\infty$. Dann $\L{q}\subseteq\L{p}$ und
		$\forall f\in\L{q}:\norm{f}_p\leq\lambda_d(X)^{\frac{1}{p}-\frac{1}{q}}\norm{f}_q$.
	\end{satz}
	\begin{satz}[Konvergenzsatz von Lebesgue]
		Sei $1\leq p<\infty$, $f\colon X\to\R$ messbar, $g\colon X\to[0,\infty]$ integrierbar,
		$(f_n)$ eine Folge in $\L{p}$, $f_n\to f$ fast überall, $\forall n\in\mathbb{N}: \abs{f_n}^p\leq g~\text{f.ü}$.
		Dann $f\in\L{p}$ und $\norm{f_n-f}_p\to 0\en$.
	\end{satz}
	\begin{definition}
		\begin{enumerate}[label=(\alph*)]
			\item $\mathcal{N}\coloneqq\set{f\colon X\to\R\given f~\text{messbar}, f=0~\text{f.ü.}}$.
			\item $\LL{p}\coloneqq\faktor{\L{p}}{\mathcal{N}}$.
			\item Sei $\hat{f}\in\LL{p}$. $\int_X\hat{f}\dd{x}\coloneqq\int_Xf\dd{x}$, $\norm*{\hat{f}}_p\coloneqq\norm{f}_p$.
			\item Seien $\hat{f}, \hat{g}\in\LL{2}$. $(\hat{f}\mid\hat{g})\coloneqq\int_Xfg\dd{x}$.
		\end{enumerate}
	\end{definition}
	\begin{satz}[von Riesz-Fischer]
		Es sei $p\in[1,\infty)$. $(\hat{f}_n)$ sei eine Cauchyfolge in $\LL{p}$. Dann existieren $f\in\L{p}$ und
		eine Teilfolge $(f_{n_j})$ mit $f_{n_j}\to f$ fast überall, $\norm{f_n-f}_p\to0\en$.
	\end{satz}
	\begin{satz}
		Es seien $p, q\in[1,\infty)$, $(f_n)$ eine Folge in $\L{p}\cap\L{q}$, $f\in\L{p}$, $g\in\L{q}$,
		$\norm{f_n-f}_p\to 0\en$, $\norm{f_n-g}_q\to 0\en$. Dann $f=g$ fast überall.
	\end{satz}
	\begin{definition}
		\begin{enumerate}[label=(\alph*)]
			\item Für $f\colon X\to\R$ heißt $\supp(f)\coloneqq\overline{\set{x\in X\given f(x)\neq0}}$ der Träger von $f$.
			\item $C_c(X,\R)\coloneqq\set{f\in C(X,\R)\given \supp(f)\subseteq X, \supp(f)~\text{kompakt}}$.
		\end{enumerate}
	\end{definition}
	\begin{satz}
		\begin{enumerate}[label=(\alph*)]
			\item $C_c(X,\R)\subseteq\L{p}$.
			\item $X~\text{offen}\implies\forall f\in\L{p},\varepsilon>0~\exists g\in C_c(X,\R):\norm{f-g}_p<\varepsilon$.
		\end{enumerate}
	\end{satz}
	\newpage
	\textsc{Funktionentheorie: Analysis IV}

	\sep
	\begin{definition}
		$\varnothing\neq D\subseteq\C$ offen. $f\colon D\to\C$ heißt komplex differenzierbar
		in $z_0\in D$, falls $f'(z_0)\coloneqq\lim_{z\to z_0}\frac{f(z)-f(z_0)}{z-z_0}$ in $\C$
		existiert.
	\end{definition}
	\begin{defsatz}[Die Cauchy-Riemannschen Differentialgleichungen]
		$f=u+\i v$ ist in $z_0=x_0+\i y_0$ komplex db genau dann wenn $u, v$ in $(x_0, y_0)$
		reell db sind und
		\begin{align*}
			u_x(x_0, y_0) &= v_y(x_0, y_0), & u_y(x_0, y_0) &= -v_x(x_0, y_0)
		\end{align*}
		gilt. In diesem Fall haben wir $f'(z_0) = u_x(x_0, y_0) + \i v_x(x_0, y_0) = v_y(x_0, y_0) - \i u_y(x_0, y_0)$.
	\end{defsatz}
	\begin{satz}[Potenzreihen]
		Sei $(a_n)$ eine Folge in $\C$, $z_0\in\C$. Beträchte die PR $\sum_{n=0}^\infty a_n(z-z_0)^n$.
		$\varrho\coloneqq\limsup_{n\to\infty}\sqrt[n]{\abs{a_n}}$. $R\coloneqq\frac{1}{\varrho}$.
		\begin{enumerate}[label=(\alph*)]
			\item Falls $R=0$, so konvergiert die PR nur in $z=z_0$.
			\item Falls $R=\infty$, so konvergiert die PR absolut für jedes $z\in\C$.
			\item Falls $0<R<\infty$, konvergiert die PR in jedem $z\in U_R(z_0)$ absolut
			und divergiert in jedem $z$ mit $\abs{z-z_0}>R$.
			\item Die PR konvergiert auf jedem kompakten $K\subseteq U_R(z_0)$ gleichmäßig.
			\item $f\colon U_R(z_0)\to\C$; $z\mapsto \sum_{n=0}^\infty a_n(z-z_0)^n$ ist stetig.
			\item $f\in H(U_R(z_0))$ mit $f'(z) = \sum_{n=1}^\infty na_n(z-z_0)^{n-1}$.
		\end{enumerate}
	\end{satz}
	\begin{definition}
		$\varnothing\neq D\subseteq\C$.
		\begin{enumerate}[label=(\alph*)]
			\item $D$ heißt zusammenhängend, falls $\forall A, B\subseteq D~\text{offen}:
				A\cap B=\varnothing, A\cup B=D\implies A=\varnothing\vee B=\varnothing$.
			\item $D$ heißt Gebiet, falls $D$ offen und zusammenhängend.
			\item $D$ heißt wegzusammenhängend, falls $\forall z_1, z_2~\exists \gamma\colon [0, 1]\to D: \gamma(0)=z_1, \gamma(1)=z_2$.
		\end{enumerate}
	\end{definition}
	\begin{satz}
		$\varnothing\neq D\subseteq\C$ ist genau dann zusammenhängend, wenn $D$ wegzusammenhängend ist.
	\end{satz}
	\begin{satz}
		Sei $D$ ein Gebiet, $f\in H(D)$, $f'=0$ auf $D$. Dann ist $f$ auf $D$ konstant.
	\end{satz}
	\begin{definition}
		Es sei $w\in\C\setminus\set{0}$, $n\in\mathbb{N}$.
		\begin{enumerate}[label=(\alph*)]
			\item $z\in\C: z^n=w$ heißt $n$-te Wurzel von $w$.
			\item $z\in\C: \exp(z)=w$ heißt Logarithmus von $w$.
		\end{enumerate}
	\end{definition}
	\begin{satz}
		Sei $a, b, z\in\C, w\in \C\setminus\set{0}, x, y\in\R$.
		\begin{enumerate}[label=(\alph*)]
			\item $\exp(a+b)=\exp(a)\exp(b)$. Insbesondere $\exp(a)\neq 0$.
			\item $\exp(x)=e^x$, $\exp(\i y) = \cos y + \i\sin y$, $\abs{\exp(\i y)} = 1$.
			\item Für $z=\i x+y:\exp(z)=e^x(\cos y +\i\sin y)$. $e^z\coloneqq\exp{z}$.
			\item $z=\abs{z}e^{\i\Arg z}$.
			\item $\forall k\in\mathbb{Z}:e^z=e^{z+2k\pi\i}$.
			\item $\cos z+\i\sin z = e^{\i z}$, $\cos z=\frac{1}{2}(e^{\i z}+e^{-\i z})$, $\sin z = \frac{1}{2\i}(e^{\i z}-e^{-\i z})$,
				$\sin^2 z + \cos^2 z= 1$.
			\item $\cos z = 0\iff\exists k\in\mathbb{Z}: z=(k+\frac{1}{2})\pi$, $\sin z = 0\iff\exists k\in\mathbb{Z}:z=k\pi$.
			\item $z~\text{ist Logarithmus von $w$}\iff\exists k\in\mathbb{Z}: z = \log\abs{w}+\i\Arg w+2k\pi\i$.
			\item $e^z=1\iff\exists k\in\mathbb{Z}:z=2k\pi\i$.
			\item $z_0,\ldots, z_{n-1}$ mit $z_k\coloneqq\sqrt[n]{\abs{w}}\exp(\frac{\i}{n}(\Arg w+2k\pi))$ sind paarweise
				verschieden und alle $n$-ten Wurzeln von $w$.
		\end{enumerate}
	\end{satz}
	\begin{definition}
		\begin{enumerate}[label=(\alph*)]
			\item Für $w\in\C\setminus\set{0}$ definiere den Hauptwert des Logarithmus $\Log w\coloneqq \log\abs{w}+\i\Arg w$.
			\item $\C_-\coloneqq\C\setminus\set{z\in\R\given z\leq 0}$ heißt die geschlitzte Ebene.
		\end{enumerate}
	\end{definition}
	\begin{satz}
		$D\coloneqq\set{z\in\C\given \abs{\Im z}<\pi}$ ist ein Gebiet. Definiere $f\colon D\to\C$, $z\mapsto e^z$.
		\begin{enumerate}[label=(\alph*)]
			\item $f$ ist auf $D$ injektiv.
			\item $f(D) = \C_-$.
			\item $f$ besitzt die Umkehrfunktion $\Log$ auf $\C_-$.
			\item $\Log\in H(\C_-)$ mit $\forall w\in\C_-:\Log'(w) = \frac{1}{w}$.
		\end{enumerate}
	\end{satz}
	\sep
	\begin{definition}
		Seien $\gamma\colon [a,b]\to\C$, $\gamma_1\colon[\alpha,\beta]\to\C$ ssd, $f\colon\abs{\gamma}\to\C$ stetig.
		\begin{enumerate}
			\item $L(\gamma)\coloneqq\int_a^b\abs{\gamma'(t)}\dd{t}$ heißt die Länge von $\gamma$.
			\item $\int_\gamma f\dd{z}\coloneqq \int_\gamma f(z)\dd{z}\coloneqq\int_a^bf(\gamma(t))\cdot\gamma'(t)\dd{t}$ heißt Wegintegral.
			\item $\gamma, \gamma_1~\text{äquivalent}\defiff \abs{\gamma}=\abs{\gamma_1}\wedge\forall g\in C(\abs{\gamma}, \C):
				\int_\gamma g\dd{z}=\int_{\gamma_1}g\dd{z}$. Insbesondere ist dann $L(\gamma)=L(\gamma_1)$.
		\end{enumerate}
	\end{definition}
	\begin{satz}
		$\gamma\colon I\to\C$ ssd, $f, g\colon\abs{\gamma}\to\C$ stetig, $\lambda, \mu\in\C$.
		\begin{enumerate}[label=(\alph*)]
			\item $L(\gamma)=L(\gamma^-)$.
			\item $\int_{\gamma^-}f\dd{z}=-\int_\gamma f\dd{z}$.
			\item $\int_\gamma(\lambda f+\mu g)\dd{z}=\lambda\int_\gamma f\dd{z}+\mu\int_\gamma g\dd{z}$.
			\item Mit $M\coloneqq\max_{z\in\abs{\gamma}}\abs{f(z)}$: $\abs{\int_\gamma f\dd{z}}\leq ML(\gamma)$.
			\item Ist $f_n\colon\abs{\gamma}\to\C$ stetig, $f_n\to f$ glm: $\int_\gamma f_n\dd{z}\to\int_\gamma f\dd{z}~(n\to\infty)$.
		\end{enumerate}
	\end{satz}
	\begin{satz}
		Sei $\gamma\colon I\to\C$ ssd, $f\colon\abs{\gamma}\to\C$ stetig, $D\coloneqq \C\setminus\abs{\gamma}$,
		$z_0\in D$, $g\colon D\to\C$, $g(z)\coloneqq\int_\gamma\frac{\abs{f(w)}}{w-z}\dd{w}$. Weiter
		\begin{align*}
			r &\coloneqq \dist(z_0,\abs{\gamma}) = \inf\set{\abs{z_0-w}\given w\in\abs{\gamma}},\\
			a_n &\coloneqq \int_\gamma\frac{f(w)}{(w-z_0)^{n+1}}\dd{w}.
		\end{align*}
		Es gilt
		\begin{enumerate}[label=(\alph*)]
			\item $U_r(z_0)\subseteq D$,
			\item $\forall z\in U_r(z_0): g(z)=\sum_{n=0}^\infty a_n(z-z_0)^n$,
			\item $g\in H(D)$.
		\end{enumerate}
	\end{satz}
	\begin{definition}[Zusammenhangskomponente]
		$\varnothing\neq D\subseteq\C$ offen, $z\in D$. $D_z\coloneqq\set{w\in D\given
			\exists~\text{Weg in $D$ von $z$ nach $w$}}$ heißt Zusammenhangskomponente von $D$.
	\end{definition}
	\begin{defsatz}[Umlaufzahl]
		$\gamma\colon I\to\C$ ssd und geschlossen, $D\coloneqq\C\setminus\abs{\gamma}$.
		Für $z\in D$ heißt
		\[
			n(\gamma, z)\coloneqq\frac{1}{2\pi\i}\int_\gamma\frac{1}{w-z}\dd{w}
		\]
		die Umlaufzahl von $\gamma$ bezüglich $z$. Es gilt:
		\begin{enumerate}[label=(\alph*)]
			\item $\forall z\in D: n(\gamma, z)\in\mathbb{Z}$.
			\item $z\mapsto n(\gamma, z)$ ist konstant auf jeder Zusammenhangskomponente.
			\item Auf der unbeschränkten Zusammenhangskomponente ist die Umlaufzahl $0$.
		\end{enumerate}
	\end{defsatz}
	\begin{satz}
		$I\coloneqq [a, b]$, $\gamma\colon I\to\C$ sei ein ssd Weg, $f\colon D\to\C$ stetig,
		$F\colon D\to\C$ eine SF von $f$ auf $D$. Dann $\int_\gamma f\dd{z}=F(\gamma(b))-F(\gamma(a))$.
	\end{satz}
	\begin{satz}[Lemma von Goursat]
		Sei $f\colon D\to\C$ stetig, $p\in D$, $f\in H(D\setminus\set{p})$, $\Delta\subseteq D$ ein Dreieck.
		Dann $\int_{\partial\Delta}f\dd{z}=0$.
	\end{satz}
	\begin{satz}[Der lokale Cauchysche Integralsatz]
		$D$ konvex, $f\colon D\to\C$ stetig, $p\in D$, $f\in H(D\setminus\set{p})$.
		Dann besitzt $f$ auf $D$ eine SF.
	\end{satz}
	\begin{satz}[Die lokale Cauchysche Integralformel]
		$D$ konvex, $f\in H(D)$, $\gamma$ ssd geschlossener Weg in $D$. Dann für $z\in D\setminus\abs{\gamma}$:
		\[
			f(z)n(\gamma, z)=\frac{1}{2\pi\i}\int_\gamma\frac{f(w)}{w-z}\dd{w}.
		\]
	\end{satz}
	\begin{satz}[Der Anfang vom Ende, a.k.a. warum Funktionentheorie doch langweilig ist]\label{pre}
		Sei $f\in H(D)$, $z_0\in D$, $R\coloneqq\dist(z_0,\partial D)$. Dann existiert $(a_n)$ in $\C$ mit
		$\forall z\in U_R(z_0): f(z)=\sum_{n=0}^\infty a_n(z-z_0)^n$. Es gilt $a_n=\frac{f^{(n)}(z_0)}{n!}$.
	\end{satz}
	\begin{satz}[Eigentlich ist jeder Satz ab hier kein Satz mehr, sondern eine Folgerung aus \ref{pre}]
		Sei $f\in H(D)$. Dann ist $f$ beliebig oft komplex differenzierbar.
	\end{satz}
	\begin{satz}[von Morera]
		Sei $f\colon D\to\C$ stetig und für jedes Dreieck $\Delta\subseteq D$ gelte
		$\int_{\partial\Delta}f\dd{z}=0$. Dann $f\in H(D)$.
	\end{satz}
	\begin{satz}
		Sei $f\in H(D)$, $a\in D$ und $m\in\mathbb{N}$. Dann sind äquivalent:
		\begin{enumerate}[label=(\alph*)]
			\item $\exists g\in H(D): g(a)\neq 0\wedge \forall z\in D: (z-a)^mg(z)$,
			\item $0=f(a)=f'(a)=\cdots=f^{(m-1)}(a)$, $f^{(m)}(a)\neq 0$.
		\end{enumerate}
		$g$ und $m$ sind eindeutig bestimmt; $m$ heißt die Ordnung der Nullstelle $a$ von $f$.
	\end{satz}
	\sep
	\begin{definition}[Nullstellen]
		Sei $f\colon D\to\C$, $A\subseteq\C$.
		\begin{enumerate}[label=(\alph*)]
			\item $Z(f)\coloneqq\set{z\in D\given f(z)=0}$.
			\item $A$ heißt diskret in $D$, falls $A$ keine HP in $D$ hat, also falls
				$\forall a\in D~\exists r>0: U_r(a)\subseteq D\wedge A\cap \dot{U}_r(a)=\varnothing$.
		\end{enumerate}
	\end{definition}
	\begin{satz}
		$D$ Gebiet, $f\in H(D)$ \emph{nicht} konstant.
		\begin{enumerate}[label=(\alph*)]
			\item $\forall a\in Z(f)~\exists m\in\mathbb{N}, g\in H(D): g(a)\neq 0 \wedge \forall z\in D: f(z)=(z-a)^mg(z)$.
			\item $Z(f)$ ist diskret in $D$.
		\end{enumerate}
	\end{satz}
	\begin{satz}[Der Identitätssatz]
		Sei $D$ ein Gebiet, $f, g\in H(D)$, $A\coloneqq\set{z\in D\given f(z)=g(z)}$
		habe einen Häufungspunkt in $D$. Dann $f=g$ auf $D$.
	\end{satz}
	\begin{definition}[Singularitäten]
		Sei $a\in D$ mit $f\in H(D\setminus\set{a})$.
		\begin{enumerate}[label=(\alph*)]
			\item $a$ heißt isolierte Singularität von $f$.
			\item $a$ heißt hebbare Singularität von $f$, falls $\exists g\in H(D): \forall z\in D\setminus\set{a}: f(z)=g(z)$.
			\item $a$ heißt Pol, falls $\exists g\in H(D), m\in\mathbb{N}: g(a)\neq 0\wedge\forall z\in D\setminus\set{a}: f(z) = \frac{g(z)}{(z-a)^m}$.
				$m$ ist eineutig bestimmt und heißt Ordnung des Pols.
			\item Ist $a$ weder hebbar noch ein Pol, so heißt $a$ wesentliche Singularität.
		\end{enumerate}
	\end{definition}
	\begin{satz}[Der Riemannsche Hebbarkeitssatz]
		Es sind äquivalent:
		\begin{enumerate}[label=(\alph*)]
			\item $f$ hat in $a$ eine hebbare Singularität.
			\item $\exists r>0: U_r(a)\subseteq D\wedge f~\text{beschränkt auf $U_r(a)$}$
			\item $\lim_{z\to a}f(z)$ existiert in $\C$.
		\end{enumerate}
	\end{satz}
	\begin{satz}[Der Satz von Casorati-Weierstraß]
		Sei $a$ wesentlich, $r>0$ mit $U_r(a)\subseteq D$. Dann $\overline{f(\dot{U}_r(a))}=\C$.
	\end{satz}
	\begin{satz}[Charakterisierung von Polen]
		\begin{enumerate}[label=(\alph*)]
			\item $a~\text{Pol}\iff \abs{f(z)}\to\infty~(z\to a)$.
			\item $a~\text{Pol der Ordnung $m$}\iff \lim_{z\to a}(z-a)^mf(z)\in\C\setminus\set{0}$.
		\end{enumerate}
	\end{satz}
	\begin{satz}[Die Cauchysche Integralformel für Ableitungen]
		Sei $D$ konvex, $f\in H(D)$, $\gamma$ ssd geschlossen. Dann für $n\in\mathbb{N}_0, z\in D\setminus\abs{\gamma}$:
		\[
			f^{(n)}(z)n(\gamma, z) = \frac{n!}{2\pi\i}\int_\gamma \frac{f(w)}{(w-z)^{n+1}}\dd{w}
		\]
	\end{satz}
	\begin{satz}[Der Satz von Liouville]
		Sei $f\in H(\C)$ beschränkt. Dann ist $f$ konstant auf $\C$.
	\end{satz} %TODO FIXME AAAAA Lemma 7.13?
	\begin{satz}[Der Satz von der Gebietstreue]
		Es sei $f\colon H(D)$ nicht konstant. Ist $D$ offen, so ist $f(D)$ offen. Ist $D$ ein Gebiet, so ist $f(D)$ ein Gebiet.
	\end{satz}
	\begin{satz}[Das Maximumsprinzip und das Minimumsprinzip]
		$D$ Gebiet, $f\in H(D)$ nicht konstant.
		\begin{enumerate}[label=(\alph*)]
			\item $\abs{f}$ hat auf $D$ kein lokales Maximum.
			\item Ist $Z(f)=\varnothing$, so hat $\abs{f}$ auf $D$ kein lokales Minimum.
			\item Ist $D$ beschränkt und $f\colon\overline{D}\to\C$ stetig, so gilt $\forall z\in D: \abs{f(z)} < \max_{w\in\partial D}\abs{f(w)}$.
		\end{enumerate}
	\end{satz}
	\begin{satz}[Das Lemma von Schwarz]
		Sei $f\in H(\D)$, $f(\D)\subseteq\D$, $f(0)=0$. Dann:
		\begin{enumerate}[label=(\alph*)]
			\item $\abs{f'(0)}\leq 1$, $\forall z\in\D:\abs{f(z)}\leq \abs{z}$.
			\item $(\abs{f'(0)}=1\vee\exists z_0\in\D\setminus\set{0}:\abs{f(z_0)}=\abs{z_0})\implies\exists c\in\C:\abs{c}=1\wedge\forall z\in\D:
				f(z)=cz$.
		\end{enumerate}
	\end{satz}
	\begin{satz}[So eine Art Umkehrsatz]
		$f\in H(D)$, $z_0\in D$, $f'(z_0)\neq 0$. Dann $\exists r>0:U_r(z_0)\subseteq D$, sodass
		\begin{enumerate}[label=(\alph*)]
			\item $\forall z_1, z_2\in U_r(z_0): \abs{f(z_1)-f(z_2)}\geq\frac{1}{2}\abs{f'(z_0)}\abs{z_1-z_2}$.
			\item $f$ ist auf $U_r(z_0)$ injektiv.
			\item $f(U_r(z_0))$ ist ein Gebiet.
			\item $f^{-1}\in H(f(U_r(z_0)))$ mit $\forall w\in f(U_r(z_0)): (f^{-1})'(w)=\frac{1}{f'(f^{-1}(w))}$.
		\end{enumerate}
	\end{satz}
	\begin{satz}
		Ist $f\in H(D)$ injektiv, so ist $f'\neq 0$ auf $D$.
	\end{satz}
	\begin{satz}[Ein Injektivitätskriterium]
		$D$ konvex, $f\in H(D)$, $\Re f'\neq 0$ oder $\Im f'\neq 0$ auf $D$. Dann ist $f$ auf $D$ injektiv.
	\end{satz}
	\begin{definition}[Lokal gleichmäßige Konvergenz]
		$(f_n)$ Folge in $H(D)$ konvergiert auf $D$ lokal gleichmäßig, falls $(f_n)$ auf
		jeder kompakten Teilmenge von $D$ gleichmäßig konvergiert.
	\end{definition}
	\newpage
	\textsc{Differentialgleichungen: Analysis IV}

	\sep
	\begin{definition}[Lokal Lipschitz-stetig]
		$\varnothing\neq D\subseteq \R\times\R^p$, $f\colon D\to\R^p$ heißt lLs bzgl. $x$
		$\defiff\forall(x_0, t_0)\in D~\exists\delta>0, L>0~\forall (t, x), (t, \bar{x})\in D$:
		\[
			t\in U_\delta(t_0)\wedge x, \bar{x}\in U_\delta(x_0)\implies\norm{f(t, x)-f(t, \bar{x})}\leq L\norm{x-\bar{x}}.
		\]
	\end{definition}
	\begin{satz}[Picard-Lindelöf]
		$\varnothing\neq D\subseteq\R\times\R^p$ offen, $(x_0, t_0)\in D$, $f\colon D\to\R^p$
		stetig und lLs. Dann hat das AWP $\begin{cases}x'(t)=f(t, x(t))\\x(t_0)=x_0\end{cases}$
		eine eindeutige nicht fortsetzbare Lösung $x\colon J\to\R^p$ mit $t_0\in J^\circ$.
	\end{satz}
	\begin{satz}
		AWP wie oben, $x\colon [t_0,t_+)\to\R^p$ eine Lösung, $G_+\coloneqq\set{(t, x(t))\given t\in[t_0,t_+)}\subseteq D$.
		Es sind äquivalent:
		\begin{enumerate}[label=(\alph*)]
			\item $\omega_+=t_+$
			\item $\nexists K~\text{kompakt}: G_+\subseteq K\subseteq D$.
		\end{enumerate}
	\end{satz}
	\sep
	\begin{definition}[Autonome Differentialgleichung] % TODO FIXME AAAAAAAAAAAAA lLs
		$\varnothing\neq D\subseteq \R^p$, $g\colon D\to\R^p$. $x'(t)=g(x(t))$ heißt autonom. $x_0\in D: g(x_0)=0$ heißt
		stationäre Stelle.
	\end{definition}
	\begin{satz}
		$g\colon D\to\R^p$ stetig, $x\colon [t_1,\infty)\to\R^p$ Lösung der autonomen Gleichung,
		es existiere $x_0\coloneqq\lim_{t\in\infty}x(t)\in D$. Dann $x_0$ stationär, $x'(t)\to 0~(t\to\infty)$.
	\end{satz}
	\begin{satz}
		$D\subseteq\R$ Intervall, $g\colon D\to\R$ stetig, $x\colon I\to\R$ löst autonome Gleichung. Dann ist $x$
		monoton.
	\end{satz}
	\begin{definition}[Bahn]
		Löst $x\colon I\to\R^p$ die autonome Gleichung, so heißt $x(I)\subseteq\R^p$ Bahn.
	\end{definition}
	\begin{definition}[Erstes Integral]
		$H\in C^1(D,\R)$ mit $\forall x\in D: H'(x)\cdot f(x)=0$ heißt erstes Integral der autonomen Gleichung.
	\end{definition}
	\begin{satz}
		$H$ erstes Integral, $x\colon I\to\R^p$ löst autonome Gleichung. Dann $\exists c\in\R~\forall t\in I: H(x(t)) = c$.
	\end{satz}
	\begin{defsatz}[Stabilität]
		Sei $x_0$ stationär.
		\begin{enumerate}[label=(\alph*)]
			\item $x_0$ heißt stabil, wenn $\forall\varepsilon>0~\exists\delta>0~\forall x_1\in D:
				\norm{x_1-x_0}<\delta\implies \omega_+=\infty\wedge\forall t\geq t_0:\norm{x(t)-x_0}<\varepsilon$,
				wobei $x\colon [t_0,\omega_+)\to\R^p$ die nach rechts nicht fortsetzbare Lösung des AWP
				$\begin{cases}x'(t)=f(x(t))\\x(t_0)=x_1\end{cases}$ ist.
			\item $x_0$ heißt asymptotisch stabil, falls $x_0$ stabil und $x(t)\to x_0~(t\to\infty)$.
			\item $x_0$ heißt instabil, falls $x_0$ nicht stabil.
		\end{enumerate}
		Diese Definitionen sind unabhängig von $t_0$.
	\end{defsatz}
	\begin{satz}[Stabilitätssatz] % TODO FIXME AAAAAAAAA f g und alles
		Sei $x_0$ stationär, $f$ differenzierbar in $x_0$.
		\begin{enumerate}[label=(\alph*)]
			\item Ist $\Re\lambda<0$ für \emph{jeden}
				Eigenwert $\lambda$ von $f'(x_0)$, so ist $x_0$ asympotisch stabil.
			\item Ist $\Re\lambda>0$ für \emph{einen} Eigenwert $\lambda$ von $f'(x_0)$, so ist
				$x_0$ instabil.
		\end{enumerate}
	\end{satz}
	\begin{definition}[Lyapunov-Funktion]
		$r>0: U_r(0)\subseteq D$. $V\colon U_r(0)\to\R$ stetig db heißt Lyapunov-Funktion
		zu $x'=f(x)$ im Punkt $x_0=0$, falls $V(0)=0, \forall x\in \dot{U}_r(0): V(x)>0,
		\forall x\in U_r(0): V'(x)\cdot f(x)\leq 0$.
	\end{definition}
	\begin{satz}
		Hat $x'=f(x)$ LF $V\colon U_r(0)\to\R$, so ist $x_0=0$ stabil. Gilt weiter
		$\forall x\in\dot{U}_r(0): V'(x)\cdot f(x)<0$, so ist $x_0=0$ asymptotisch stabil.
	\end{satz}
	\sep
	\begin{definition}[Randwertprobleme]
		$a, b\in\R$, $I\coloneqq [a, b]$, $q, r\in C(I)$, $p\in C^1(I)$, $p>0$ auf $I$,
		$\alpha_0,\alpha_1,\beta_0,\beta_1\in\R$, $(\alpha_0,\alpha_1)\neq(\beta_0,\beta_1)$, $\eta_a,\eta_b\in\R$.
		\begin{enumerate}[label=(\alph*)]
			\item $L\colon C^2(I)\to C(I)$; $u\mapsto (pu')'+qu = pu''+p'u'+qu$ heißt selbstadjungierter
				Differentialoperator 2. Ordnung.
			\item $R_a, R_b\colon C^2(I)\to\R$; $u\mapsto\begin{cases}\alpha_0 u(a)+\alpha_1p(a)u'(a)\\
				\beta_0u(b)+\beta_1p(b)u'(b)\end{cases}$ heißen Randoperatoren.
			\item $\begin{cases}Lu(t)=r(t)\\R_a(u)=\eta_a\\R_b(u)=\eta_b\end{cases}$ heißt
				Sturmsches Randwertproblem.
			\item RWP heißt homogen, falls $r=0$, $\eta_a=0=\eta_b$.
			\item Dirichletsches RWP hat Randoperatoren erster Art: $R_au=u(a)$, $R_bu=u(b)$.
			\item Neumannsches RWP hat Randoperatoren zweiter Art: $R_au=u'(a)$, $R_bu=u'(b)$.
		\end{enumerate}
	\end{definition}
	\begin{satz}
		\begin{enumerate}[label=(\alph*)]
			\item Ist $u_S\colon I\to\R$ eine feste spezielle Lösung, so haben alle Lösungen
				die Form $u=u_S+u_H$ für eine Lösung $u_H\colon I\to\R$ des homogenen RWP.
			\item Für $a_0, a_1, s\in C(I)$ lässt sich das RWP $\begin{cases}u''(t)+a_1(t)u'(t)+a_0(t)u(t)=s(t)\\
				\alpha_0u(a)=\alpha_1u'(a)=\eta_a\\\beta_0u(b)+\beta_1u'(b)=\eta_b\end{cases}$ als sturmsches
				RWP formulieren.
		\end{enumerate}
	\end{satz}
	\begin{satz}
		Es sind äquivalent:
		\begin{enumerate}[label=(\alph*)]
			\item Das sturmsche RWP ist für jedes $r,\eta_a,\eta_b$ eindeutig lösbar;
			\item Das homogene RWP ist nur trivial lösbar;
			\item Für jedes FS $u_1, u_2\colon I\to\R$ von $Lu=0$ gilt $\det\begin{pmatrix}
				R_au_1 & R_au_2\\ R_bu_1 & R_bu_2\end{pmatrix}\neq 0$,
		\end{enumerate}
		wobei die letzte Aussage genau dann für alle FS gilt, wenn sie für ein FS gilt.
	\end{satz}
	\begin{satz}
		Seien $u, v\colon I\to\R$ Lösungen von $Lu=0$. Dann gilt:
		\begin{enumerate}[label=(\alph*)]
			\item $\exists c(u,v)\coloneqq c\in\R~\forall t\in I: p(t)(u(t)v'(t) - v(t)u'(t))=c$.
			\item $u, v$ ist FS von $Lu$ genau dann wenn $c\neq 0$.
		\end{enumerate}
	\end{satz}
	\begin{satz}
		Für $\tau\in I$: $u_\tau(t)\coloneqq\frac{1}{c}(u(\tau)v(t)-u(t)v(\tau))$.
		Sei $u, v\colon I\to\R$ ein FS von $Lu=0$. Für $t\in I$ definiere \[u_S(t)\coloneqq\int_a^tu_\tau(t)r(\tau)\dd{\tau}.\]
		Dann ist $u_S(a)=0=u_S'(a)$, also $R_au_S=0$. Weiter ist $u_S$ eine spezielle Lösung von $Lu=r$.
	\end{satz}
	\begin{definition}
		Ein FS $u, v\colon I\to\R$ von $Lu=0$ heißt zulässig, wenn $R_au=0=R_bv\wedge R_bu\neq 0\neq R_av$.
	\end{definition}
	\begin{satz}
		Das homogene RWP sei nur trivial lösbar.
		\begin{enumerate}[label=(\alph*)]
			\item Es gibt zulässige FS von $Lu=0$.
			\item Sind $u, v$ und $\tilde{u},\tilde{v}$ zulässige FS von $Lu=0$, so
				$\exists\alpha,\beta\in\R:\tilde{u}=\alpha u, \tilde{v}=\beta v,
				c(\tilde{u},\tilde{v})=\alpha\beta c(u, v)$. Insbesondere $\alpha\beta\neq 0$.
		\end{enumerate}
	\end{satz}
	\begin{definition}[Die Greensche Funktion]
		Es sei $u, v\colon I\to\R$ ein zulässiges FS von $Lu=0$.
		\[G(t,\tau)\coloneqq\begin{cases}\frac{v(t)u(\tau)}{c(u, v)}, &a\leq\tau\leq t\leq b,\\
			\frac{v(\tau)u(t)}{c(u, v)}, &a\leq t\leq\tau\leq b\end{cases}
		\]
		heißt die Greensche Funktion zu $L, R_a, R_b$.
	\end{definition}
	\begin{satz}
		Das homogene RWP sei nur trivial lösbar.
		Das halbhomogene RWP $\begin{cases}Lu(t)=r(t)\\R_au=R_bu=0\end{cases}$ hat die Lösung
		$w(t)=\int_a^b G(t,\tau)r(\tau)\dd{\tau}$.
	\end{satz}
	\begin{satz}
		Das homogene RWP sei nur trivial lösbar. Sei $u, v\colon I\to\R$ ein zulässiges FS
		von $Lu=0$, $G$ die Greensche Funktion und $w$ wie oben. Dann ist für $t\in I$ durch
		\[
			\tilde{w}(t)\coloneqq w(t)+\frac{\eta_b}{R_bu}u(t)+\frac{\eta_a}{R_av}v(t)
		\]
		die eindeutige Lösung des sturmschen RWP auf $I$ gegeben.
	\end{satz}
	\begin{definition}[Das namenlose RWP]
		$I\coloneqq [0, 1]$, $D\coloneqq I\times\R$, $f\colon D\to\R$ stetig.
		$\begin{cases}u''(t)=f(t,u(t))\\0=u(0)=u(1)\end{cases}$ ist das namenlose RWP.
	\end{definition}
	\begin{satz}
		Definiere $T\colon C(I)\to C(I)$ durch $Tu(x)\coloneqq\int_0^1G(x, t)f(t, u(t))\dd{t}$.
		Sei $r, u\in C(I)$, $\Phi(x)\coloneqq\int_0^1G(x, t)r(t)\dd{t}$.
		\begin{enumerate}[label=(\alph*)]
			\item $\Psi\in C^2(I)$, $\Psi''=r$, $\Psi(0)=\Psi(1)=0$.
			\item $u$ löst das namenlose RWP genau dann wenn $Tu=u$.
		\end{enumerate}
	\end{satz}
	\begin{satz}[Lettenmeyer]
		Es sei $L\in(0,\pi^2)$ und $\forall t\in[0,1], x, \bar{x}\in\R:\abs{f(t, x)-f(t,\bar{x})}\leq L\abs{x-\bar{x}}$.
		Dann hat das namenlose RWP genau eine Lösung auf $I$.
	\end{satz}
	\begin{satz}[Existenzsatz]
		Seien $A\geq 0$, $0\leq B<\pi^2$ und $\forall t\in I, x\in\R:\abs{f(t, x)}\leq A+b\abs{x}$. Dann
		hat das namenlose RWP eine Lösung auf $I$.
	\end{satz}
\end{document}

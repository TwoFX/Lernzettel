\documentclass[a4paper]{article}

\usepackage[l2tabu, orthodox]{nag}

\usepackage[utf8]{inputenc}
\usepackage[T1]{fontenc}

\usepackage[ngerman]{babel}

\usepackage{amsmath}
\usepackage{amssymb}
%\usepackage{amsthm}
\usepackage{mathtools}
\usepackage{physics}

\usepackage[framed]{ntheorem}

\usepackage{csquotes}
\usepackage{lmodern}
\usepackage{microtype}
\usepackage{enumitem}
\usepackage{stmaryrd}

\usepackage{parskip}
\usepackage{multicol}

\usepackage[left=1.8cm, right=1.8cm, top=1.8cm, bottom=2.5cm]{geometry}

\newcounter{Sec}

\theoremstyle{marginbreak}
\theorembodyfont{\normalfont}
\newtheorem{definition}{Definition}[Sec]
\newtheorem{satz}[definition]{Satz}
\newtheorem{defsatz}[definition]{Definition und Satz}
\newtheorem{verfahren}[definition]{Verfahren}
\newtheorem{defver}[definition]{Definition und Verfahren}
\newtheorem{defsatzver}[definition]{Definition, Satz und Verfahren}
\newtheorem{satzver}[definition]{Satz und Verfahren}

\MakeOuterQuote{"}
\DeclareMathOperator{\ffa}{ffa}

\newcommand{\sep}{%
	\rule{\textwidth}{0.3pt}%
	\stepcounter{Sec}%
	}
\newcommand{\defiff}{:\Longleftrightarrow}

\newcommand{\en}{~(n\to\infty)}
\newcommand{\series}[1][1]{\sum_{n=#1}^\infty}
\newcommand{\ps}[1][a]{\series[0]#1_n(x-x_0)^n}

\DeclarePairedDelimiterX\set[1]\lbrace\rbrace{\def\given{\;\delimsize\vert\;}#1}

\begin{document}
	\textsc{Analysis I}

	\sep
	\begin{defsatz}
		Zu $\varnothing\neq M\subseteq\mathbb{R}$ heißt $\gamma\in\mathbb{R}$
		\begin{description}
			\item[Obere Schranke] $\defiff \forall x\in M: \gamma\geq x$
			\item[Supremum] $\defiff \gamma~\text{ist OS} \wedge \forall~\text{OS $x$ von $M$}: \gamma\leq x$
			\item[Maximum] $\defiff \gamma = \sup M\in M$
		\end{description}
		Besitzt $M$ eine obere Schranke, besitzt es auch ein Supremum.
	\end{defsatz}
	\begin{satz}
		Für $\varnothing\neq M\subseteq\mathbb{R}$ und obere Schranke $\gamma$ von $M$ gilt
		$\gamma = \sup M\iff\forall\varepsilon>0~\exists x\in M:x>\gamma-\varepsilon$.
	\end{satz}
	\sep
	\begin{satz}[Betragssätze]
		$\forall a,b\in\mathbb{R}:$
		\begin{enumerate}[label=(\alph*)]
			\item $\abs{ab}=\abs{a}\abs{b}$
			\item $\pm a\leq\abs{a}$
			\item $\abs{a+b}\leq\abs{a}+\abs{b}$
		\end{enumerate}
	\end{satz}
	\begin{defsatz}
		\begin{enumerate}[label=(\alph*)]
			\item Für $a\in\mathbb{R}$ ist $[a]$ mit $[a]\in\mathbb{Z}\wedge [a]\leq a<[a]+1$ existent und eindeutig.
			\item Für $x,y\in\mathbb{R}$ mit $x<y$ existiert $r\in\mathbb{Q}$ mit $x<r<y$.
		\end{enumerate}
	\end{defsatz}
	\sep
	\begin{definition}[Folge]
		Eine Abbildung $\mathbb{N}\to B$ heißt Folge in $B$.
	\end{definition}
	\begin{defsatz}
		Eine nichtleere Menge $B$ heißt
		\begin{description}
			\item[endlich] $\defiff\exists n\in\mathbb{N},f\colon \set{1,\ldots,n}\to B~\text{surjektiv}$
			\item[abzählbar] $\defiff\exists f\colon\mathbb{N}\to B~\text{surjektiv}$
		\end{description}
		Die abzählbaren Mengen sind über dem kartesischen Produkt und der
		Vereinigung abzählbar vieler Mengen abgeschlossen.
	\end{defsatz}
	\sep
	\begin{defsatz}[Binomischer Lehrsatz]
		Für $n, k\in\mathbb{N}$ ist $\binom{n}{k}\coloneqq\frac{n!}{k!(n-k)!}$.
		Für $a, b\in\mathbb{R}, n\in\mathbb{N}$ gilt
		\[(a+b)^n = \sum_{k=0}^n\binom{n}{k}a^{n-k}b^k\]
	\end{defsatz}
	\begin{satz}[Bernoullische Ungleichung]
		Für $n\in\mathbb{N}, -1\leq x\in\mathbb{R}$ gilt $(1+x)^n\geq1+nx$.
	\end{satz}
	\sep
	\begin{definition}
		Eine Folge heißt beschränkt, wenn die Menge der Folgengleider beschränkt
		ist. $\sup_{n\in\mathbb{N}}a_n\coloneqq\sup\set{a_n\given n\in\mathbb{N}}$.
	\end{definition}
	\begin{defsatz}[Konvergenz]
		Eine Folge $(a_n)$ in $\mathbb{R}$ heißt konvergent, wenn ein $a\in\mathbb{R}$ existiert, sodass
		\[\lim_{n\to\infty}(a_n) = a \defiff a_n\to a\en \defiff
			\forall\varepsilon>0~\exists n_0\in\mathbb{N}~\forall n>n_0:\abs{a_n-a}<\varepsilon\]
		Falls der Grenzwert $a$ existiert, ist er eindeutig und $(a_n)$ ist beschränkt.
	\end{defsatz}
	\begin{satz}
		Es seien $(a_n), (b_n)$ Folgen in $\mathbb{R}$ und $a\in\mathbb{R}$.
		\begin{enumerate}[label=(\alph*)]
			\item Aus $(a_n) = (b_n) \ffa n\in\mathbb{N}$ folgt für $a_n\to a\en \iff b_n\to a\en$.
			\item $a_n\to a\en \iff \abs{a_n-a}\to0\en$.
		\end{enumerate}
	\end{satz}
	\begin{satz}[Konvergenzsätze]
		Es seien $(a_n), (b_n), (c_n)$ Folgen in $\mathbb{R}$ und mit $(a_n)\to a,b_n\to b\en$.
		\begin{enumerate}[label=(\alph*)]
			\item $a_n\leq b_n\ffa n\in\mathbb{N}\implies a\leq b$.
			\item $a=b\wedge a_n\leq c_n\leq b_n\ffa n\in\mathbb{N}\implies c_n\to a\en$.
			\item $\abs{a_n}\to\abs{a}\en$.
			\item $a_n+b_n\to a+b\en$.
			\item $\alpha a_n\to\alpha a\en$ für $\alpha\in\mathbb{R}$.
			\item $a_n b_n\to ab\en$.
			\item $b\neq0\implies \exists m\in\mathbb{N}: (\forall n\geq m: b_n\neq0) \wedge ((\frac{1}{b_n})_{n\geq m}\to\frac{1}{b}\en)$.
		\end{enumerate}
	\end{satz}
	\begin{definition}
		Eine Folge $(a_n)\in\mathbb{R}$ heißt
		\begin{description}
			\item[monoton wachsend] $\defiff\forall n\in\mathbb{N}:a_{n+1}\geq a_n$
			\item[streng monoton wachsend] $\defiff\forall n\in\mathbb{N}:a_{n+1}>a_n$
		\end{description}
	\end{definition}
	\begin{satz}
		Ist $(a_n)$ streng monoton wachsend und nach oben beschränkt, dann ist $\lim_{n\to\infty} a_n=\sup_{n\in\mathbb{N}}a_n$.
	\end{satz}
	\begin{satz}
		Für $(a_n)$ Folge in $\mathbb{R}$ mit $a_n>0~\forall n\in\mathbb{N}$ und $a_n\to a\en$,
		$p\in\mathbb{N}$ mit $p\geq2$. Dann $\sqrt[p]{a_n}\to\sqrt[p]{a}\en$.
	\end{satz}
	\begin{satz}
		Es sei $x\in\mathbb{R}$ und $\forall n\in\mathbb{N}:a_n\coloneqq x^n$.
		\begin{enumerate}[label=(\alph*)]
			\item $(a_n)~\text{konvergiert}\iff x\in(-1,1]$.
			\item $\lim_{n\to\infty}a_n=0\iff x\in(-1,1)$.
			\item $\lim_{n\to\infty}a_n=1\iff x =1$.
		\end{enumerate}
	\end{satz}
	\begin{satz}
		Für $x\in\mathbb{R},n\in\mathbb{N}$ gilt \[s_n\coloneqq\sum_{k=0}^nx^k=\begin{cases}
			n+1 &\text{falls $x=1$}\\
			\frac{1-x^{n+1}}{1-x} &\text{sonst}
		\end{cases}\]
		$(s_n)~\text{konvergiert}\iff\abs{x}<1$. In diesem Fall $\lim_{n\to\infty}s_n=\frac{1}{1-x}$.
	\end{satz}
	\begin{satz}
		$\sqrt[n]{n}\to1\en$. $\forall c>0:\sqrt[n]{c}\to1\en$.
	\end{satz}
	\begin{defsatz}[Die eulersche Zahl]
		\[ e\coloneqq \lim_{n\to\infty}(1+\frac{1}{n})^n = \lim_{n\to\infty}\sum_{k=0}^n\frac{1}{k!}\]
	\end{defsatz}
	\sep
	\begin{definition}[$\varepsilon$-Umgebung]
		$\forall\alpha\in\mathbb{R},\varepsilon>0:U_\varepsilon(\alpha)\coloneqq\set{a\in\mathbb{R}\given\abs{a-\alpha}<\varepsilon}
		=(\alpha-\varepsilon,\alpha+\varepsilon)$.
	\end{definition}
	\begin{definition}[Häufungswert]
		Es sei $(a_n)$ Folge in $\mathbb{R}$. $\alpha\in\mathbb{R}~\text{ist Häufungswert von $(a_n)$}\defiff\forall\varepsilon>0:a_n\in
		U_\varepsilon(\alpha)~\text{für unendlich viele}~n\in\mathbb{N}$. $\mathcal{H}(a_n)\coloneqq\set{\alpha\in\mathbb{R}\given
		\alpha~\text{ist Häufungswert von $(a_n)$}}$.
	\end{definition}
	\begin{definition}[Teilfolge]
		Für $(a_n)_n$ Folge in $\mathbb{R}$ und $(n_k)_k$ streng monoton wachsende
		Folge in $\mathbb{N}$ heißt $(a_{n_k})_k$ Teilfolge von $(a_n)$.
	\end{definition}
	\begin{satz}
		Für Folge $(a_n)$ in $\mathbb{R}$ und $\alpha\in\mathbb{R}$ gilt
		$\alpha\in\mathcal{H}(a_n)\iff\exists (n_k)_k~\text{streng monoton wachsend in}~\mathbb{N}:a_{n_k}\to\alpha~(k\to\infty)$.
	\end{satz}
	\begin{satz}
		Für $(a_n)$ konvergent gilt $\mathcal{H}(a_n)=\set{\lim_{n\to\infty} a_n}$. Für jede Teilfolge $(a_{n_k})$ gilt
		$\lim_{k\to\infty} a_{n_k} = \lim_{n\to\infty} a_n$.
	\end{satz}
	\begin{satz}[von Bolzano-Weierstraß]
		Für $(a_n)$ beschränkt gilt $\mathcal{H}(a_n)\neq\varnothing$.
	\end{satz}
	\sep
	\begin{defsatz}[Oberer und unterer Limes]
		Für $(a_n)$ beschränkte Folge in $\mathbb{R}$ existieren $\lim\sup a_n\coloneqq\max\mathcal{H}(a_n)$
		und $\lim\inf a_n\coloneqq\min\mathcal{H}(a_n)$.
	\end{defsatz}
	\begin{satz}
		Für beschränkte Folge $(a_n)$ in $\mathbb{R}$ sind äquivalent:
		\begin{enumerate}[label=(\alph*)]
			\item $\lim\inf a_n=\lim\sup a_n$
			\item $\abs{\mathcal{H}(a_n)} = 1$
			\item $(a_n)$ ist konvergent
		\end{enumerate}
	\end{satz}
	\sep
	\begin{defsatz}[Cauchy-Kriterium]
		Eine Folge $(a_n)$ in $\mathbb{R}$ heißt Cauchyfolge, wenn
		$\forall\varepsilon>0~\exists n_0\in\mathbb{N}~\forall n,m\geq n_0:\abs{a_n-a_m}<\varepsilon$.
		Dies ist genau dann der Fall, wenn $(a_n)$ konvergent ist.
	\end{defsatz}
	\sep
	\begin{definition}[Unendliche Reihe]
		Ist $(a_n)$ eine Folge in $\mathbb{R}$, so heißt die Folge $\series a_n\coloneqq(s_n)$
		mit $\forall n\in\mathbb{N}:s_n\coloneqq\sum_{k=1}^na_k$ eine (unendliche)
		Reihe. Falls $\series a_n$ konvergent, so ist der Reihenwert
		$\series a_n\coloneqq\lim_{n\to\infty}s_n$.
	\end{definition}
	\begin{satz}
		Ist $(a_n)$ eine Folge in $\mathbb{R}$ und $\series a_n$ konvergent,
		so ist $\lim a_n=0$.
	\end{satz}
	\begin{defsatz}[Absolute Konvergenz]
		Eine Reihe $\series a_n$ ist genau dann absolut konvergent, wenn
		$\series\abs{a_n}$ konvergent ist. Eine absolut konvergente
		Reihe ist konvergent und es gilt
		$\abs{\series a_n}\leq\series\abs{a_n}$.
	\end{defsatz}
	\begin{satz}[Leibnizkriterium]
		Für monoton fallende Nullfolge $(a_n)$ ist $\series (-1)^na_n$
		konvergent.
	\end{satz}
	\begin{satz}[Majorantenkriterium]
		$\abs{a_n}\leq b_n\ffa n\in\mathbb{N}\wedge\series b_n~\text{konvergiert}
		\implies\series a_n~\text{konvergiert absolut}$.
	\end{satz}
	\begin{satz}[Minorantenkriterium]
		$a_n\geq b_n\geq0\ffa n\in\mathbb{N}\wedge\series b_n~\text{divergiert}
		\implies\series a_n~\text{divergiert}$.
	\end{satz}
	\begin{satz}[Wurzelkriterium]
		Für Folge $(a_n)$ in $\mathbb{R}$ setze $\alpha\coloneqq\lim\sup\sqrt[n]{\abs{a_n}}$.
		\begin{enumerate}[label=(\alph*)]
			\item $\alpha<1\implies\series a_n~\text{konvergiert absolut}$
			\item $\alpha>1\implies\series a_n~\text{divergiert}$
		\end{enumerate}
	\end{satz}
	\begin{satz}[Quotientenkriterium]
		Für Folge $(a_n)$ in $\mathbb{R}$ mit $a_n\neq0\ffa n\in\mathbb{N}$ definiere
		$(\alpha_n)$ mit $\alpha_n\coloneqq\frac{a_{n+1}}{a_n}\ffa n\in\mathbb{N}$.
		\begin{enumerate}[label=(\alph*)]
			\item $\abs{\alpha_n}\geq1\ffa n\in\mathbb{N}\implies\series a_n~\text{ist divergent}$
			\item Falls $(\alpha_n)$ beschränkt, $\beta\coloneqq\lim\inf\abs{\alpha_n}$,
			$\alpha\coloneqq\lim\sup\abs{\alpha_n}$.
			\begin{enumerate}[label=(\roman*)]
				\item $\beta>1\implies\series a_n~\text{ist divergent}$
				\item $\alpha<1\implies\series a_n~\text{ist absolut konvergent}$
			\end{enumerate}
		\end{enumerate}
	\end{satz}
	\sep
	\begin{definition}
		Ist $(a_n)$ eine Folge und $\sigma\colon\mathbb{N}\to\mathbb{N}$ bijektiv,
		dann heißt $(b_n)$ mit $b_n\coloneqq a_{\sigma(n)}$ eine Umordnung von
		$(a_n)$. Die Reihe über die Umordnung wird als Umordnung der Reihe bezeichnet.
	\end{definition}
	\begin{satz}
		$(b_n)$ sei eine Umordnung von $(a_n)$.
		\begin{enumerate}[label=(\alph*)]
			\item $(a_n)~\text{konvergent}\implies\lim a_n=\lim b_n$
			\item $\series a_n~\text{absolut konvergent}\implies\series b_n~\text{absolut konvergent}
				\wedge\series a_n=\series b_n$
		\end{enumerate}
	\end{satz}
	\begin{satz}[Riemannscher Umordnungssatz]
		Ist $\series a_n$ konvergent,
		aber nicht absolut konvergent, so existiert für jedes $s\in\mathbb{R}\cup
		\set{\infty,-\infty}$ eine Umordnung $(b_n)$ von $(a_n)$ mit
		$\series b_n = s$.
	\end{satz}
	\sep
	\begin{defsatz}[Produktreihe]
		$\series[0] p_n$ heißt Produktreihe von $\series[0] a_n$ und
		$\series[0] b_n$, wenn $\set{p_n\given n\in\mathbb{N}_0} = \set{a_jb_k\given j,k\in\mathbb{N}_0}$,
		weobei jedes $a_jb_k$ genau einmal vorkommt. Alle Produktreihen sind
		Umordnungen voneinander.
	\end{defsatz}
	\begin{definition}[Cauchyprodukt]
		Die unendliche Reihe $\series[0] c_n\coloneqq\series[0] \sum_{k=0}^n a_kb_{n-k}$ heißt
		Cauchyprodukt von $\series[0] a_n$ und $\series[0] b_n$.
	\end{definition}
	\begin{satz}
		Sind $\series[0] a_n$ und $\series[0] b_n$ absolut konvergent, so alle
		Produktreihen $\series[0] p_n$ und das Cauchyprodukt $\series[0] c_n$
		absolut konvergent mit
		$\series[0] p_n = \series[0] c_n = (\series[0] a_n)(\series[0] b_n)$.
	\end{satz}
	\begin{defsatz}[Die e-Funktion]
		Die Definition $\forall x\in\mathbb{R}:e^x\coloneqq E(x)\coloneqq\series[0]\frac{x^n}{n!}$
		deckt sich mit der Potenz für rationale Exponenten.
	\end{defsatz}
	\begin{definition}[Kosinus und Sinus]
		$\forall x\in\mathbb{R}$:
		\begin{align*}
			\cos x &\coloneqq\series[0](-1)^n\frac{x^{2n}}{(2n)!}\\
			\sin x &\coloneqq\series[0](-1)^n\frac{x^{2n+1}}{(2n+1)!}
		\end{align*}
	\end{definition}
	\sep
	\begin{definition}[Potenzreihe]
		Für Folge $(a_n)_{n=0}^\infty$ in $\mathbb{R}$ und $x_0\in\mathbb{R}$
		heißt $\ps$ Potenzreihe.
	\end{definition}
	\begin{satz}[Konvergenz von Potenzreihen]
		Für Potenzreihe $\ps$ setze
		$\rho\coloneqq\lim\sup\sqrt[n]{\abs{a_n}}$ und $r\coloneqq\frac{1}{\rho}$,
		wobei $r\coloneqq0$, falls $\rho=\infty$ und $r\coloneqq\infty$, falls
		$\rho=0$.
		\begin{enumerate}[label=(\alph*)]
			\item Falls $r=0$ konvergiert die Potenzreihe genau für $x=x_0$.
			\item Falls $r=\infty$ konvergiert die Potenzreihe absolut für alle $x\in\mathbb{R}$.
			\item Falls $r\in\mathbb{R}$ konvergiert die Potenzreihe für $x \in (x_0 - r, x_0 + r)$
				absolut und divergiert für $x \in (-\infty, x_0-r) \cup (x_0+r, \infty)$. Für
				$x\in\set{x_0+r,x_0-r}$ ist keine allgemeine Aussage möglich.
		\end{enumerate}
		$r$ wird auch als der Konvergenzradius der Potenzreihe bezeichnet.
	\end{satz}
	\begin{satz}[Konvergenzradien von Cauchyprodukten]
		Für Potenzeihen $\ps$ und $\ps[b]$ mit Konvergenzradien $r_1$ und $r_2$
		setze $R\coloneqq\min\set{r_1,r_2}$, $c_n\coloneqq\sum_{k=0}^na_kb_{n-k}$.
		$r$ sei der Konvergenzradius von $\ps[c]$. Dann ist $R\leq r$ und für
		$x\in(x_0-R,x_0+R)$ gilt $\ps[c]=(\ps)(\ps[b])$.
	\end{satz}
	\sep
	\begin{definition}[Häufungspunkt]
		$x_0~\text{ist Häufungspunkt von}~D\subseteq\mathbb{R}
		\defiff\exists(x_n)~\text{in $D$}:x_n\to x_0\en$
	\end{definition}
	\begin{definition}
		Für $\varnothing\neq D\subseteq\mathbb{R}$, $\delta>0$ sind
		$D_\delta(x_0)\coloneqq D\cap U_\delta(x_0)$, $\dot{D}_\delta(x_0)\coloneqq
		D_\delta(x_0)\setminus\set{x_0}$.
	\end{definition}
	\begin{definition}
		Ist $(a_n)$ Folge in $\mathbb{R}$, so ist
		$\lim a_n = \infty\defiff \forall c>0~\exists n_0\in\mathbb{N}~\forall n\geq n_0:a_n>c$.
	\end{definition}
	\begin{defsatz}[Funktionsgrenzwert]
		Für $\varnothing\neq D\subseteq\mathbb{R}$, $x_0$ Häufungspunkt von $D$,
		$f\colon D\to\mathbb{R}$ ist
		\begin{align*}
			\lim_{x\to x_0}f(x) = a
			\defiff&~\forall(x_n)~\text{in}~D\setminus\set{x_0}~\text{mit}~x_n\to x_0\en:f(x_n)\to a\en\\
			\Longleftrightarrow&~\forall\varepsilon>0~\exists\delta>0~\forall x\in\dot{D}_\delta(x_0):\abs{f(x)-a}<\varepsilon
		\end{align*}
		Existiert $\lim_{x\to x_0}f(x)$, dann ist er eindeutig.
	\end{defsatz}
	\sep
	\begin{definition}[Stetigkeit]
		Für $\varnothing\neq D\subseteq\mathbb{R}$, $x_0\in D$, $f\colon D\to\mathbb{R}$
		ist \[f~\text{stetig in}~x_0\defiff\forall(x_n)\in D~\text{mit}~x_n\to x_0\en:f(x_n)\to f(x_0)\en\]
	\end{definition}
\end{document}

\documentclass[a4paper]{article}

\usepackage[l2tabu, orthodox]{nag}

\usepackage[utf8]{inputenc}
\usepackage[T1]{fontenc}

\usepackage[ngerman]{babel}

\usepackage{amsmath}
\usepackage{amssymb}
\usepackage{mathtools}
\usepackage{physics}
\usepackage{dsfont}

\usepackage[framed]{ntheorem}

\usepackage{csquotes}
\usepackage{lmodern}
\usepackage{microtype}
\usepackage{enumitem}

\usepackage{faktor}

\usepackage{parskip}
\usepackage{multicol}
\usepackage{etoolbox}

\usepackage[hidelinks]{hyperref}

\usepackage[left=1.8cm, right=1.8cm, top=1.8cm, bottom=2.5cm]{geometry}

\newcounter{Sec}

\theoremstyle{marginbreak}
\theorembodyfont{\normalfont}
\newtheorem{definition}{Definition}[Sec]
\newtheorem{satz}[definition]{Satz}
\newtheorem{defsatz}[definition]{Definition und Satz}
\newtheorem{verfahren}[definition]{Verfahren}
\newtheorem{defver}[definition]{Definition und Verfahren}
\newtheorem{defsatzver}[definition]{Definition, Satz und Verfahren}
\newtheorem{satzver}[definition]{Satz und Verfahren}
\newtheorem{folgerung}[definition]{Folgerung}

\MakeOuterQuote{"}


\newcommand{\sep}{%
	\rule{\textwidth}{0.3pt}%
	\stepcounter{Sec}%
	}
\newcommand{\defiff}{\mathrel{\vcentcolon\Longleftrightarrow}}
\newcommand\divides\mid
\newcommand\ndivides\nmid
\newcommand{\Z}{\mathbb{Z}}
\newcommand{\N}{\mathbb{N}}
\renewcommand{\P}{\mathbb{P}}
\newcommand{\R}{\mathbb{R}}
\newcommand\gen[2][]{\left\langle#2\right\rangle_{\text{#1}}}
\newcommand\nt\triangleleft

\DeclareMathOperator{\ggT}{ggT}
\DeclareMathOperator{\kgV}{kgV}
\DeclareMathOperator{\Hom}{Hom}
\DeclareMathOperator{\Kern}{Kern}
\DeclareMathOperator{\Bild}{Bild}
\undef\v
\DeclareMathOperator{\v}{v}

\DeclarePairedDelimiterX\set[1]\lbrace\rbrace{\def\given{\;\delimsize\vert\;}#1}

\begin{document}
\textsc{Einführung in Algebra und Zahlentheorie}

\sep
\begin{satz}[Eigenschaften des $\ggT$]
	Es seien $m, n\in\Z$.
	\begin{enumerate}[label=(\alph*)]
		\item $\exists c, d\in\Z:mc+nd=\ggT(m, n)$.
		\item Für $g\coloneqq \ggT(m, n)$ sind $\frac{m}{g}$ und $\frac{n}{g}$ teilerfremd.
		\item Sind $m, n$ teilerfremd, $u\in\N: m\divides nu$, dann $m\divides u$.
		\item $\kgV(m, n)\cdot\ggT(m, n)=mn$.
	\end{enumerate}
\end{satz}
\begin{defsatz}[Primzahl]
	Sei $p\in\N\setminus\set{1}$. Es sind äquivalent:
	\begin{enumerate}[label=(\alph*)]
		\item $\forall n\in\N: n\divides p\implies n\in\set{1, p}$.
		\item $\forall (a, b)\in\N^2: p\divides ab\implies p\divides a\vee p\divides b$.
	\end{enumerate}
	In diesem Fall heißt $p$ Primzahl.
\end{defsatz}
\begin{satz}[Fundamentalsatz der Arithmetik, $p$-adische Bewertung]
	Es sei $p\in\P$, $k\in\Z\setminus\set{0}$. Dann existiert ein eindeutiges
	$\v_p(k)\in\N_0$ mit $p^{\v_p(k)}\divides k$, $p^{\v_p(k)+1}\ndivides k$.
	Insbesondere besitzt $k$ die eindeutige Darstellung
	\[
		k=\pm\prod_{p\in\P}p^{\v_p(k)}
	\]
	als Produkt von Primzahlen. Für $k=0$ schreibe formal $\v_p(0)=\infty$.
	$\forall k,l\in\Z$:
	\begin{align*}
		\v_p(k+l)&\geq\min\set{\v_p(k), \v_p(l)}\\
		\v_p(kl)&=\v_p(k)+v_p(l).
	\end{align*}
\end{satz}
\begin{satz}[Eigenschaften]
	Es seien $a, b\in\N$.
	\begin{enumerate}[label=(\alph*)]
		\item $b\divides a\iff\forall p\in\P:\v_p(b)\leq v_p(a)$.
		\item Mit $e_p\coloneqq\min\set{\v_p(a), \v_p(b)}$ gilt $\ggT(a, b)=\prod_{p\in\P}p^{e_p}$.
		\item Mit $e_p\coloneqq\max\set{\v_p(a), \v_p(b)}$ gilt $\kgV(a, b)=\prod_{p\in\P}p^{e_p}$.
	\end{enumerate}
\end{satz}
\begin{satz}[Der kleine Satz von Fermat]
	$\forall p\in\P, c\in\Z: p\divides c^p-c$.
\end{satz}
\begin{satz}[Zur Verteilung der Primzahlen]
	\begin{enumerate}[label=(\alph*)]
		\item $\forall k\in\N~\exists M\in\N: [M, M+k]\cap\P=\varnothing$.
		\item $\forall x>1:\sum_{p\in\P, p\leq x}\frac{1}{p}\geq\log(\log x)-\log 2$.
		\item $\forall\varepsilon>0~\exists x_0\in\R~\forall x\geq x_0: [x, (1+\varepsilon)x]\cap\P\neq\varnothing$.
	\end{enumerate}
\end{satz}
\sep
\begin{definition}
	\begin{enumerate}[label=(\alph*)]
		\item Ein Magma ist eine Menge $M$ mit einer Verküpfung $*\colon M\cross M\to M$.
		\item Magma $(M, *)$ heißt Halbgruppe, falls $\forall l, m, n\in M: (l*m)*n=l*(m*n)$ (Assoziativgesetz).
		\item Halbgruppe mit $\exists e\in M~\forall m\in M: m*e=e*m=m$ heißt Monoid. $e$ heißt Neutralelement und ist eindeutig.
		\item $U\subseteq M$ mit $U*U\subseteq U$ heißt Untermagma.
		\item $\gen[Magma]{X}\coloneqq \bigcap_{X\subseteq U\subseteq M~\text{Untermagma}}U$.
		\item Untermagma eines Monoids, das das neutrale Element enthält, heißt Untermonoid.
		\item Sind $(M, *)$, $(N,\diamond)$ Magmen, so heißt $\Phi\colon M\to N$ mit
			$\forall m_1, m_2\in M:\Phi(m_1*m_2)=\Phi(m_1)\diamond\Phi(m_2)$ Magmenhomomorphismus.
		\item Monoid $(M, *)$ mit $\forall x\in M~\exists y\in M: x*y=y*x=e$.
		\item Für Gruppe $(G, *)$ heißt Untermagma $U$ mit $U^{-1}\subseteq U$ Untergruppe.
		\item Untergruppe $N\leq G$ mit $\forall n\in N, g\in G: gng^{-1}\in N$ heißt Normalteiler $U\nt G$.
	\end{enumerate}
\end{definition}
\begin{defsatz}[von Lagrange]
	Es sei $H\leq G$, $G$ endlich. Definiere $\forall g_1, g_2\in G: g_1\sim g_2\defiff g_1g_2^{-1}\in H$.
	Die Zahl der Äquivalenzklassen heißt Index $(G:H)$. Es gilt $\abs{G}=\abs{H}\cdot (G:H)$.
\end{defsatz}
\begin{definition}[Faktorgruppe]
	Ist $U\leq G$, so ist $G/U\coloneqq\set{gU\given g\in G}$. Falls $U\nt G$, so wird $G/U$
	auf die naheliegende Art und Weise zu einer Gruppe.
\end{definition}
\begin{satz}[Der Homomorphiesatz $\heartsuit$]
	Seien $G, H$ Gruppen, $N\nt G$.
	\begin{enumerate}[label=(\alph*)]
		\item
			$
				L\colon\Hom(G/N, H)\to\Hom(G, H); \Phi\mapsto \Phi\circ\pi_N
			$
			ist injektiv. $\Bild L=\set{\Phi\in\Hom(G, H)\given N\subseteq\Kern\Phi}$.
		\item Ist $\Phi\colon G\to H$ ein Homomorphismus, so ist durch
			$\tilde{\Phi}\colon G/\Kern\Phi\to \Bild\Phi; gN\mapsto \Phi(g)$ ein
			Isomorphismus zwischen $G/\Kern\Phi$ und $\Bild\Phi$ gegeben.
	\end{enumerate}
\end{satz}
\begin{satz}[Der erste Isomorphiesatz]
	Es sei $G$ eine Gruppe, $H\leq G$, $N\nt G$. Dann ist $HN\leq G$ und $H/(N\cap H)\cong (HN)/N$.
\end{satz}
\end{document}

\documentclass[a4paper]{article}

\usepackage[l2tabu, orthodox]{nag}

\usepackage[utf8]{inputenc}
\usepackage[T1]{fontenc}

\usepackage[ngerman]{babel}

\usepackage{amsmath}
\usepackage{amssymb}
\usepackage{amsthm}
\usepackage{mathtools}
\usepackage{physics}
\usepackage{centernot}

\usepackage{csquotes}
\usepackage{lmodern}
\usepackage{microtype}
\usepackage{enumitem}
\usepackage{stmaryrd}

\usepackage{parskip}

\newtheorem*{lemma}{Lemma}
\newtheorem{aufgabe}{Aufgabe}

\theoremstyle{definition}
\newtheorem*{mdef}{Definition}

\theoremstyle{remark}
\newtheorem*{anmerkung}{Anmerkung}
\newtheorem*{folgerung}{Folgerung}
\newtheorem{beispiel}{Beispiel}

\MakeOuterQuote{"}

\newcommand{\defiff}{:\Longleftrightarrow}

\renewcommand{\d}[1]{\mathrm{d}#1}

\DeclareMathOperator{\arsinh}{arsinh}
\DeclareMathOperator{\arcosh}{arcosh}

\DeclarePairedDelimiterX\set[1]\lbrace\rbrace{\def\given{\;\delimsize\vert\;}#1}

\begin{document}

	\section*{Äquivalente Wege}
		Seien $\gamma_1\in C([a, b], \mathbb{R}^n), \gamma_2\in C([\alpha, \beta], \mathbb{R}^n)$.
		$\gamma_1\sim\gamma_2 \defiff\exists h\in C([a, b], \mathbb{R})$ strikt wachsend mit
		$h([a, b])=[\alpha, \beta]~\text{und}~\forall t\in[a,b]:\gamma_1(t)=\gamma_2(h(t))$.
		$\gamma_1$ und $\gamma_2$ heißen dann äquivalent, in der Literatur auch oft:
		"$\gamma_2$ ist eine Umparametrisierung von $\gamma_1$". Beachte:
		$h\colon[a, b]\to[\alpha, \beta]$ ist bijektiv und $h^{-1}\colon[\alpha, \beta]
		\to[a,b]$ ist stetig und strikt wachsend und $\forall\tau\in[\alpha, \beta]:
		\gamma_2(\tau)=\gamma_1(h^{-1}(\tau))$.

		Satz 12.7: Wenn $\gamma_1\sim\gamma_2$, dann $\gamma_1~\text{rektifizierbar}\defiff
		\gamma_2~\text{rektifizierbar}$. In diesem Fall $L(\gamma_1)=L(\gamma_2)$. Achtung:
		\begin{align*}
			\gamma_1\sim\gamma_2, \gamma_1\in C^1 &\centernot\implies \gamma_2~\text{differenzierbar},\\
			\gamma_1\sim\gamma_2, \gamma_1~\text{glatt} &\centernot\implies \gamma_2~\text{glatt}.
		\end{align*}

		Sei $\gamma\in C^1([a, b], \mathbb{R}^n)$ ein Weg. Setze $\ell\coloneqq L(\gamma)$.
		\begin{align*}
			\gamma~\text{glatt}&\defiff\forall t\in[a, b]:\gamma'(t)\neq0,\\
			\gamma~\text{parametrisiert nach der Weglänge}&\defiff\forall t\in[a,b]:\norm{\gamma'(t)}=1.
		\end{align*}
		Gesehen: $\gamma~\text{glatt}\implies s\in C^1([a, b], \mathbb{R})\wedge\forall t\in[a, b]:
		s'(t)=\norm{\gamma'(t)}>0\implies s^{-1}~\text{ist strikt wachsend}
		\implies\hat{\gamma}\in C^1([0, \ell], \mathbb{R}^n), \forall\tau\in[0, \ell]:
		\hat{\gamma}\coloneqq\gamma(s^{-1}(\tau))$ ist die Umparametrisierung von
		$\gamma$ nach der Weglänge. Tatsächlich gilt $\forall\tau\in[0, \ell]$:
		\begin{align*}
			\norm{\hat{\gamma}'(\tau)}&=\norm{\gamma'(s^{-1}(\tau))(s^{-1})'(\tau)}=\norm{\gamma'(s^{-1}(\tau))\frac{1}{s'(s^{-1}(\tau))}}\\
			&= \norm{\frac{\gamma'(s^{-1}(\tau)}{\norm{\gamma'(s^{-1}(\tau))}}}=1.
		\end{align*}
		\begin{aufgabe}
			Sei $\gamma\colon[-1, 1]\to\mathbb{R}^2$; $\forall t\in[-1, 1]:\gamma(t)\coloneqq(t, \cosh t)$.
		Zeige, dass $\gamma$ glatt ist und bestimme die Parametrisierung nach der Weglänge.
		\end{aufgabe}
		\begin{proof}[Lösung]
			Es ist $\gamma\in C^1([-1, 1], \mathbb{R}^2)$ und es gilt
			$\gamma'(t)=(1, \sinh t)$ für $t\in[-1, 1]$. Offenbar ist
			$\gamma (t)\neq(0, 0)~\forall t\in[-1, 1]\implies\gamma~\text{glatt}$.
			Die Weglängenfunktion $s\colon[-1, 1]\to[0,\infty)$ ist für alle $t\in[-1, 1]$ gegeben durch
			\begin{align*}
				s(t) &= \int_{-1}^t\norm{\gamma'(\sigma)}\d{\sigma}\\
				&= \int_{-1}^t\sqrt{\cosh^2\sigma-\sinh^2\sigma+\sinh^2\sigma}\d{\sigma} \\
				&= \int_{-1}^t\abs{\cosh\sigma}\d{\sigma} \stackrel{\text{AI}}{=} \int_{-1}^t\cosh\sigma\d{\sigma} \\
				&= \sinh t - \sinh(-1).
			\end{align*}
			Bestimme Umkehrfunktion von $s$: Für $t\in[-1, 1], \tau\in[0, \infty)$:
			\begin{align*}
				\tau=s(t)=\sinh t- \sinh(-1)&\iff\sinh(t)=\tau+\sinh(-1)\\
				&\iff t=\arsinh(\tau+\sinh(-1))=s^{-1}(\tau).
			\end{align*}
			Somit gilt:
			\begin{equation*}
				\hat{\gamma}(\tau)=\gamma(s^{-1}(\tau))=(\arsinh(\tau + \sinh(-1)), \cosh(\arsinh(\tau + \sinh(-1)))).\qedhere
			\end{equation*}
		\end{proof}
		\begin{beispiel}
			$\gamma\colon[0, \pi]\to\mathbb{R}^2$, $\forall t\in[0, \pi]:\gamma(t)\coloneqq(\cos t + t \sin t, \sin t - t \cos t)
			\stackrel{(\ldots)}{\implies} \gamma'(t)=(t\cos t, t\sin t) \implies \norm{\gamma'(t)}=t$.
			Beachte $\gamma'(0)=0$. $s\colon [0, \pi]\to[0, \frac{1}{2}\pi^2]$; $s(t)=\frac{1}{2}t^2$.
			$s^{-1}\colon[0, \frac{1}{2}\pi^2]$; $s^{-1}(\tau)=\sqrt{2\tau}$.
			$\tilde{\gamma}\colon[0, \frac{1}{2}\pi^2]\to\mathbb{R}^2$ mit
			\[
				\tilde{\gamma}(\tau)=(\cos(\sqrt{2\tau})+\sqrt{2\tau}\sin(\sqrt{2\tau}),
				\sin(\sqrt{2\tau})-\sqrt{2\tau}\cos(\sqrt{2\tau}))
			\]
			ist eine Umparametrisierung von $\gamma$; die Weglängenfunktion ist die
			Identität.
		\end{beispiel}
	\section*{Wegintegral längs eines Weges}
		Sei $\gamma\in C^1([a, b], \mathbb{R}^n)$ ein Weg und $f\in C(D, \mathbb{R}^n)$,
		wobei $D\subseteq\mathbb{R}^n$ und $\Gamma_\gamma = \gamma([a, b])\subseteq D$. Definiere
		\[
			\int_\gamma f(x)\cdot \d{x}\coloneqq\int_a^bf(\gamma(t))\cdot\gamma'(t)\d{t}.
		\]
		In der Literatur heißt dieses Integral oft "Wegintegral 2. Art".

		\begin{aufgabe}
			Seien $f\colon\mathbb{R}^3\to\mathbb{R}^3$; $f(x, y, z)\coloneqq(y - x, -y, 1)$
			und $\gamma\colon[0, 2\pi]\to\mathbb{R}^3$; $\gamma(t)\coloneqq(-\sin t, \cos t, 0)$.
			Berechne
			\[
				\int_\gamma f(x, y, z)\cdot\d{(x, y, z)}.
			\]
		\end{aufgabe}
		\begin{proof}[Lösung]
			Es ist "klar", dass $\gamma\in C^1([0, 2\pi], \mathbb{R}^3)$ mit
			$\gamma'(t)=(-\cos t, -\sin t, 0)$, $t\in[0, 2\pi]$. Weiter gilt
			$f(\gamma(t)) = (\cos t + \sin t, -\cos t, 1)$. Es folgt
			\begin{align*}
				\int_\gamma f(x, y, z)\cdot\d{(x, y, z)} &= \int_0^{2\pi}
					\begin{pmatrix}\cos t + \sin t\\-\cos t\\1\end{pmatrix}\cdot
						\begin{pmatrix}-\cos t\\-\sin t\\0\end{pmatrix}\d{t}\\
						&= \int_0^{2\pi}-\cos^2(t)-\sin(t)\cos(t)+\sin(t)\cos(t)\d{t}\\
						&= -\int_0^{2\pi}\cos^2t\d{t}\\
						&= -\int_0^{2\pi}1\d{t} + \int_0^{2\pi}\underset{\downarrow}{\sin(t)}\underset{\uparrow}{\sin(t)}\d{t}\\
						&= -2\pi - [\sin(t)\cos(t)]^{2\pi}_0-\int_0^{-2\pi} \cos(t)(-\cos(t))\d{t}\\
						\implies\int_\gamma f(x, y, z)\cdot\d{(x, y, z)} &= \frac{1}{2}(-2\pi)=-\pi.\qedhere
			\end{align*}
		\end{proof}
	\section*{Wegintegral bezüglich der Weglänge}
		Sei $\gamma\in C^1([a, b], \mathbb{R}^n)$ ein Weg, $g\in C(D, \mathbb{R})$,
		$\Gamma_\gamma\subseteq D\subseteq\mathbb{R}^n$. Definiere
		\[
			\int_\gamma g(x)\d{s}\coloneqq\int_a^b g(\gamma(\sigma))\norm{\gamma'(\sigma)}\d{\sigma}.
		\]
		$s$ symbolisiert hier die Weglängenfunktion.
		\begin{beispiel}
			$g(x)\coloneqq\frac{xy}{x^2+y^2}$, $\gamma\colon[0, 2\pi]\to\mathbb{R}^2$;
			$\gamma(t)\coloneqq\frac{e^t}{\sqrt{2}}(\cos t, \sin t)$.
			\begin{align*}
				\stackrel{(\ldots)}{\implies}\norm{\gamma'(t)}&=\frac{e^t}{\sqrt{2}}
					\sqrt{2\cos^2(t)-\sin(t)\cos(t)+2\sin^2(t)+\sin(t)\cos(t)} = e^t\\
				\implies g(\gamma(t))&=\left(\frac{e^t}{\sqrt{2}}\right)^2
					\frac{\sin(t)\cos(t)}{\left(\frac{e^t}{\sqrt{2}}\right)^2(\cos^2(t)+\sin^2(t))}=\sin(t)\cos(t)\\
				\implies \int_\gamma g(x, y)\d{s}&=\int_0^{2\pi}\underset{\downarrow}{\sin(t)\cos(t)}\underset{\uparrow}{\vphantom{)}e^t}\d{t}
				= \ldots
			\end{align*}
		\end{beispiel}
\end{document}

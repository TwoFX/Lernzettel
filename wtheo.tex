\documentclass[a4paper,8pt]{article}

\usepackage[l2tabu, orthodox]{nag}

\usepackage[utf8]{inputenc}
\usepackage[T1]{fontenc}

\usepackage[ngerman]{babel}

\usepackage{amsmath}
\usepackage{amssymb}
\usepackage{mathtools}
\usepackage{physics}

\usepackage[framed]{ntheorem}

\usepackage{csquotes}
\usepackage{lmodern}
\usepackage{microtype}
\usepackage{enumitem}

\usepackage{parskip}
\usepackage{multicol}

\usepackage{dsfont}
\usepackage[left=0.8cm, right=0.8cm, top=0.8cm, bottom=0.8cm,landscape]{geometry}

\newcounter{Sec}
\pagestyle{empty}

\theoremstyle{nonumberbreak}
\theorembodyfont{\normalfont\footnotesize
\theorempreskip{0pt}
\theorempostskip{0pt}
\setlength{\abovedisplayskip}{0pt}
\setlength{\belowdisplayskip}{0pt}
\setlength{\abovedisplayshortskip}{0pt}
\setlength{\belowdisplayshortskip}{0pt}}
\setlength{\jot}{0pt}
\newtheorem{definition}{Definition}[Sec]
\newtheorem{satz}[definition]{Satz}
\newtheorem{defsatz}[definition]{Definition/Satz}
\newtheorem{verfahren}[definition]{Verfahren}
\newtheorem{defver}[definition]{Definition/Verfahren}
\newtheorem{defsatzver}[definition]{Definition/Satz/Verfahren}
\newtheorem{satzver}[definition]{Satz/Verfahren}
\newtheorem{lemma}[definition]{Lemma}

\MakeOuterQuote{"}
\DeclareMathOperator{\ffa}{ffa}

\setlength{\parskip}{0pt}
\setlength{\columnseprule}{0.15pt}
\setlength{\columnsep}{10pt}
\setlength{\multicolsep}{2pt}
\setlist[enumerate,itemize]{noitemsep, nosep}

\newcommand{\sep}{%
	\rule{\linewidth}{0.15pt}%
	\stepcounter{Sec}%
	}

\newcommand{\defiff}{\mathrel{\vcentcolon\Longleftrightarrow}}
\newcommand{\defas}{\coloneqq}

\DeclareMathOperator{\Hyp}{Hyp}
\DeclareMathOperator{\Po}{Po}
\DeclareMathOperator{\Nb}{Nb}
\DeclareMathOperator{\Mult}{Mult}
\DeclareMathOperator{\G}{G}
\DeclareMathOperator{\U}{U}
\DeclareMathOperator{\Exp}{Exp}
\DeclareMathOperator{\Nd}{N}
\DeclareMathOperator{\Gd}{\Gamma}
\DeclareMathOperator{\Bd}{B}
\DeclareMathOperator{\LN}{LN}
\DeclareMathOperator{\Wd}{W}
\DeclareMathOperator{\Cd}{C}

\DeclarePairedDelimiterX\set[1]\lbrace\rbrace{\def\given{\;\delimsize\vert\;}#1}
\DeclarePairedDelimiter\floor\lfloor\rfloor

\newcommand{\mean}{\bar}
\newcommand{\median}{\tilde}
\newcommand{\conj}{\overline}
\renewcommand{\P}{\mathbb{P}}
\newcommand{\R}{\mathbb{R}}
\newcommand{\E}{\mathbb{E}}
\newcommand{\V}{\mathbb{V}}
\newcommand{\N}{\mathbb{N}}
\newcommand{\cf}{\varphi}

\newcommand{\e}{\mathrm{e}}
\renewcommand{\i}{\mathrm{i}}

\newcommand{\sk}{\mathrel{\stackrel{\P}{\longrightarrow}}}
\newcommand{\fsk}{\mathrel{\stackrel{\operatorname{f.s.}}{\longrightarrow}}}
\newcommand{\lpk}{\mathrel{\stackrel{\mathcal{L}^p}{\longrightarrow}}}
\newcommand{\vk}{\mathrel{\stackrel{\mathcal{D}}{\longrightarrow}}}

\newcommand{\fs}[1]{~\operatorname{#1-f.s.}}
\newcommand{\fu}[1]{~\operatorname{#1-f.ü.}}

\begin{document}
\begin{multicols*}{3}
	\textsc{Name:}

	\textsc{Matrikelnummer:}

	\sep
	\begin{definition}
		\begin{enumerate}[label=(\alph*)]
			\item $\limsup_{n\to\infty} A_n\defas \bigcap_{n=1}^\infty\bigcup_{k=n}^\infty A_k$.
			\item $\liminf_{n\to\infty} A_n\defas \bigcup_{n=1}^\infty\bigcap_{k=n}^\infty A_k$.
		\end{enumerate}
	\end{definition}
	\begin{lemma}[von Borel-Cantelli]
		\begin{enumerate}[label=(\alph*)]
			\item $\sum_{n=1}^{\infty}\P(A_n)<\infty\implies \P(\limsup_{n\to\infty}A_n)=0$.
			\item $(A_n)$ unabh: $\sum_{n=1}^{\infty}\P(A_n)=\infty\implies \P(\limsup_{n\to\infty}A_n)=1$.
		\end{enumerate}
	\end{lemma}
	\sep
	\begin{definition}[Konvergenzbegriffe]
		\begin{enumerate}[label=(\alph*)]
			\item $X_n \fsk X \defiff \P(\set{\omega\in\Omega : \lim_{n\to\infty} X_n(\omega) = X(\omega)}) = 1$,
			\item $X_n \sk X \defiff \forall\varepsilon>0:\lim_{n\to\infty}\P(\abs{X_n-X}>\varepsilon)=0$,
			\item $X_n \lpk X \defiff \lim_{n\to\infty} \E\abs{X_n-X}^p = 0$,
			\item $X_n \vk X \defiff \forall x\in\mathcal{C}(F):\lim_{n\to\infty} F_n(x) = F(x)$.
		\end{enumerate}
	\end{definition}
	\begin{satz}[Charakterisierungen]
		\begin{enumerate}[label=(\alph*)]
			\item $X_n \fsk X \iff \forall\varepsilon>0:\lim\limits_{n\to\infty}\P(\set{\sup_{k\geq n}\abs{X_k-X}>\varepsilon}) = 0$,
			\item $X_n \fsk X \impliedby\forall \varepsilon>0:\sum_{n=1}^\infty\P(\abs{X_n-X}>\varepsilon)<\infty$,\\
			\item $X_n \sk X \iff \forall (X_{n_k})~\exists (X_{n_{k_j}}): X_{n_{k_j}}\fsk X$.
		\end{enumerate}
	\end{satz}
	\begin{satz}[Rechenregeln]
		$h$ stetig, $a_n\to a$, $X_n\sk X$, $Y_n\sk Y$.
		{\setlength{\columnseprule}{0pt}\begin{multicols}{2}
		\begin{enumerate}[label=(\alph*)]
			\item $h(X_n)\sk h(X)$
			\item $a_nX_n\sk aX$
			\item $aX_n+bY_n\sk aX+bY$
			\item $X_nY_n\sk XY$
			\item $\abs{X_n}\sk\abs{X}$
		\end{enumerate}
		\end{multicols}}
	\end{satz}
	\begin{lemma}[von Sluzki]
		$X_n\vk X$, $Y_n\sk a$.
		{\setlength{\columnseprule}{0pt}\begin{multicols}{2}
		\begin{enumerate}[label=(\alph*)]
			\item $X_n+Y_n\vk X+a$,
			\item $X_nY_n\vk aX$.
		\end{enumerate}
		\end{multicols}}
	\end{lemma}
	\begin{satz}[Continuous mapping theorem]
		$h\colon\R\to\R$ mb, $\P^X$-f.ü. stetig, $X_n\vk X$. Dann $h(X_n)\vk h(X)$.
	\end{satz}
	\begin{definition}[Straffheit]
		$\varnothing\neq\mathcal{Q}$ Menge von W-Maßen auf $\mathcal{B}$ heißt straff, falls
		\[\forall\varepsilon>0~\exists K\subset\R~\text{kompakt}~\forall Q\in\mathcal{Q}:Q(K)\geq 1-\varepsilon.\]
	\end{definition}
	\begin{definition}[Relative Kompaktheit]
		$\varnothing\neq\mathcal{Q}$ Menge von W-Maßen auf $\mathcal{B}$ heißt relativ kompakt, falls
		\[\forall (Q_n)~\text{in $\mathcal{Q}$}~\exists (Q_{n_k}), Q: Q_{n_k}\vk Q.\]
		Es wird \textit{nicht} $Q\in\mathcal{Q}$ gefordert.
	\end{definition}
	\begin{satz}
		$\mathcal{Q}~\text{straff}\iff\mathcal{Q}~\text{relativ kompakt}$.
	\end{satz}
	\begin{satz}
		\begin{enumerate}[label=(\alph*)]
			\item $X_n\vk D\implies\set{P^{X_n}\given n\in\N}~\text{straff}$.
			\item $(\set{P^{X_n}\given n\in\N}\wedge \exists Q~\forall (X_{n_k}): X_{n_k}\vk Q)\implies X_n\vk Q$.
		\end{enumerate}
	\end{satz}
	\begin{satz}[Lévy-Cramér]
		$X_i$ ZV mit VF $F_i$ und CF $\cf_i$. Es sind äquivalent:
		\begin{enumerate}[label=(\alph*)]
			\item $\exists F: F_n\vk F$.
			\item $\exists \cf: \cf_n\to\cf~\text{pktw.}$ und $\cf$ ist stetig in $0$.
		\end{enumerate}
		In diesem Fall: $\cf(t)=\int_\R\e^{\i tx}\dd{F(x)}$.
	\end{satz}
	\sep
	\begin{satz}[Starkes Gesetz großer Zahlen]
		$(X_n)$ u.i.v.-Folge.
		\[
			\E\abs{X_1}<\infty\iff
			\exists X:\frac{1}{n}\sum_{j=1}^nX_j\fsk X
		\]
		In diesem Fall $X=\E X_1 \fs{\P}$
	\end{satz}
	\begin{satz}[Kolmogorov-Kriterium]
		$(X_n)$ unabhängig, $\forall n:\E X_n^2\leq \infty$, $\exists (a_n): \sum_{n=1}^\infty \frac{\V X_n}{a_n^2}\leq \infty$.
		Dann \[\frac{1}{a_n}\sum_{j=1}^n(X_j-\E X_j)\fsk 0.\]
	\end{satz}
	\begin{satz}[Gleichmäßig beschränkte Varianzen]
		$(X_n)$ unabgängig, $\exists c\in[0,\infty):\sup_{n\in\N}\V X_n\leq c$. Dann
		\[\frac{1}{n}\sum_{j=1}^n(X_j-\E X_j)\fsk 0.\]
	\end{satz}
	\sep
	\begin{definition}[Charakteristische Funktion]
		\[
			\cf_X(t)\defas \E(\e^{\i tX}) = \int_\R \e^{\i tx}\P^X(\dd{x})
		\]
	\end{definition}
	\begin{satz}[Eigenschaften]
		{\setlength{\columnseprule}{0pt}\begin{multicols}{2}
		\begin{enumerate}[label=(\alph*)]
			\item $\cf_X(0)=1$, $\abs{\cf_X(t)}\leq 1$,
			\item $\cf_X$ glm stetig,
			\item $\cf_X(-t) = \conj{\cf_X(t)}$,
			\item $\cf_{aX+b}(t) = \e^{\i tb}\cf_X(at)$.
		\end{enumerate}
		\end{multicols}}
	\end{satz}
	\begin{satz}[Faltung]
		$X_1,\ldots,X_n$ unabhängig: $\cf_{X_1+\cdots+X_n}(t)=\prod_{j=1}^n\cf_{X_j}(t)$.
	\end{satz}
	\begin{satz}[Umkehrformel]
		Ist $\int_\R\abs{\cf_X(t)}\dd{t}<\infty$, so hat $X$ die
		$\lambda^1$-Dichte \[f(x) = \frac{1}{2\pi}\int_\R\e^{-\i tx}\cf_X(t)\dd{t}.\]
	\end{satz}
	\begin{satz}[Folgerungen]
		\begin{enumerate}[label=(\alph*)]
			\item $\P^X=\P^Y\iff\cf_X=\cf_Y$.
			\item $X\sim -X\iff \forall t\in\R:\cf_X(t)\in\R$.
		\end{enumerate}
	\end{satz}

\end{multicols*}
\end{document}

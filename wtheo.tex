\documentclass[a4paper,8pt]{article}

\usepackage[l2tabu, orthodox]{nag}

\usepackage[utf8]{inputenc}
\usepackage[T1]{fontenc}

\usepackage[ngerman]{babel}

\usepackage{amsmath}
\usepackage{amssymb}
\usepackage{mathtools}
\usepackage{physics}

\usepackage[framed]{ntheorem}

\usepackage{csquotes}
\usepackage{lmodern}
\usepackage{microtype}
\usepackage{enumitem}

\usepackage{parskip}
\usepackage{multicol}
\usepackage{array}
\usepackage{float}
\newcolumntype{M}[1]{>{\centering\arraybackslash}m{#1}}
\newcolumntype{N}{@{}m{0pt}@{}}

\usepackage{dsfont}
\usepackage[left=0.8cm, right=0.8cm, top=0.8cm, bottom=0.8cm,landscape]{geometry}

\newcounter{Sec}
\pagestyle{empty}

\theoremstyle{nonumberbreak}
\theorembodyfont{\normalfont\footnotesize
\theorempreskip{0pt}
\theorempostskip{0pt}
\setlength{\abovedisplayskip}{0pt}
\setlength{\belowdisplayskip}{0pt}
\setlength{\abovedisplayshortskip}{0pt}
\setlength{\belowdisplayshortskip}{0pt}}
\setlength{\jot}{0pt}
\newtheorem{definition}{Definition}[Sec]
\newtheorem{satz}[definition]{Satz}
\newtheorem{defsatz}[definition]{Definition/Satz}
\newtheorem{verfahren}[definition]{Verfahren}
\newtheorem{defver}[definition]{Definition/Verfahren}
\newtheorem{defsatzver}[definition]{Definition/Satz/Verfahren}
\newtheorem{satzver}[definition]{Satz/Verfahren}
\newtheorem{lemma}[definition]{Lemma}

\MakeOuterQuote{"}
\DeclareMathOperator{\ffa}{ffa}

\setlength{\parskip}{0pt}
\setlength{\columnseprule}{0.15pt}
\setlength{\columnsep}{10pt}
\setlength{\multicolsep}{2pt}
\setlist[enumerate,itemize]{noitemsep, nosep}

\newcommand{\sep}{%
	\rule{\linewidth}{0.15pt}%
	\stepcounter{Sec}%
	}

\newcommand{\defiff}{\mathrel{\vcentcolon\Longleftrightarrow}}
\newcommand{\defas}{\coloneqq}

\DeclareMathOperator{\Bin}{Bin}
\DeclareMathOperator{\Hyp}{Hyp}
\DeclareMathOperator{\Po}{Po}
\DeclareMathOperator{\Nb}{Nb}
\DeclareMathOperator{\Mult}{Mult}
\DeclareMathOperator{\G}{G}
\DeclareMathOperator{\U}{U}
\DeclareMathOperator{\Exp}{Exp}
\DeclareMathOperator{\Nd}{N}
\DeclareMathOperator{\Gd}{\Gamma}
\DeclareMathOperator{\Bd}{B}
\DeclareMathOperator{\LN}{LN}
\DeclareMathOperator{\Wd}{W}
\DeclareMathOperator{\Cd}{C}

\DeclarePairedDelimiterX\set[1]\lbrace\rbrace{\def\given{\;\delimsize\vert\;}#1}
\DeclarePairedDelimiter\floor\lfloor\rfloor

\newcommand{\mean}{\bar}
\newcommand{\median}{\tilde}
\newcommand{\conj}{\overline}
\newcommand{\ind}{\mathds{1}}
\renewcommand{\P}{\mathbb{P}}
\newcommand{\R}{\mathbb{R}}
\newcommand{\F}{\mathcal{F}}
\newcommand{\E}{\mathbb{E}}
\newcommand{\EV}[2]{\E[#1|#2]}
\newcommand{\BE}[2]{\EV{#1}{\mathcal{#2}}}
\newcommand{\V}{\mathbb{V}}
\newcommand{\N}{\mathbb{N}}
\newcommand{\cf}{\varphi}

\newcommand{\e}{\mathrm{e}}
\renewcommand{\i}{\mathrm{i}}

\newcommand{\sk}{\mathrel{\stackrel{\P}{\longrightarrow}}}
\newcommand{\fsk}{\mathrel{\stackrel{\operatorname{f.s.}}{\longrightarrow}}}
\newcommand{\lpk}{\mathrel{\stackrel{\mathcal{L}^p}{\longrightarrow}}}
\newcommand{\vk}{\mathrel{\stackrel{\mathcal{D}}{\longrightarrow}}}

\newcommand{\fs}[1]{~\operatorname{#1-f.s.}}
\newcommand{\fu}[1]{~\operatorname{#1-f.ü.}}

\begin{document}
\begin{multicols}{3}
	\textsc{Name:}

	\textsc{Matrikelnummer:}

	\sep
	\begin{lemma}[von Scheffé]
		$\P=f\mu$, $Q=g\mu$, $\P_n=f_n\mu$.
		\begin{enumerate}[label=(\alph*)]
			\item $\sup_{A\in\mathcal{A}}\abs{\P(A)-Q(A)}=\frac{1}{2}\smallint_\Omega\abs{f-g}\dd{\mu}$.
			\item $f_n\to f\fu{\mu}\implies \lim_{n\to\infty}\smallint_\Omega\abs{f_n-f}\dd{\mu}=0$.
		\end{enumerate}
	\end{lemma}
	\sep
	\begin{definition}
		$\limsup_{n\to\infty} A_n\defas \bigcap_{n=1}^\infty\bigcup_{k=n}^\infty A_k$.
	\end{definition}
	\begin{satz}[$\epsilon$-$\delta$-Kriterium für absolute Stetigkeit]
		$\nu(\Omega) < \infty$: \\ $\nu << \mu \iff \forall\epsilon>0\exists\delta>0\forall A\in\mathcal{A}: \mu(A)\le\delta\implies\nu(A)\le\epsilon$.
	\end{satz}
	\begin{lemma}[von Borel-Cantelli]
		\begin{enumerate}[label=(\alph*)]
			\item $\sum_{n=1}^{\infty}\P(A_n)<\infty\implies \P(\limsup_{n\to\infty}A_n)=0$.
			\item $(A_n)$ unabh: $\sum_{n=1}^{\infty}\P(A_n)=\infty\implies \P(\limsup_{n\to\infty}A_n)=1$.
		\end{enumerate}
	\end{lemma}
	\begin{lemma}[Ungleichung aus Lindeberg-Feller]
		$z_i, w_i \in \mathbb{D} \implies \abs{\prod_{j=1}^m z_j - \prod_{j=1}^m w_j} \leq \sum_{j=1}^m \abs{z_j - w_j}$.
	\end{lemma}
	\sep
	\begin{satz}[Charakterisierungen]
		\begin{enumerate}[label=(\alph*)]
			\item $X_n \fsk X \iff \forall\varepsilon>0:\lim\limits_{n\to\infty}\P(\set{\sup_{k\geq n}\abs{X_k-X}>\varepsilon}) = 0$,
			\item $X_n \fsk X \impliedby\forall \varepsilon>0:\sum_{n=1}^\infty\P(\abs{X_n-X}>\varepsilon)<\infty$,\\
			\item $X_n \sk X \iff \forall (X_{n_k})~\exists (X_{n_{k_j}}): X_{n_{k_j}}\fsk X$.
		\end{enumerate}
	\end{satz}
	\begin{satz}[Rechenregeln]
		$h$ stetig, $a_n\to a$, $X_n\sk X$, $Y_n\sk Y$.
		{\setlength{\columnseprule}{0pt}\begin{multicols}{2}
		\begin{enumerate}[label=(\alph*)]
			\item $h(X_n)\sk h(X)$
			\item $a_nX_n\sk aX$
			\item $aX_n+bY_n\sk aX+bY$
			\item $X_nY_n\sk XY$
			\item $\abs{X_n}\sk\abs{X}$
		\end{enumerate}
		\end{multicols}}
	\end{satz}
	\begin{lemma}[von Sluzki]
		$X_n\vk X$, $Y_n\sk a$.
		{\setlength{\columnseprule}{0pt}\begin{multicols}{2}
		\begin{enumerate}[label=(\alph*)]
			\item $X_n+Y_n\vk X+a$,
			\item $X_nY_n\vk aX$.
		\end{enumerate}
		\end{multicols}}
	\end{lemma}
	\begin{satz}[Continuous mapping theorem]
		$h\colon\R\to\R$ mb, $\P^X$-f.ü. stetig, $X_n\vk X$. Dann $h(X_n)\vk h(X)$.
	\end{satz}
	\begin{definition}[Straffheit]
		$\varnothing\neq\mathcal{Q}$ Menge von W-Maßen auf $\mathcal{B}$ heißt straff, falls
		\[\forall\varepsilon>0~\exists K\subset\R~\text{kompakt}~\forall Q\in\mathcal{Q}:Q(K)\geq 1-\varepsilon.\]
	\end{definition}
	\begin{definition}[Relative Kompaktheit]
		$\varnothing\neq\mathcal{Q}$ Menge von W-Maßen auf $\mathcal{B}$ heißt relativ kompakt, falls
		\[\forall (Q_n)~\text{in $\mathcal{Q}$}~\exists (Q_{n_k}), Q: Q_{n_k}\vk Q.\]
		Es wird \textit{nicht} $Q\in\mathcal{Q}$ gefordert.
	\end{definition}
	\begin{satz}
		$\mathcal{Q}~\text{straff}\iff\mathcal{Q}~\text{relativ kompakt}$.
	\end{satz}
	\begin{satz}[Gilt auch für höhere Momente]
		$\forall n: \E \abs{X_n} < c < \infty \implies (X_n)$ straff.
	\end{satz}
	\begin{satz}[Straffheit und Verteilungskonvergenz]
		\begin{enumerate}[label=(\alph*)]
			\item $X_n\vk D\implies\set{P^{X_n}\given n\in\N}~\text{straff}$.
			\item $((X_n)$ straff $\land$ $\exists Q~\forall (X_{n_k}): X_{n_k}\vk Q)\implies X_n\vk Q$.
		\end{enumerate}
	\end{satz}
	\begin{satz}[Lévy-Cramér]
		$X_i$ ZV mit VF $F_i$ und CF $\cf_i$. Es sind äquivalent:
		\begin{enumerate}[label=(\alph*)]
			\item $\exists F: F_n\vk F$.
			\item $\exists \cf: \cf_n\to\cf~\text{pktw.}$ und $\cf$ ist stetig in $0$.
		\end{enumerate}
		In diesem Fall: $\cf(t)=\smallint_\R\e^{\i tx}\dd{F(x)}$.
	\end{satz}
	\sep
	\begin{satz}[Starkes Gesetz großer Zahlen]
		$(X_n)$ u.i.v.-Folge.
		\[
			\E\abs{X_1}<\infty\iff
			\exists X:\frac{1}{n}\sum_{j=1}^nX_j\fsk X
		\]
		In diesem Fall $X=\E X_1 \fs{\P}$
	\end{satz}
	\begin{satz}[Kolmogorov-Kriterium]
		$(X_n)$ unabhängig, $\forall n:\E X_n^2<\infty$, $\exists (a_n): 0\le a_n\uparrow\infty, \displaystyle\sum_{n=1}^\infty \frac{\V X_n}{a_n^2}<\infty$.
		Dann
		\vspace{-1.5em}
		\[\frac{1}{a_n}\sum_{j=1}^n(X_j-\E X_j)\fsk 0.\]
	\end{satz}
	\begin{satz}[Gleichmäßig beschränkte Varianzen]
		$(X_n)$ unabgängig, $\exists c\in[0,\infty):\sup_{n\in\N}\V X_n\leq c$. Dann
		\[\frac{1}{n}\sum_{j=1}^n(X_j-\E X_j)\fsk 0.\]
	\end{satz}
	\sep
	\begin{definition}[Charakteristische Funktion]
		$
			\cf_X(t)\defas \E(\e^{\i tX})
		$
	\end{definition}
	\begin{satz}[Eigenschaften]
		{\setlength{\columnseprule}{0pt}\begin{multicols}{2}
		\begin{enumerate}[label=(\alph*)]
			\item $\cf_X(0)=1$, $\abs{\cf_X(t)}\leq 1$,
			\item $\cf_X$ glm stetig,
			\item $\cf_X(-t) = \conj{\cf_X(t)}$,
			\item $\cf_{aX+b}(t) = \e^{\i tb}\cf_X(at)$.
		\end{enumerate}
		\end{multicols}}
	\end{satz}
	\begin{satz}[Faltung]
		$X_1,\ldots,X_n$ unabhängig: $\cf_{X_1+\cdots+X_n}(t)=\prod_{j=1}^n\cf_{X_j}(t)$.
	\end{satz}
	\begin{satz}[Umkehrformel]
		Ist $\smallint_\R\abs{\cf_X(t)}\dd{t}<\infty$, so hat $X$ die
		$\lambda^1$-Dichte \[f(x) = \frac{1}{2\pi}\smallint_\R\e^{-\i tx}\cf_X(t)\dd{t}.\]
	\end{satz}
	\begin{satz}[Folgerungen]
		\begin{enumerate}[label=(\alph*)]
			\item $\P^X=\P^Y\iff\cf_X=\cf_Y$.
			\item $X\sim -X\iff \forall t\in\R:\cf_X(t)\in\R$.
		\end{enumerate}
	\end{satz}
	\begin{satz}[Momente]
		Sei $X$ ZV. $\E \abs{X}^r < \infty \implies \cf_X^{(r)}(0) = \i^r \E X^r$.
		
		$\abs{\cf_X - \sum_{r=0}^k{\frac{(it)^r}{r!}\E X^r}} \le \E\left[ \frac{2\abs{tX}^k}{k!} \land \frac{2\abs{tX}^{k+1}}{(k+1)!} \right]$
	\end{satz}
	\sep
	\begin{satz}[Lindeberg-Lévy]
		$(X_n)$ u.i.v, $\E X_1^2<\infty$, $\sigma^2\defas \V X_1>0$, $a\defas \E X_1$.
		\[
			\frac{\sum_{j=1}^nX_j - na}{\sigma\sqrt{n}}\vk \Nd(0, 1).
		\]
	\end{satz}
	\begin{definition}[Dreiecksschema]
		$\set{X_{nj}\given n\in\N, j=1,\ldots,k_n}$. $X_{n1},\ldots,X_{nk_n}$ auf dem
		gleichen Raum und unabhängig. $0<\sigma_{nj}^2\defas \V X_{nj}<\infty$.
		$a_{nj}\defas \E X_{nj}$, $\sigma_n^2\defas\sum_{i=1}^{k_n}\sigma_{ni}^2$.
		$S_n\defas\sum_{i=1}^{k_n}X_{ni}\implies \V S_n=\sigma_n^2$.
		$Y_{nj}\defas\frac{X_{nj}-a_{nj}}{\sigma_n}\implies \E Y_{nj}=0$.
		$S_n^*\defas\frac{S_n-\E S_n}{\sqrt{\V S_n}} = \sum_{j=1}^{k_n} Y_{nj}$.
		$\tau_{nj}^2\defas \V Y_{nj} = \frac{\V X_{nj}}{\sigma_n^2}=\frac{\sigma_{nj}^2}{\sigma_n^2}
		\implies \sum_{j=1}^{k_n}\tau_{nj}^2=1$.
	\end{definition}
	\begin{satz}[Lindeberg-Feller]
		Dreiecksschema. Lindeberg-Bedingung:
		\[
			\forall \varepsilon>0:L_n(\varepsilon)\defas \sum_{j=1}^{k_n}\E[Y_{nj}^2\ind\set{\abs{Y_{nj}}\geq\varepsilon}]\to 0.
		\]
		Dann $S_n^*\vk\Nd(0, 1)$.
	\end{satz}
	\begin{satz}[Notwendig für die Lindeberg-Bedingung]
		Feller-Bedingung:
		\[
			\lim_{n\to\infty}\frac{\max_{1\leq j\leq k_n}\sigma_{nj}^2}{\sigma_{n1}^2+\cdots+\sigma_{nk_n}^2}.
		\]
	\end{satz}
	\begin{satz}[Ljapunov]
		Dreiecksschema. $\delta>0$ (man wählt fast immer $\delta\defas 2$) mit
		\[
			\lim_{n\to\infty}\frac{1}{\sigma_n^{2+\delta}}\sum_{j=1}^{k_n}\E\left[\abs{X_{nj}-a_{nj}}^{2+\delta}\right]=0.
		\]
		Dann $S_n^*\vk\Nd(0, 1)$.
	\end{satz}
	\sep
	\begin{definition}[Übergangswahrscheinlichkeit]
		$\P_{1, 2}\colon\Omega_1\times\mathcal{A}_2\to\R$ mit
		\begin{enumerate}[label=(\alph*)]
			\item $\forall \omega_1\in\Omega_1:\P_{1,2}(w_1,\cdot)~\text{ist W-Maß}$.
			\item $\forall A_2\in\mathcal{S}_2:\P_{1,2}(\cdot,A_2)~\text{$(\mathcal{A}_1,\mathcal{B})$-mb}$.
		\end{enumerate}
	\end{definition}
	\begin{defsatz}[Kopplung]
		\begin{align*}
			\P_1\otimes\P_{1,2}(A)\defas&\smallint_{\Omega_1}\smallint_{\Omega_2}\ind_A\P_{1,2}(\omega_1,\dd{\omega_2})\P_1(\dd{\omega_1})\\
			\P_1\otimes\P_{1,2}(A_1\times A_2) = &\smallint_{A_1}\P_{1,2}(\omega_1, A_2)\P_1(\dd{\omega_1})
		\end{align*}
	\end{defsatz}
	\begin{satz}[Fubini]
		$f\colon\Omega_1\times\Omega_2\to\overline\R$ $\mathcal{A}_1\otimes\mathcal{A}_2$-mb; nichtnegativ
		oder $\P_1\otimes\P_{1,2}$-ib.
		\[
			\smallint_{\Omega_1\times\Omega_2}f\dd{\P_1\otimes\P_{1,2}} = \smallint_{\Omega_1}\smallint_{\Omega_2}f\P_{1,2}(\omega_1,\dd{\omega_2})\P_1(\dd{\omega_1})
		\]
	\end{satz}
	\begin{satz}[Dichten]
		$\mu_1$ $\sigma$-endlich, $\P_1=g_1\mu_1$, $\mu_2$ $\sigma$-endlich,
		$g_{1,2}\colon\Omega_1\times\Omega_2\to\R_{\geq0}$ $\mathcal{A}_1\otimes\mathcal{A}_2$-mb,
		$\forall\omega_1\in\Omega_1$:
		\[
			\smallint_{\Omega_1}g_{1, 2}(\omega_1, \omega_2)\mu_2(\dd{\omega_2})=1.
		\]
		ÜW:
		\begin{align*}
			\P_{1,2}(\omega_1, A_2)\defas&\smallint_{A_2}g_{1,2}(\omega_1,\omega_2)\mu_2(\dd{\omega_2})\\
		\end{align*}
	\end{satz}
	\begin{defsatz}[Zerlegung]
		ZV $Z, X$. Dann existiert ÜW $\P^X_Z$ mit $\P^{(Z, X)}=\P^Z\otimes\P^X_Z$.
		$\P^X_Z$ ist eindeutig bis auf $\P^Z$-Nullmenge.
		$\P^Z_{Z=z}(\cdot)\defas\P^X_Z(z, \cdot)$ heißt bedingte Verteilung von $X$ unter $Z=z$.
	\end{defsatz}
	\begin{satz}[Bedingte Verteilung bei gemeinsamer Dichte]
		$\mu, \nu$ $\sigma$-endlich, $\P^{(Z, X)}=f(\mu\otimes\nu)$, $f_1(z)\defas\smallint_{\R^\ell}f(z, x)\nu(\dd{x})$,
		$g_0\colon\R^\ell\to\R_{\geq0}$ mb, $\smallint_{\R^\ell}g_0\dd{\nu}=1$.
		\[
			f(x|z)\defas\begin{cases*}
				\frac{f(z, x)}{f_1(z)}, &$f_1(z)>0$,\\
				g_0(x), &sonst.
			\end{cases*}
		\]
		Dann ist $\P^X_{Z=z}(B)\defas\smallint_Bf(x|z)\nu(\dd{x})$ eine bedingte Verteilung.
	\end{satz}
	\sep
	\begin{definition}[Bedingte Erwartung]
		$\BE{X}{G}\defas Y$ ist definiert durch:

		$\E\abs{Y}<\infty$, $Y$ mb, $\forall A\in\mathcal{G}:\E(Y\ind_A)=\E(X\ind_A)$.
	\end{definition}
	\begin{satz}[Eigenschaften]
		\begin{enumerate}[label=(\alph*)]
			\item $\BE{X}{G}$ existiert und ist $\P$-fast sicher eindeutig,
			\item $\E(\BE{X}{G})=\E X$,
			\item $Y$ $\mathcal{G}$-mb: $\BE{XY}{G}=Y\BE{X}{G}$,
			\item $\BE{aX+bY}{G}=a\BE{X}{G}+b\BE{Y}{G}$,
			\item $X\leq Y\fs{\P}\implies\BE{X}{G}\leq\BE{Y}{G}$,
			\item $\mathcal{F}\subset\mathcal{G}\implies \BE{X}{F}=\BE{\BE{X}{F}}{G}=\BE{\BE{X}{G}}{F}$,
			\item $\abs{\BE{X}{G}}\leq\BE{\abs{X}}{G}$,
			\item $X, G~\text{unabhängig}\implies\BE{X}{G}=\E X$.
		\end{enumerate}
	\end{satz}
	\begin{satz}[Konvergenz]
		$X, X_1,\ldots$ ib.
		\begin{enumerate}[label=(\alph*)]
			\item $0\leq X_n\uparrow X$ oder
			\item $X_n\fsk X$, $\abs{X_n}\leq Y\fs{\P}$, $\E Y<\infty$.
		\end{enumerate}
		Dann $\BE{X_n}{G}\to\BE{X}{G}\fs{\P}$.
	\end{satz}
	\begin{satz}[Jensen]
		$g\colon\R\to\R$ konvex, $\E X<\infty$, $\E g(X)<\infty$: $\BE{g(X)}{G}\geq g(\BE{X}{G})$.
	\end{satz}
	\begin{satz}[Unabhängige $\sigma$-Algebra]
		$\mathcal{H}$ und $\sigma(\sigma(X)\cup\mathcal{G})$ unabh: $\BE{X}{G}=\EV{X}{\sigma(\mathcal{G}\cup\mathcal{H})}$.
	\end{satz}
	\begin{satz}[Faktorisierung]
		$\EV{X}{Z=z}\defas h$ mit $\EV{X}{Z}=h\circ Z$ heißt bedingter Erwartungswert von
		$X$ unter $Z=z$. $h$ wird durch
		\[
			\forall A'\in\mathcal{A}': \smallint_{A'}\EV{X}{Z=z}\P^Z(\dd{z})=\smallint_{Z^{-1}(A')}X\dd{\P}
		\]
		$\P$-fast sicher charakterisiert.
	\end{satz}
	\begin{definition}[Bedingte Wahrscheinlichkeit]
		$\P(A|Z)\defas\P(A|\sigma(Z))\defas\EV{\ind_A}{Z}$. $\P(A|Z=z)\defas\EV{\ind_A}{Z=z}$.
		\begin{align*}
			\P(A\cap\set{Z\in A'}) &= \smallint_{A'}\P(A|Z=z)\P^Z(\dd{z})\\
			\P(A) &= \smallint_{\Omega'}\P(A|Z=z)\P^Z(\dd{z})
		\end{align*}
	\end{definition}
	\begin{satz}[Bedingter Erwartungswert über bedingte Verteilung]
		$Z$ ZV, $X$ reellwertig ib. Dann
		\[
			\EV{X}{Z=z} = \smallint_\R x\P^X_{Z=z}(\dd{x}).
		\]
	\end{satz}
	\begin{satz}[Dichten]
		$\mu, \nu$ $\sigma$-endlich, $X$ reellwertig ib, $\P^{(Z, X)}=f(\mu\otimes\nu)$.
		Dann
		\[
			\EV{X}{Z=z}=\smallint_\R xf(x|z)\nu(\dd{x}).
		\]
	\end{satz}
	\begin{satz}
		$f\colon\R^{k+\ell}\to\overline\R$ mb, $\E\abs{f(Z, X)}<\infty$. Dann
		\[
			\E f(Z, X) = \smallint_{\R^k}\EV{f(Z, X)}{Z=z}\P^Z(\dd{z})
		\]
		mit
		\[
			\EV{f(Z, X)}{Z=z}\defas \smallint_{\R^\ell}f(z, x)\P^X_{Z=z}(\dd{x}).
		\]
	\end{satz}
	\sep
	\begin{definition}
		\begin{enumerate}[label=(\alph*)]
			\item Aufsteigende Folge von Sub-$\sigma$-Algebren heißt Filtration.
			\item $\tau\colon\Omega\to\N_0\cup\set\infty$ heißt Stoppzeit bezüglich
				$\mathbb{F}=(\F_n)_{n\geq0}$, falls $\forall n\in\N_0:\set{\tau=n}\in\F_n$.
			\item Stoppzeit mit $\P(\tau<\infty)=1$ heißt endlich.
			\item $(X_n)_{n\in\N_0}$ heißt adaptiert, falls $X_n$ $\F_n$-messbar.
			\item $\F_n^X\defas\sigma(X_0,\ldots,X_n)$ heißt natürliche Filtration.
		\end{enumerate}
	\end{definition}
	\begin{satz}[Charakterisierung]
		$\tau$ ist Stoppzeit genau dann wenn $\forall n\in\N_0:\set{\tau\leq n}\in\F_n$.
	\end{satz}
	\begin{definition}[$\sigma$-Algebra der $\tau$-Vergangenheit]
		$\mathcal{A}_\tau\defas\set{A\in\mathcal{A}\given \forall n\in\N_0:A\cap\set{\tau\leq n}\in\F_n}$.
	\end{definition}
	\begin{defsatz}
		$\tau$ endlich, $(X_n)$ adaptiert. $X_\tau\colon\Omega\to\R$,
		\[
			\omega\mapsto\begin{cases}
				X_{\tau(\omega)}(\omega), &\tau(\omega)<\infty,\\
				0, &\tau(\omega)=\infty
			\end{cases}
		\]
		ist $\mathcal{A}_\tau$-messbar.
	\end{defsatz}
	\begin{definition}
		$(X_n)$ adaptiert, $\P$-integrierbar (!) heißt
		\begin{enumerate}[label=(\alph*)]
			\item Submartingal, falls $\forall n\in\N_0:\EV{X_{n+1}}{\F_n}\geq X_n\fs{\P}$,
			\item Supermartingal, falls $\forall n\in\N_0:\EV{X_{n+1}}{\F_n}\leq X_n\fs{\P}$,
			\item Martingal, falls $\forall n\in\N_0:\EV{X_{n+1}}{\F_n} = X_n\fs{\P}$.
		\end{enumerate}
	\end{definition}
	\begin{satz}[Entsprechend für Supermartingale und Martingale]
		$(X_n)$ Submartingal. $\EV{X_m}{\F_n}\geq X_n\fs\P$, $\E X_n\leq \E X_{n+1}$.
	\end{satz}
	\begin{satz}
		$(X_n)$ Supermartinal oder Submartingal. $(X_n)$ ist Martinal genau dann wenn
		$\forall n\geq 1:\E X_n = \E X_0$.
	\end{satz}
	\begin{defsatz}[Doobsches Martingal]
		$X$ ZV, $(\F_n)$ Filtration. $X_n\defas\EV{X}{\F_n}$ ist Martingal.
	\end{defsatz}
	\begin{definition}[Prävisibel]
		$(V_n)$ heißt prävisibel/vorhersagbar, falls $V_n$ $\F_{n-1}$-messbar,
		wobei $\F_{-1}\defas\set{\varnothing,\Omega}$.
	\end{definition}
	\begin{satz}
		$(X_n)$ vorhersagbares Martingal, dann $X_n=X_0\fs\P$.
	\end{satz}
	\begin{satz}[Doob-Zerlegung]
		$(X_n)$ adaptiert, $\P$-integrierbar. Dann existiert eindeutige Zerlegung
		$X_n=M_n+V_n$, $(M_n)$ Martingal, $(V_n)$ vorhersagbar, $V_0=0$.
		$(X_n)$ ist genau dann Submartingal, wenn $(V_n)$ $\fs\P$ monoton wachsend.
	\end{satz}
	\begin{satz}[Orthogonale Zuwächse]
		$(X_n)$ Martingal, $\E X_n^2<\infty$.
		\begin{enumerate}[label=(\alph*)]
			\item $\forall \ell\neq m: \E[(X_m-X_{m-1})(X_\ell - X_{\ell - 1})] = 0$.
			\item $\V X_n =\V X_0 + \sum_{j=1}^n\E[(X_j - X_{j-1})^2]$.
		\end{enumerate}
	\end{satz}
	\begin{satz}
		$(X_n)$ Martingal, $g\colon\R\to\R$ konvex, $\E\abs{g(X_n)}<\infty$. Dann
		ist $(g(X_n))$ ein Submartingal.
	\end{satz}
	\begin{satz}[Spielsystem]
		$(X_n)$ Martingal, $(C_n)$ prävisibel. $(C\bullet X)_n\defas \sum_{k=1}^n C_k(X_k-X_{k-1})$
		heißt Martingaltransformation. Falls $\E[C_n(X_n-X_{n-1})] <\infty$, ist $C\bullet X$ ein
		Martingal. Ist $C_n\geq 0$, so bleibt auch Sub-/Supermartingal erhalten.
	\end{satz}
	\begin{satz}
		$(X_n)$ Martingal, $\tau$ Stoppzeit. $X_{\tau\wedge n}(\omega)\defas X_{\tau(\omega)\wedge n}(\omega)$
		ist Martingal. Entsprechend für Sub-/Supermartingale.
	\end{satz}
	\begin{satz}[Optional stopping theorem]
		$(X_n)$ Martingal, $\tau$ Stoppzeit bzgl. natürlicher Filtration. $\E\tau<\infty$.
		Es existiere $c\in(0,\infty)$ mit $\forall n\geq 1$:
		\[
			\EV{\ind\set{\tau\geq n}\cdot\abs{X_n-X_{n-1}}}{X_0,\ldots,X_{n-1}}\leq c\ind\set{\tau\geq n}\fs\P.
		\]
		Dann $\E X_\tau = \E X_0$. Ist $(X_n)$ stattdessen Sub/Supermartingal, so
		gilt \enquote{$\geq$} bzw. \enquote{$\leq$}.
	\end{satz}
	\begin{satz}[Waldsche Gleichung]
		$(X_n)$ u.i.v., $\E\abs{X_1}<\infty$, $N$ Stoppzeit bzgl. natürlicher Filtration,
		$\E N<\infty$. Dann $\E(\sum_{j=1}^N X_j) = \E X_1\cdot \E N$.
	\end{satz}
\end{multicols}
\vspace{0.2cm}

\begin{minipage}{11.35cm}
	\begin{defsatz}
		\begin{align*}
			\Gamma(z)\defas&\smallint_0^\infty x^{z-1}e^{-x}\dd{x} & \Bd(x, y)\defas&\smallint_0^1 t^{x-1}(1-t)^{y-1}\dd{t}\\
			\Gamma(z+1)=&z\Gamma(z) & \Bd(x, y) =& \frac{\Gamma(x)\Gamma(y)}{\Gamma(x+y)}
		\end{align*}
	\end{defsatz}

	\tiny
	\begin{tabular}{ | M{1.2cm} | M{2.2cm} | M{1cm} | M{2.7cm} | M{2cm} | N}
		\hline
		Name & $\P(X=k)$ & $\E X$ & $\V X$ & $X_1 + X_2$ & \\ \hline \hline
		$\Bin(n, p)$ & $\binom{n}{k}p^k(1-p)^{n-k}$ & $np$ & $np(1-p)$ & $\Bin(n_1+n_2, p)$ & \\ \hline
		$\G(p)$ & $(1-p)^kp$ & $\frac{1-p}{p}$ & $\frac{1-p}{p^2}$ & $\Nb(2, p)$ & \\ \hline
		$\Nb(r, p)$ & $\binom{k+r-1}{k}p^r(1-p)^k$ & $r\frac{1-p}{p}$ & $r\frac{1-p}{p^2}$ & $\Nb(r_1 + r_2, p)$ & \\ \hline
		$\Hyp(n, r, s)$ & $\frac{\binom{r}{k}\binom{s}{n-k}}{\binom{r+s}{n}}$ & $n\frac{r}{r+s}$ & $np(1-p)\left(1-\frac{n-1}{r+s-1}\right)$ & & \\ \hline
		$\Po(\lambda)$ & $e^{-\lambda}\frac{\lambda^k}{k!}$ & $\lambda$ & $\lambda$ & $\Po(\lambda_1 + \lambda_2)$ & \\ \hline
	\end{tabular}
\end{minipage}
\tiny
\begin{tabular}{ | M{1.2cm} | M{4.3cm} | M{2.2cm} | M{1.6cm} | M{1.6cm} | M{3.2cm} | N}
	\hline
	Name & $f(x)$ & $F(x)$ & $\E X$ & $\E X^2$ & $\V X$ & \\ \hline \hline
	$\U(a, b)$ & $\begin{cases}\frac{1}{b-a}, &x\in(a, b)\\0, &\text{sonst}\end{cases}$ &
		$\begin{cases}0,&x\leq a\\\frac{x-a}{b-a}, &x\in(a, b)\\1,&x\geq b\end{cases}$ &
		$\frac{a+b}{2}$ & $\frac{a^2+ab+b^2}{3}$ & $\frac{(a-b)^2}{12}$ & \\ \hline
	$\Exp(\lambda)$ & $\lambda e^{-\lambda x}, x>0$ &
		$\begin{cases}1-e^{-\lambda x},&x>0\\0,&x\leq 0\end{cases}$ & $\frac{1}{\lambda}$ & $\frac{2}{\lambda^2}$
		& $\frac{1}{\lambda^2}$ & \\\hline
	$\Nd(\mu, \sigma^2)$ & $\frac{1}{\sqrt{2\pi}\sigma}\exp(-\frac{(x-\mu)^2}{2\sigma^2})$ & $\Phi(\frac{x-\mu}{\sigma})$
		& $\mu$ & $\mu^2+\sigma^2$ & $\sigma^2$ & \\ \hline
	$\Gd(\alpha, \beta)$ & $\frac{\beta^\alpha}{\Gamma(\alpha)}x^{\alpha - 1}\exp(-\beta x), x>0$
		& & $\frac{\alpha}{\beta}$ & $\frac{\Gamma(\alpha+2)}{\Gamma(\alpha)\beta^\alpha}$ & $\frac{\alpha}{\beta^2}$ &\\ \hline
	$\Bd(\alpha, \beta)$ & $\frac{1}{B(\alpha,\beta)}x^{\alpha-1}(1-x)^{\beta-1},x\in(0,1)$
		& & $\frac{\alpha}{\alpha+\beta}$ & $\frac{B(2+\alpha,\beta)}{B(\alpha, \beta)}$
		& $\frac{\alpha\beta}{(\alpha+\beta)(\alpha+\beta+1)}$ & \\ \hline
	$\chi_k^2$ & $\mathds{1}_{(0, \infty)}(x)\frac{1}{2^{\frac{k}{2}}\Gamma(\frac{k}{2})}e^{-\frac{t}{2}}x^{\frac{k}{2}-1}$
		& & $k$ & $k^2+2k$ & $2k$ & \\ \hline
	$\LN(\mu,\sigma^2)$ & $\mathds{1}_{(0,\infty)}(x)\frac{1}{\sigma x\sqrt{2\pi}}\exp(-\frac{(\log(x)-\mu)^2}{2\sigma^2})$
		& $\Phi(\frac{\log(x)-\mu}{\sigma})$ & $\exp(\mu+\frac{\sigma^2}{2})$ & $\exp(2(\mu+\sigma^2))$
		& $\exp(2\mu+\sigma^2)(\exp(\sigma^2)-1)$ & \\ \hline
	$\Wd(\alpha, \beta)$ & $\mathds{1}_{(0,\infty)}(x)\alpha\beta x^{\beta-1} \exp(-\alpha x^\beta)$
		& $1-\exp(-\alpha x^\beta)$ & $\frac{\Gamma(1+\frac{1}{\beta})}{\alpha^{\frac{1}{\beta}}}$
		& & & \\ \hline
	$\Cd$ & $\frac{1}{\pi}\frac{1}{1+x^2}$ & $\frac{1}{\pi}(\arctan(x)-\frac{\pi}{2})$ & $\infty$ & & & \\ \hline
\end{tabular}
\end{document}

\documentclass[a4paper]{article}

\usepackage[l2tabu, orthodox]{nag}

\usepackage[utf8]{inputenc}
\usepackage[T1]{fontenc}

\usepackage[ngerman]{babel}

\usepackage{amsmath}
\usepackage{amssymb}
%\usepackage{amsthm}
\usepackage{mathtools}
\usepackage{physics}

\usepackage[framed]{ntheorem}

\usepackage{csquotes}
\usepackage{lmodern}
\usepackage{microtype}
\usepackage{enumitem}
\usepackage{stmaryrd}

\usepackage{parskip}
\usepackage{multicol}

\usepackage{array}
\usepackage{blindtext}
\usepackage{float}

\usepackage[left=1.8cm, right=1.8cm, top=1.8cm, bottom=2.5cm]{geometry}

\newcounter{Sec}

\theoremstyle{marginbreak}
\theorembodyfont{\normalfont}
\newtheorem{definition}{Definition}[Sec]
\newtheorem{satz}[definition]{Satz}
\newtheorem{defsatz}[definition]{Definition und Satz}
\newtheorem{verfahren}[definition]{Verfahren}
\newtheorem{defver}[definition]{Definition und Verfahren}
\newtheorem{defsatzver}[definition]{Definition, Satz und Verfahren}
\newtheorem{satzver}[definition]{Satz und Verfahren}

\MakeOuterQuote{"}
\DeclareMathOperator{\ffa}{ffa}

\newcommand{\sep}{%
	\rule{\textwidth}{0.3pt}%
	\stepcounter{Sec}%
	}
\newcommand{\defiff}{:\Longleftrightarrow}

\newcommand{\en}{~(n\to\infty)}
\newcommand{\series}[1][1]{\sum_{n=#1}^\infty}
\newcommand{\ps}[1][a]{\series[0]#1_n(x-x_0)^n}
\renewcommand{\d}[1]{\mathrm{d}#1}

\newcolumntype{M}[1]{>{\centering\arraybackslash}m{#1}}
\newcolumntype{N}{@{}m{0pt}@{}}

\setlength\columnsep{1.5cm}

\DeclareMathOperator{\arsinh}{arsinh}
\DeclareMathOperator{\arcosh}{arcosh}
\DeclareMathOperator{\artanh}{artanh}

\DeclarePairedDelimiterX\set[1]\lbrace\rbrace{\def\given{\;\delimsize\vert\;}#1}
\DeclarePairedDelimiter\ceil{\lceil}{\rceil}

\begin{document}
	\textsc{Algorithmen}

	\sep

	\begin{definition}[Landau-Notation]
		\begin{align*}
			\mathcal{O}(f(n)) &\coloneqq \set{g(n)\given\exists c>0~\exists n_0\in\mathbb{N}_{+}~\forall n\geq n_0:g(n)\leq c\cdot f(n)}\\
			\Omega(f(n)) &\coloneqq \set{g(n)\given\exists c>0~\exists n_0\in\mathbb{N}_{+}~\forall n\geq n_0:g(n)\geq c\cdot f(n)}\\
			\Theta(f(n)) &\coloneqq \mathcal{O}(f(n)) \cap \Omega(f(n))\\
			o(f(n)) &\coloneqq \set{g(n)\given\forall c>0~\exists n_0\in\mathbb{N}_{+}~\forall n\geq n_0:g(n)\leq c\cdot f(n)}\\
			\omega(f(n)) &\coloneqq \set{g(n)\given\forall c>0~\exists n_0\in\mathbb{N}_{+}~\forall n\geq n_0:g(n)\geq c\cdot f(n)}\\
		\end{align*}
	\end{definition}

	\begin{satz}[Äquivalenzen]
		\begin{align*}
			g(n)\in\mathcal{O}(f(n)) &\iff 0\leq\limsup_{n\to\infty}\frac{g(n)}{f(n)}=c<\infty\\
			g(n)\in\Omega(f(n)) &\iff 0<\liminf_{n\to\infty}\frac{g(n)}{f(n)}\leq\infty\\
			g(n)\in\Theta(f(n)) &\impliedby 0<\lim_{n\to\infty}\frac{g(n)}{f(n)}=c<\infty\\
			g(n)\in o(f(n)) &\iff \lim_{n\to\infty}\frac{g(n)}{f(n)}=0\\
			g(n)\in\omega(f(n)) &\iff \limsup_{n\to\infty}\frac{g(n)}{f(n)}=\infty
		\end{align*}
	\end{satz}

	\begin{satz}[Rechenregeln]
		\begin{enumerate}[label=(\alph*)]
			\item $\forall c>0:c\cdot f(n)\in\Theta(f(n))$
			\item $\sum_{i=0}^k a_in^i\in\mathcal{O}(n^k)$
			\item $f(n)+g(n)\in\Omega(f(n))$
			\item $g(n)\in\mathcal{O}(f(n))\implies f(n)+g(n)\in\mathcal{O}(f(n))$
			\item $\mathcal{O}(f(n))\cdot\mathcal{O}(g(n)) = \mathcal{O}(f(n)\cdot g(n))$
		\end{enumerate}
	\end{satz}

	\begin{satz}[Basiswechsel]
		\[
			\log_a b = \frac{\log_x b}{\log_x a}
		\]
	\end{satz}

	\sep

	\begin{satz}[Mastertheorem]
		Es seien $a, b, c, d \in\mathbb{R}$. Ferner sei
		\[
			r(n)\coloneqq
			\begin{cases}
				a, &n=1\\
				cn+dr(\ceil*{\frac{n}{b}}), &n>1.
			\end{cases}
		\]
		Dann
		\[
			r(n)\in
			\begin{cases}
				\Theta(n), &d<b\\
				\Theta(n\log n), &d=b\\
				\Theta(n^{\log_b d}), &d>b.
			\end{cases}
		\]
	\end{satz}

	\sep

	\begin{verfahren}[$(a,b)$-Bäume]
		Einfügen: Links vom nächstgrößeren.

		Split: Links klein, rechts groß.

		Löschen: Fuse wenn möglich, ansonsten balance.

		Balance: Rechter Nachbar außer wenn nicht existent.
	\end{verfahren}

\end{document}

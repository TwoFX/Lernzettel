\documentclass[a4paper]{article}

\usepackage[l2tabu, orthodox]{nag}

\usepackage[utf8]{inputenc}
\usepackage[T1]{fontenc}

\usepackage[ngerman]{babel}

\usepackage{amsmath}
\usepackage{amssymb}
%\usepackage{amsthm}
\usepackage{mathtools}
\usepackage{physics}

\usepackage[framed]{ntheorem}

\usepackage{csquotes}
\usepackage{lmodern}
\usepackage{microtype}
\usepackage{enumitem}
\usepackage{stmaryrd}

\usepackage{parskip}

\usepackage[left=2cm, right=2cm, top=2cm, bottom=2.5cm]{geometry}

\newcounter{Sec}

\theoremstyle{marginbreak}
\theorembodyfont{\normalfont}
\newtheorem{definition}{Definition}[Sec]
\newtheorem{satz}[definition]{Satz}
\newtheorem{defsatz}[definition]{Definition und Satz}
\newtheorem{verfahren}[definition]{Verfahren}
\newtheorem{defver}[definition]{Definition und Verfahren}

\MakeOuterQuote{"}

\DeclareMathOperator{\chop}{char}
\newcommand{\sep}{%
	\rule{\textwidth}{0.3pt}%
	\stepcounter{Sec}%
	}
\DeclarePairedDelimiterX\set[1]\lbrace\rbrace{\def\given{\;\delimsize\vert\;}#1}

\begin{document}

	\sep
	\begin{definition}[Gruppe]
		Tupel $(G, *)$ mit
		\begin{description}
			\item[G1 (assoziativ)] $\forall a, b, c \in G: (a * b) * c = a * (b * c)$
			\item[G2 (neutrales Element)] $\exists e \in G~\forall a \in G: e * a = a = a * e$
			\item[G3 (inverses Element)] $\forall a \in G~\exists a^{-1} \in G: a^{-1} * a = e = a * a^{-1}$
		\end{description}
	\end{definition}
	\begin{definition}[Abelsche Gruppe]
		\begin{description}
			\item[G4 (kommutativ)] $\forall a, b \in G: a * b = b * a$
		\end{description}
	\end{definition}
	\begin{satz}[Untergruppe]
		\begin{description}
			\item[UG1] $U \neq \varnothing$
			\item[UG2] $\forall a, b \in U: a * b^{-1} \in U$
		\end{description}
	\end{satz}
	\begin{definition}[Erzeugte Gruppe]
		Für $G$ Gruppe und $M \subset G$: $\langle M\rangle \coloneqq \text{Kleinste Untergruppe von $G$, die $M$ enthält}$
	\end{definition}
	\begin{definition}[Zyklische Gruppe]
		Für $a \in G$: $\langle a\rangle \coloneqq \langle\set{a}\rangle$
	\end{definition}
	\sep
	\begin{definition}[Symmetrische Gruppe]
		Menge der bijektiven Selbstabbildungen (Permutationen) einer endlichen Menge $M$.
		Verknüpfung ist die Verkettung $\circ$.
	\end{definition}
	\begin{definition}[Fehlstandszahl]
		Für endliche Menge $M$ und Permutation $\pi$:
		$F(\pi) \coloneqq \abs{\set{1 \leq i, j \leq \abs{M} \given i < j \wedge \pi(i) > \pi(j)}}$

		$\pi~\text{gerade} :\Longleftrightarrow F(\pi)~\text{gerade}$
	\end{definition}
	\begin{satz}[Anzahl Transpositionen]
		Für $\pi$ Permutation und $\tau_1, \ldots, \tau_n$ Transpositionen mit
		$\pi = \tau_1 \circ \ldots \circ \tau_n$: $\pi~\text{gerade} \iff n~\text{gerade}$.
	\end{satz}
	\sep
	\begin{definition}[Ring]
		Tupel $(R, +, \cdot)$ mit
		\begin{description}
			\item[R1] $(R, +)~\text{abelsch}$
			\item[R2] $\cdot~\text{assoziativ}$
			\item[R3] $\forall a, b, c \in R$ gilt: $a\cdot(b + c) = a\cdot b + a\cdot c$ und $(b + c)\cdot a = b\cdot a + c\cdot a$
		\end{description}

		Neutrales Element von $(R, +)$ heißt $0$, inverses Element zu $a \in R$ heißt $-a$, $b - a \coloneqq b + (-a)$.
	\end{definition}
	\begin{definition}[Kommutativer Ring]
		\begin{description}
			\item[R4] $\forall a, b \in R: a \cdot b = b \cdot a$
		\end{description}
	\end{definition}
	\begin{definition}[Ring mit Eins]
		\begin{description}
			\item[R5] $\exists 0 \neq 1 \in R~\forall a \in R: 1 \cdot a = a = a \cdot 1$
		\end{description}
	\end{definition}
	\begin{definition}[Nullteiler]
		Wenn $a, b \in R$, $a \neq 0 \neq b$ und $ab = 0$, dann ist $a$ linker Nullteiler
		und $b$ rechter Nullteiler von $R$.
	\end{definition}
	\sep
	\begin{definition}[Körper]
		$(\mathbb{K}, +, \cdot)~\text{Körper} :\Longleftrightarrow (\mathbb{K}, +, \cdot)~\text{Ring}
		\wedge (\mathbb{K}\setminus\set{0}, \cdot)~\text{abelsch}$
	\end{definition}
	\begin{satz}
		Körper haben keine Nullteiler.
	\end{satz}
	\begin{satz}
		$\mathbb{Z}/n\mathbb{Z}~\text{Körper} \iff n~\text{prim}$
	\end{satz}
	\begin{definition}[Charakteristik]
		\[\chop\mathbb{K} \coloneqq\begin{cases}
			m, &\text{falls ein kleinstes $m$ existiert, sodass $\underbrace{1 + \ldots + 1}_{\text{$m$ mal}} = 0$}\\
			0, & \text{falls kein solches $m$ existiert}
		\end{cases}\]
	\end{definition}
	\begin{satz}
		$\chop\mathbb{K}$ ist entweder 0 oder eine Primzahl.
	\end{satz}
	\sep
	\begin{definition}[Matrixmultiplikation]
		Für $A = (a_{ij}) \in \mathbb{K}^{p\cross q}$ und $B = (b_{ij}) \in \mathbb{K}^{q\cross r}$
		ist $AB = C = (c_{jk}) \in \mathbb{K}^{p\cross r}$ mit
		\[ c_{jk} \coloneqq \sum_{s = 1}^q a_{js}b_{sk}\]
	\end{definition}
	\begin{defsatz}[Allgemeine lineare Gruppe]
		Die Menge $\mathbf{GL}(n, \mathbb{K})$ der regulären/invertierbaren $n\cross n$-Matrizen ist eine Gruppe.
	\end{defsatz}
	\begin{verfahren}[Inverse Matrix berechnen]
		Um eine $n\cross n$-Matrix zu invertieren, setzt man die $n\cross n$-Einheitsmatrix daneben und wendet
		den Gauß-Algorithmus an.
	\end{verfahren}
	\sep
	\begin{definition}[Grad eines Polynoms]
		\[\deg f\coloneqq\begin{cases}
			n, &\text{falls $a_n \neq 0$ und $a_k = 0$ für alle $k < n$}\\
			-\infty,  &\text{falls $a_k = 0$ für alle $k \geq 0$}
		\end{cases}\]
	\end{definition}
	\begin{satz}
		$(\mathbb{K}[X], +, \cdot)$ ist ein kommutativer Ring mit Eins.
	\end{satz}
	\sep
	\begin{definition}[Vektorraum]
		Die Menge $V$ heißt mit den Abbildungen $+\colon V\cross V\to V$, $\cdot\colon\mathbb{K}\cross V\to V$ ein $\mathbb{K}$-Vektorraum, wenn
		\begin{description}
			\item[V1] $(V, +)$ abelsch
			\item[V2] Für alle $\lambda,\mu\in\mathbb{K}$ und $x, y\in V$ gilt:
				\begin{enumerate}[label=(\alph*)]
					\item $1 \cdot x = x$
					\item $\lambda \cdot (\mu \cdot x) = (\lambda \cdot \mu) \cdot x$
					\item $(\lambda + \mu) \cdot x = \lambda \cdot x + \mu \cdot x$
					\item $\lambda \cdot (x + y) = \lambda \cdot x + \lambda \cdot y$
				\end{enumerate}
		\end{description}
	\end{definition}
	\begin{definition}[Lineare Unabhängigkeit]
		Endlich viele Vektoren $v_1,\ldots,v_k \in V$ heißen linear unabhängig, wenn
		\[\sum_{i=1}^k\lambda_iv_i=0\implies\lambda_1=\ldots=\lambda_k=0\]
		Eine unendlich große Menge heißt linear unabhängig, wenn jede endliche Teilmenge
		linear unabhängig ist.
	\end{definition}
	\begin{satz}
		Eine Obermenge einer linear abhängigen Menge ist linear abhängig.
		Eine Teilmenge einer linear unabhängigen Menge ist linear unabhängig.
	\end{satz}
	\begin{satz}
		$k + 1$ Linearkombinationen von $k$ Vektoren sind linear abhängig.
	\end{satz}
	\begin{definition}[Lineare Hülle]
		Für $M \subset V$ ist $[M]$ die Menge der Linearkombinationen von Vektoren aus $M$.
		$[\varnothing] \coloneqq \set{0}$.
	\end{definition}
	\begin{definition}[Erzeugendensystem]
		$M \subset V~\text{erzeugend} :\Longleftrightarrow [M] = V$
	\end{definition}
	\sep
	\begin{defsatz}[Basis]%
		\begin{align*}%
			\text{$B \subset V$ ist Basis von $V$} :\Longleftrightarrow &~\text{$B$ ist erzeugend und linear unabhängig}\\
			\Longleftrightarrow &~\text{$B$ ist erzeugend und minimal}\\
			\Longleftrightarrow &~\text{$B$ ist linear unabhängig und maximal}
		\end{align*}
	\end{defsatz}
	\begin{satz}[Basisergänzungssatz]
		Für Vektorraum $V \neq \set{0}$, $E \subset V$ erzeugend und $L \subset E$ linear unabhängig
		existiert eine Basis $B$ von $V$ mit $L \subset B \subset E$.
	\end{satz}
	\begin{satz}
		Jeder Vektorraum hat eine Basis.
	\end{satz}
	\begin{defsatz}[Dimension]
		Ist $B \subset V$ eine Basis, dann $\dim V \coloneqq \abs{B}$. Die Dimension eines Vektorraums ist eindeutig.
	\end{defsatz}
	\begin{satz}
		Ist $\dim V = n$, dann:
		\begin{enumerate}[label=(\alph*)]
			\item $n + 1$ Vektoren aus $V$ sind linear abhängig.
			\item $n$ linear unabhängige Vektoren sind eine Basis von $V$.
		\end{enumerate}
	\end{satz}
	\sep
	\begin{definition}[Komponentenvektor]
		Ist $b_1, \ldots, b_n \in V$ eine geordnete Basis von $V$, dann hat $v \in V$ die
		eindeutige Basisdarstellung $v = \sum_{i = 1}^nv_ib_i$. Dann
		$\Theta_B(v) \coloneqq (v_1,\ldots,v_n) \in \mathbb{K}^n$.
	\end{definition}
	\begin{defver}[Basiswechsel]
		Für Basen $B, \bar{B}$ heißt die eindeutige Matrix $A$ mit der Eigenschaft
		$\Theta_{\bar{B}}(v) = A\cdot\Theta_B(v)$ für alle $v \in V$ die Übergangsmatrix von
		$B$ nach $\bar{B}$. Die $i$-te Spalte der Basiswechselmatrix sind die Komponenten
		des $i$-ten Basisvektors der Ursprungsbasis bezüglich der neuen Basis, also
		$b_i = a_{1i}\bar{b}_1 + a_{2i}\bar{b}_2 + \ldots + a_{ni}\bar{b}_n$.
	\end{defver}
	\begin{satz}
		Wenn $\Theta_{\bar{B}}(v) = A\cdot\Theta_B(v)$, dann
		$\Theta_B(v) = A^{-1}\cdot\Theta_{\bar{B}}(v)$.
	\end{satz}
	\sep
	\begin{satz}[Untervektorraum]
		\begin{description}
			\item[U1] $U\neq\varnothing$
			\item[U2] $\forall x, y \in U~\forall \lambda\in\mathbb{K}: x + y \in U \wedge \lambda x\in U$
		\end{description}
	\end{satz}
	\begin{satz}
		Für $M \subset V$ ist $[M]$ ein Untervektorraum von $V$.
	\end{satz}
	\begin{satz}
		Für $\varnothing\neq\mathfrak{U}\subset\mathcal{P}(V)$ ist $\bigcap_{U\in\mathfrak{U}}U$
		ein Untervektorraum von U.
	\end{satz}
	\begin{definition}[Summe und direkte Summe]
		$U_1 + U_2 \coloneqq [U_1 \cup U_2]$. Falls $U_1 \cap U_2 = \set{0}$, dann
		$U_1 \oplus U_2 \coloneqq U_1 + U_2$.
	\end{definition}
	\begin{satz}
		\begin{enumerate}[label=(\alph*)]
			\item $U_1 + U_2 = \set{v \in V \given \exists u_1 \in U_1, u_2 \in U_2: v = u_1 + u_2}$
			\item Die Summe ist genau dann direkt, wenn die Wahl von $u_1$ und $u_2$ immer eindeutig ist.
		\end{enumerate}
	\end{satz}
	\begin{satz}[Komplement]
		Für $U_1$ Untervektorraum von V existiert ein Untervektorraum $U_2$ von $V$ mit $U_1 \oplus U_2 = V$.
	\end{satz}
	\sep
	\begin{satz}[Dimension Untervektorraum]
		Ist $V$ endlich-dimensional und $U$ Untervektorraum von $V$, dann $\dim U\leq\dim V$.
		$\dim U=\dim V \iff U = V$.
	\end{satz}
	\begin{satz}[Dimension Summe]
		Ist $V$ endlich-dimensional und $U_1, U_2$ Untervektorräume von $V$, dann
		\[\dim(U_1 + U_2) = \dim U_1 + \dim U_2 - \dim(U_1 \cap U_2).\]
	\end{satz}
	\sep
	\begin{verfahren}[Linear unbhängige Teilmenge finden]
		Vektoren als Spaltenvektoren schreiben, Gauß-Algorithmus anwenden.
		Aus der Treppenform ergibt sich, welche Vektoren entfernt werden können.
	\end{verfahren}
	\begin{verfahren}[Einfache Basis der linearen Hülle finden]
		Vektoren als Zeilenvektoren schreiben, Gauß-Algorithmus anwenden.
		Die Zeilen, die nicht null sind, sind eine Basis der linearen Hülle.
	\end{verfahren}
	\begin{satz}
		Der Zeilenrang und Spaltenrang einer Matrix sind gleich.
	\end{satz}
	\sep
	\begin{definition}[Gruppenhomomorphismus]
		Für $(G, *)$, $(H, \circ)$ Gruppen und $\Phi\colon G \to H$:
		\[\Phi~\text{Gruppenhomomorphismus} :\Longleftrightarrow \forall x, y \in G: \Phi(x * y) = \Phi(x)\circ\Phi(y)\]
	\end{definition}
	\begin{definition}[Ringhomomorphismus]
		Für $(R_1, +, \cdot)$, $(R_2, +, \cdot)$ Ringe und $\Phi\colon R_1\to R_2$:
		\[\Phi~\text{Ringhomomorphismus} :\Longleftrightarrow \forall x, y \in R_1: \Phi(x + y) = \Phi(x) + \Phi(y) \wedge
		\Phi(x \cdot y) = \Phi(x) \cdot \Phi(y)\]
	\end{definition}
	\begin{definition}[Körperhomomorphismus]
		Für $(\mathbb{K}_1, +, \cdot)$, $(\mathbb{K}_2, +, \cdot)$ Körper und $\Phi\colon\mathbb{K}_1\to \mathbb{K}_2$:
		\[\Phi~\text{Körperhomomorphismus} :\Longleftrightarrow \forall x, y \in \mathbb{K}_1: \Phi(x + y) = \Phi(x) + \Phi(y) \wedge
		\Phi(x \cdot y) = \Phi(x) \cdot \Phi(y)\]
	\end{definition}
	\begin{definition}[Endomorphismus]
		Homorphismus, der gleichzeitig Selbstabbildung ist
	\end{definition}
	\begin{definition}[Isomorhpismus, Automorphismus]
		Bijektiver Homomorphismus bzw. bijektiver Endomorphismus
	\end{definition}
\end{document}
